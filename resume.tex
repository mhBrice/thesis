\francais

\chapter*{Résumé}

Le principal objectif de ma thèse était de comprendre et de quantifier
l'effet combiné du changement climatique récent et des perturbations sur
la dynamique des communautés forestières de l'écotone boréal-tempéré au
cours des dernières décennies. Pour ce faire, j'ai analysé les
changements au niveau de la composition en espèces, de la dynamique de
transition et du recrutement dans les forêts du Québec de 1970 à 2018.

Dans le premier chapitre, j'ai montré que les perturbations naturelles
et anthropiques étaient les principaux moteurs des changements de
composition, i.e. la diversité \(\beta\) temporelle. Malgré la
prévalence des perturbations, l'analyse des traits écologiques de la
communauté a révélé une thermophilisation des forêts à travers le
Québec, i.e.~une augmentation des espèces de climat chaud au détriment
des espèces de climat froid. Ce phénomène de thermophilisation a même
été amplifié par les perturbations modérées, soulevant une nouvelle
question: si les perturbations peuvent favoriser une telle
réorganisation des communautés, pourraient-elles catalyser un
basculement des forêts vers des états alternatifs?

Le second chapitre a apporté des réponses à cette question en analysant
la dynamique de transition des forêts du Québec avec un modèle à quatre
états, soit boréal, mixte, tempéré et pionnier. La dynamique de
transition était principalement influencée par les perturbations et
secondairement par le climat et les conditions édaphiques. Les
perturbations majeures ont entraîné surtout des transitions vers l'état
pionnier, tandis que les perturbations modérées ont favorisé les
transitions de mixte à tempéré. À long terme, les perturbations modérées
pourraient catalyser un déplacement plus rapide de l'écotone
boréal-tempéré vers le nord sous l'effet du changement climatique.
Toutefois, le recrutement des espèces tempérées a joué un rôle
négligeable dans cette dynamique comparativement aux processus de
mortalité et de croissance.

Les deux premiers chapitres s'appuient sur des analyses de l'évolution
des communautés d'arbres matures. Cependant, pour comprendre la
dynamique forestière, une analyse de la dynamique de régénération est
nécessaire, étude que j'ai menée au troisième chapitre. J'ai d'abord mis
en lumière des déplacements de plusieurs kilomètres vers le nord pour
les gaulis de \emph{Acer rubrum}, \emph{Acer saccharum} et \emph{Betula
alleghaniensis} dans les forêts non perturbées. Toutefois, sous
l'influence des perturbations modérées, seuls les \emph{Acer spp.} ont
migré; aucune espèce ne s'est déplacée sous l'influence des
perturbations majeures. En revanche, les gaulis de \emph{Fagus
grandifolia} n'ont pas du tout migré. Bien que les coupes partielles
aient favorisé une augmentation du recrutement des quatre espèces, elles
n'ont pas entraîné de migration plus au nord, possiblement parce que le
recrutement était freiné par une faible capacité de dispersion, une
forte compétition par les espèces boréales et des conditions édaphiques
défavorables.

Dans l'ensemble, mes résultats ont souligné que les communautés
forestières de l'écotone boréal-tempéré répondent déjà au changement
climatique récent et que les perturbations accélèrent cette réponse. En
effet, le réchauffement érode la résilience des forêts mixtes tandis que
les perturbations éliminent les espèces boréales en place, ce qui
accélère le processus de succession et facilite l'établissement des
espèces tempérées.

\vspace{1cm}

\textbf{Mots-clés}

Changement climatique, Diversité beta, Dynamique de transition, Écologie
des communautés, Écotone boréal-tempéré, Équilibre et dynamique
transitoire, Forêts, Inventaire forestier du Québec, Migration des
arbres, Perturbations naturelles et coupes forestières, Recrutement des
gaulis, Thermophilisation.
