%%
%% This is file `gabaritTPA.tex',
%% generated with the docstrip utility.
%%
%% The original source files were:
%%
%% dms.dtx  (with options: `TPA,gabarit')
%% Example TeX file for the documentation
%% of the jurabib package
%% Copyright (C) 1999, 2000, 2001 Jens Berger
%% See dms.ins  for the copyright details.
%%
%%% ====================================================================
%%%  @LaTeX-file{
%%%     filename        = "dms.dtx",
%%%     author    = "Nicolas Beauchemin, Damien Rioux-Lavoie, Victor Fardel, Jonathan Godin",
%%%     copyright = "Copyright (C) 2000 , DMS
%%%                  all rights reserved.  Copying of this file is
%%%                  authorized only if either:
%%%                  (1) you make absolutely no changes to your copy,
%%%                  including name; OR
%%%                  (2) if you do make changes, you first rename it
%%%                  to some other name.",
%%%     address   = "Département de Mathématiques et de Statistique",
%%%     telephone = "514-343-6705",
%%%     FAX       = "514-343-5700",
%%%     email     = "aide@dms.umontreal.ca (Internet)",
%%%     keywords  = "latex, amslatex, ams-latex, theorem",
%%%     abstract  = " Ce fichier est un package conçu pour être
%%%                  utilisé avec la version de LaTeX2e 1995/06/01. Il
%%%                  est prévue pour la classe ``amsbook''. Il en
%%%                  modifie le format des pages, l'entête des
%%%                  sections, etc, afin d'être  conforme au modèle de
%%%                  mémoire de maîtrise de l'Université de
%%%                  Montréal. Finalement ce fichier est grandement
%%%                  inspiré du fichier amsclass.dtx.",
%%%     docstring = "The checksum field contains: CRC-16 checksum,
%%%                  word count, line count, and character count, as
%%%                  produced by Robert Solovay's checksum utility."}
%%%  ====================================================================

%% Pour voir les accents de ce fichier, assurez-vous que votre
%% éditeur de texte lise le fichier en utf-8!

%% La classe <dms> est construite au-dessus de <amsbook>, donc
%% <amsmath>, <amsfonts> et <amsthm> sont automatiquement chargés.
\documentclass[12pt,twoside,phd]{dms}
\usepackage[utf8]{inputenc} %Obligatoires
\usepackage[T1]{fontenc}    %

%% <lmodern> incorpore les fontes en T1, pour
%% faciliter le dépôt final. Ceci n'est pas la
%% seule option :
%%  1. Si cm-super est installé, vous pouvez enlever <lmodern>
%%     (à ce moment, la police est un peu plus fidèle
%%      au Computer Modern orginal);
%%  2. Si vous avez une police préférée, par exemple,
%%     <times> ou <euler> ou <mathpazo> (et bien d'autres),
%%     alors vous pouvez remplacer <lmodern> ci-bas.
%% Par contre, si vous faîtes face à un problème d'encapsulation
%% lors dépôt final, il se peut que la solution soit d'utiliser <lmodern>.
%% (Parfois le problème est au niveau de l'installation, donc
%%  essayez de compiler sur un autre ordinateur sur lequel vous êtes
%%  certain·e que l'installation est bonne.)
\usepackage{lmodern}

%% Il n'est pas nécessaire d'utiliser <babel>, car
%% les commandes intégrées par la classe <dms>
%% \francais et \anglais font le travail. Néanmoins,
%% certains autres packages nécessitent <babel> (comme
%% <natbib>), donc simplement enlever les % devant <babel>
%% dans ce cas. Attention! Certains packages sont sensibles
%% à l'ordre dans lequel ils sont chargés.
\francais % or
%%\anglais
%% KC added
\usepackage[english,frenchb]{babel}
\usepackage{natbib}
\usepackage{longtable}
\usepackage{tabularx}
\usepackage{booktabs}
\usepackage{afterpage}
\usepackage{floatpag}
\usepackage{float}
\floatplacement{figure}{H}
\usepackage{caption}
\usepackage{listings}
\providecommand{\tightlist}{%
  \setlength{\itemsep}{0pt}\setlength{\parskip}{0pt}}
\usepackage{lscape}
\newcommand{\blandscape}{\begin{landscape}}
\newcommand{\elandscape}{\end{landscape}}
\usepackage{makecell}
\usepackage{colortbl}
\usepackage{xcolor}

 % ENGLISH OPTION
 % If you call \anglais here before the \begin{document},
 % all the chater's header will be in english, even if you
 % call \francais. To change this, use
 % \entetedynamique

%% La commande \sloppy peut avoir des effets étranges sur les
%% lignes de certains paragraphes.  Dans ce cas, essayez \fussy
%% qui suppresse les effets de \sloppy.
%% (\fussy est normalement le comportement par défaut.)
%% On redéfinit \sloppy, pour tenter de réduire les comportements
%% étranges. Le seul changement apporté à la version originale
%% est la valeur de \tolerance.
\def\sloppy{%
  \tolerance 500%  %9999 dans LaTeX ordinaire, mauvaise idée.
  \emergencystretch 3em%
  \hfuzz .5pt
  \vfuzz\hfuzz}
\sloppy   %appel de \sloppy pour le document
%%\fussy  %ou \fussy

%% Packages utiles.
\usepackage{graphicx,amssymb,subfigure,icomma}
%% icomma       permet d'écrire les nombres décimaux en
%%                  français (p.ex. 1,23 plutôt que 1.23)
%% subfigure    simplifie l'inclusion de figures côtes-à-côtes

%% Packages parfois utiles.
%%\usepackage{dsfont,mathrsfs,color,url,verbatim,booktabs}
%% dsfont       symboles mathématiques \mathds
%% mathrsfs     plus de symboles mathématiques \mathscr
%% color        pour utiliser des couleurs (comparer avec <xcolor>)
%% url          permet l'écriture d'url
%% verbatim     pour écrire du code ou du texte tel quel
%% booktabs     plus de macros pour faire les tableaux
%%                  (voir documentation du package)

%% pour que la largeur de la légende des figures soit = \textwidth
\usepackage[labelfont=bf, width=\linewidth]{caption}

%% les 3 lignes suivante servent à l'affichage de l'index
%% dans le visionneur de pdf. <hyperref> et <bookmark>
%% devraient être les dernier package a être chargé,
%% donc chargez vos packages avant.
\usepackage{hyperref}  % Ajoute les hyperlien
\hypersetup{colorlinks=true,allcolors=black}
\usepackage{hypcap}   % Corrige la position du lien pour les images
\usepackage{bookmark} % Remédie à des petits problème
                      % de <hyperref> (important qu'il
                      % apparaisse APRÈS <hyperref>)

  % Enlever les commentaires du prochaine \hypersetup et
  % le remplir avec l'information pertinente.
  % Ceci ajoute des « méta-données » au pdf.  C'est optionnel,
  % mais recommandé. Vous pouvez voir ces méta-données en
  % ouvrant un visionneur de pdf et en cherchant les propriétés
  % du pdf. (Vous pouvez aussi tapez ' pdfinfo <nom-du-pdf> '
  % dans un terminal.) Ces données sont utiles, par exemple,
  % pour augmenter les chances qu'un algorithme de recherche
  % trouve votre document sur Internet, une fois diffusé.
  \hypersetup{
    pdftitle = {Dynamique spatio-temporelle des forêts dans l'écotone boréal-tempéré en réponse aux changements globaux},
    pdfauthor = {Marie-Hélène Brice},
    pdfsubject = {Dynamique forestière},
    pdfkeywords = {écologie des communautés, changements climatiques, forêts, diversité bêta, dynamique de transition, recrutement, Québec}
  }

%% Définition des environnements utiles pour un mémoire scientifique.
%% La numérotation est laissée à la discrétion de l'auteur·e. L'exemple
%% illustré ici produit « Définition x.y.z » à l'extérieur d'un article
%%   x = no. chapitre
%%   y = no. section
%%   z = no. définition
%% et « Définition x » à l'intérieur d'un article
%%   x = no. définition
%% Les numérotations des corollaires, définitions, etc.
%% se font de façon successive.
%%
%% Les macros \<type>name sont telles qu'ils suivent
%% la langue actuelle. (P.ex. si \francais est utilisé,
%% alors \begin{theo} va faire un Théorème et si \anglais
%% est utilisé, \begin{theorem} fera un Theorem.)
%%
  % Environnement à utiliser à l'extérieur des articles
\newtheorem{cor}{\corollaryname}[section]
\newtheorem{deff}[cor]{\definitionname}
\newtheorem{ex}[cor]{\examplename}
\newtheorem{lem}[cor]{\lemmaname}
\newtheorem{prop}[cor]{Proposition}
\newtheorem{rem}[cor]{\remarkname}
\newtheorem{theo}[cor]{\theoremname}

  % Environnement à utiliser à l'intérieur des articles
\newtheorem{corA}{\corollaryname}
\newtheorem{deffA}[corA]{\definitionname}
\newtheorem{exA}  [corA]{\examplename}
\newtheorem{lemA} [corA]{\lemmaname}
\newtheorem{propA}[corA]{Proposition}
\newtheorem{remA} [corA]{\remarkname}
\newtheorem{theoA}[corA]{\theoremname}
%% IMPORTANT : Il faut faire \setcounter{corA}{0}
%% au début d'un article pour recommancer à compter à 1.
%%
%% NOTE : Il peut être commode de redéfinir \the<type> pour
%% obtenir la numérotation désirée. Par exemple, pour
%% que les corollaires soit numérotés #article.#section.#sous-section,
%% on fait
%% \renewcommand\thecorA{\thepart.\thesubsection.\arabic{corA}}

%%%
%%% Si vous préférez que les corollaires, définitions, théorèmes,
%%% etc. soient numérotés séparément, utilisez plutôt un bloc de
%%% commandes de la forme :
%%%

%%\newtheorem{cor}{\corollaryname}[section]
%%\newtheorem{deff}{\definitionname}[section]
%%\newtheorem{ex}{\examplename}[section]
%%\newtheorem{lem}{\lemmaname}[section]
%%\newtheorem{prop}{Proposition}[section]
%%\newtheorem{rem}{\remarkname}[section]
%%\newtheorem{theo}{\theoremname}[section]

%%
%% Numérotation des équations par section
%% et des  tableaux et figures par chapitre.
%% Ceci peut être modifié selon les préférences de l'utilisateur.
\numberwithin{equation}{section}
\numberwithin{table}{chapter}
\numberwithin{figure}{chapter}

%%
%% Si on veut faire un index, il faut décommenter la ligne
%% suivante. Ajouter des mots à l'index avec la commande \index{mot cle} au
%% fur et à mesure dans le texte.  Compiler, puis taper la commande
%% makeindex pour creer les indexs.  Après une nouvelle compilation,
%% vous aurez votre index.
%%

%%\makeindex

%% Il est obligatoire d'écrire à double interligne
%% ou à interligne et demi. On peut soit utiliser
%% le package <setspace> ou \baselinestretch.
%% Le package est un peu plus propre, mais le choix
%% reste à la discrétion de l'usager.
\usepackage[onehalfspacing]{setspace}
 % ou
%%\renewcommand{\baselinestretch}{1.5}


%%%%%%%%%%%%%%%%%%%%%%%%%%%%%%%%%%%%%%%%%%%%%%%%%%%%%%%%%%%%
%%%%%%%%%%%%%%%%%%%%%%%%%%%%%%%%%%%%%%%%%%%%%%%%%%%%%%%%%%%%
%%%%%%%%%%                                     %%%%%%%%%%%%%
%%%%%%%%%% D é b u t    d u    d o c u m e n t %%%%%%%%%%%%%
%%%%%%%%%%                                     %%%%%%%%%%%%%
%%%%%%%%%%%%%%%%%%%%%%%%%%%%%%%%%%%%%%%%%%%%%%%%%%%%%%%%%%%%
%%%%%%%%%%%%%%%%%%%%%%%%%%%%%%%%%%%%%%%%%%%%%%%%%%%%%%%%%%%%
\begin{document}

%%
%% Voici des options pour annoter les différentes versions de votre
%% mémoire. La commande \brouillon imprime, au bas de chacune des pages, la
%% date ainsi que l'heure de la dernière compilation de votre fichier.
%%
%%\brouillon
%%
%%
%% \version est la version de votre manuscrit
%%
\version{1}

%%------------------------------------------------- %
%%              pages i et ii                       %
%%------------------------------------------------- %

%%%
%%% Voici les variables à définir pour les deux premières pages de votre
%%% mémoire.
%%%

\title{Dynamique spatio-temporelle des forêts dans l'écotone boréal-tempéré en réponse aux changements globaux}

\author{Marie-Hélène Brice}

\copyrightyear{2020}

\department{Département de sciences biologiques}

\date{\today} %Date du DÉPÔT INITIAL (ou du 2e dépôt s'il y a corrections majeures)

\sujet{sciences biologiques}
\orientation{option biodiversité, écologie et évolution}%Ce champ est optionnel
%%\orientation{orientation}%Ce champ est optionnel
%%
%% Voici les disciplines possibles (voir avec votre directeur):
%% \sujet{statistique},
%% \sujet{mathématiques}, \orientation{mathématiques appliquées},
%% \orientation{mathématiques fondamentales}
%% \orientation{mathématiques de l'ingénieur} et
%% \orientation{mathématiques appliquées}

\president{Pierre-Luc Chagnon}

\directeur{Pierre Legendre}

\codirecteur{Marie-Josée Fortin}

\membrejury{Steven Kembel}

\examinateur{Sylvie de Blois}

%%\codirecteur{Nom du 1er codirecteur}         % s'il y a lieu
%%\codirecteurs{Nom du 2e codirecteur}         % s'il y a lieu


%%\examinateur{Nom de l'examinateur externe}   %obligatoire pour la these

%% \membresjury{Deuxième membre du jury}  % s'il y a lieu

%%  \plusmembresjury{Troisième membre du jury}    % s'il y a lieu

 % Cette option existe encore, mais elle n'a plus sa place
 % dans la page titre. L'utiliser seulement si le directeur
 % insiste...
%%\repdoyen{Nom du représentant du doyen} %(thèse seulement)

%%
%% La commande \maketitle créera votre page titre.

\maketitle

 % Pour générer la deuxième page titre, il faut appeler à nouveau \maketitle
 % Cette page est obligatoire.
\maketitle

%%------------------------------------------------- %
%%              pages iii                           %
%%------------------------------------------------- %

 % Les articles peuvent être en anglais, mais
 % les autres parties du document doivent être
 % en français. Il faut une permission pour
 % écrire l'ensemble de la thèse en anglais.
 % Consulter le guide de présentation des mémoires
 % et des thèses pour de l'information plus
 % précise et à jour.
\francais

\chapter*{Résumé}

...sommaire et mots clés en français...

%%------------------------------------------------- %
%%              pages iv                            %
%%------------------------------------------------- %

\anglais
\chapter*{Abstract}

...summary and keywords in english...

%%------------------------------------------------- %
%%        page v --- Table de matieres              %
%%------------------------------------------------- %

\francais
 % \cleardoublepage termine la page actuel et force TeX
 % a poussé les éléments flottant (fig., tables, etc.) sur
 % la page (normalement TeX les garde en suspend jusqu'à ce
 % qu'il trouve un endroit approprié). Avec l'option <twoside>,
 % la commande s'assure que la prochaine page de texte est sur
 % le recto, pour l'impression. On l'utilise ici
 % pour que TeX sache que la table des matières etc. soit
 % sur la page qui suit.
%% TABLE DES MATIÈRES
\cleardoublepage
\pdfbookmark[chapter]{\contentsname}{toc}  % Crée un bouton sur
                                           % la bar de navigation
\tableofcontents
 % LISTE DES TABLES
\cleardoublepage
\phantomsection  % Crée une section invisible (utile pour les hyperliens)
\listoftables
 % LISTE DES FIGURES
\cleardoublepage
\phantomsection
\listoffigures

%%%%%%%%%%%%%%%%%%%%%%%%%%%%%%%%%%%%%
%% LISTE DES SIGLES ET ABRÉVIATION %
%%%%%%%%%%%%%%%%%%%%%%%%%%%%%%%%%%%%%
%% Il est obligatoire, selon les directives de la FESP,
%% pour une thèse ou un mémoire d'avoir une liste des sigles et
%% des abréviations.  Si vous considérez que de telles listes ne seraient pas
%% pertinentes (si, par exemple, vous n'utilisez aucun sigle ou abré.), son
%% inclusion ou omission est laissé à votre discrétion.  En cas de doute,
%% parlez-en à votre directeur de recherche, le coadministrateur ou au/à la
%% bibliothécaire.
%%
%% Le gabarit inclut un exemple d'une liste « fait à la main ».  Il existe des outils
%% plus sophistiqués si vous devez inclure une multitude de sigles et abréviations.
%% Par exemple, le package <glossaries> peut faire des index élaborés.  Comme
%% son utilisation est technique, il n'y a pas d'exemple directement dans ce gabarit.
%% On invite les gens qui aurait à l'utiliser à lire la documentation officielle,
%% soit en allant sur https://www.ctan.org/, soit en tapant dans un terminal :
%%
%% texdoc glossaries
%%

\chapter*{Liste des sigles et des abréviations}
 % Option de colonnes: definir \colun ou \coldeux
%%% Exemple
%%% \def\colun{\bf} % Première colonne en gras
%%% Pour numéroté les entrées, on peut faire
%%% \newcount\abbrlist
%%% \abbrlist=0
%%% \def\plusun{\global\advance\abbrlist by 1\relax}
%%% \def\colun{\plusun\the\abbrlist. }
%%\def\coldeux{\relax}
\begin{twocolumnlist}{.2\textwidth}{.7\textwidth}
  AIC & Critère d'information d'Akaike, de l'anglais \textit{Akaikes Information Criteria}\\
  DBH  & Diamètre hauteur poitrine (1.3m), de l'anglais
      \textit{Diameter at breast height} \\
  CTI  &  Indice de température de la communauté, de l'anglais
    \textit{Community  Temperature Index}\\
  GES & Gaz à effet de serre\\
  GIEC & Groupe d'experts intergouvernemental sur l'évolution du climat\\
  PCA & Analyse en composantes principales, de l'anglais \textit{Principal component Analysis}\\
  sd & Écart standard, de l'anglais \textit{standard deviation}\\
  SDM  &  Modèle de distribution des espèces, de l'anglais \textit{Species Distribution Model}\\
  STI  &  Indice de température des espèces, de l'anglais \textit{Species Temperature Index}\\
  TBI  &  Indice de diversité beta temporelle, de l'anglais \textit{Temporal Beta diversity Index}\\
\end{twocolumnlist}
%% L'environnement <threecolumnlist> existe aussi pour trois colonnes.

%%------------------------------------------------- %
%%              pages vi                            %
%%------------------------------------------------- %

\chapter*{Remerciements}

...remerciements...

 %
 % Fin des pages liminaires.  À partir d'ici, les
 % premières pages des chapitres ne doivent pas
 % être numérotées
 %

\NoChapterPageNumber
\cleardoublepage




%%%%%%%%%%%%%%%%%%%%%%%%%%%%%%%%%%%%%%%%%%%%%%%%%%%%%%%%%%%%
%%%%%%%%%%%%%%%%                           %%%%%%%%%%%%%%%%%
%%%%%%%%%%%%%%%%  I N T R O D U C T I O N  %%%%%%%%%%%%%%%%%
%%%%%%%%%%%%%%%%                           %%%%%%%%%%%%%%%%%
%%%%%%%%%%%%%%%%%%%%%%%%%%%%%%%%%%%%%%%%%%%%%%%%%%%%%%%%%%%%

\renewcommand\thefigure{0.\arabic{figure}}
\renewcommand\thetable{0.\arabic{table}}

% Options for packages loaded elsewhere
\PassOptionsToPackage{unicode}{hyperref}
\PassOptionsToPackage{hyphens}{url}
%
\documentclass[
]{article}
\usepackage{lmodern}
\usepackage{amssymb,amsmath}
\usepackage{ifxetex,ifluatex}
\ifnum 0\ifxetex 1\fi\ifluatex 1\fi=0 % if pdftex
  \usepackage[T1]{fontenc}
  \usepackage[utf8]{inputenc}
  \usepackage{textcomp} % provide euro and other symbols
\else % if luatex or xetex
  \usepackage{unicode-math}
  \defaultfontfeatures{Scale=MatchLowercase}
  \defaultfontfeatures[\rmfamily]{Ligatures=TeX,Scale=1}
\fi
% Use upquote if available, for straight quotes in verbatim environments
\IfFileExists{upquote.sty}{\usepackage{upquote}}{}
\IfFileExists{microtype.sty}{% use microtype if available
  \usepackage[]{microtype}
  \UseMicrotypeSet[protrusion]{basicmath} % disable protrusion for tt fonts
}{}
\makeatletter
\@ifundefined{KOMAClassName}{% if non-KOMA class
  \IfFileExists{parskip.sty}{%
    \usepackage{parskip}
  }{% else
    \setlength{\parindent}{0pt}
    \setlength{\parskip}{6pt plus 2pt minus 1pt}}
}{% if KOMA class
  \KOMAoptions{parskip=half}}
\makeatother
\usepackage{xcolor}
\IfFileExists{xurl.sty}{\usepackage{xurl}}{} % add URL line breaks if available
\IfFileExists{bookmark.sty}{\usepackage{bookmark}}{\usepackage{hyperref}}
\hypersetup{
  hidelinks,
  pdfcreator={LaTeX via pandoc}}
\urlstyle{same} % disable monospaced font for URLs
\usepackage[margin=1in]{geometry}
\setlength{\emergencystretch}{3em} % prevent overfull lines
\providecommand{\tightlist}{%
  \setlength{\itemsep}{0pt}\setlength{\parskip}{0pt}}
\setcounter{secnumdepth}{-\maxdimen} % remove section numbering
\usepackage{setspace}
\setstretch{1,5}
\usepackage{times}
\usepackage{lineno}

\date{}

\begin{document}

\hypertarget{introduction-guxe9nuxe9rale}{%
\section{Introduction générale}\label{introduction-guxe9nuxe9rale}}

L'humain est aujourd'hui une force prédominante gouvernant les processus
écologiques, amenant de nombreux chercheurs à suggérer que le système
terrestre a basculé dans une nouvelle ère géologique, l'Anthropocène
(Crutzen, 2002). Depuis environ un siècle, les activités humaines ont
largement perturbé l'équilibre dynamique des cycles naturels. Le
développement des sociétés occidentales s'est basé sur
l'industrialisation et l'exploitation des ressources, résultant
notamment en un changement d'utilisation des sols associé à la
fragmentation et la dégradation des habitats, ainsi qu'à un relargage
massif de gaz à effet de serre (e.g.~CO2, SOx, CH4, NOx\ldots) dans
l'atmosphère. Les changements environnementaux récents se caractérisent
par leur vitesse et leur intensité. La recherche scientifique
contemporaine s'intéresse à comprendre, évaluer et prédire l'impact de
ces perturbations sur les écosystèmes et les communautés (McGill et al.,
2015; Root et al., 2003; Sala et al., 2000; Vellend et al., 2017).

\hypertarget{changements-climatiques}{%
\section{Changements climatiques}\label{changements-climatiques}}

Le réchauffement climatique mesuré sur l'ensemble de la planète durant
les dernières décennies est sans équivoque, et la responsabilité de
l'humain par l'émission de gaz à effet de serre (abrégé GES par la
suite) est clairement établie (IPCC, 2014). Des projections récentes des
changements climatiques indiquent que les températures moyennes
mondiales pourraient augmenter de 2.6 à 4.8 °C d'ici la fin du XXIe
siècle dans le nord-est de l'Amérique du Nord, s'il n'y a pas de progrès
sur le contrôle des émissions de GES anthropiques (IPCC, 2014). Le
climat étant un déterminant important de la distribution des espèces, de
telles augmentations de température auront un impact majeur sur la
structure et les fonctions de tous les écosystèmes (Bellard et al.,
2012; Gauthier et al., 2015).

Selon les prévisions, les changements de température et de régimes de
précipitation devraient déplacer les niches climatiques optimales de
nombreuses espèces d'arbres vers le nord sur des centaines de kilomètres
(McKenney et al., 2007) ou plus haut en altitude d'une centaine de
mètres (Jump et al., 2009), modifiant la composition, la structure et la
diversité forestières (Price et al., 2013; Reich et al., 2015). Or, de
tels changements dans les forêts peuvent avoir des répercussions
environnementales considérables sur les fonctions et les services des
écosystèmes, tels que l'approvisionnement en bois et en produits
forestiers non ligneux, le stockage du carbone, le cycle des nutriments,
la purification de l'air et de l'eau et le maintien d'habitats pour la
faune et la flore (Mitchell et al., 2013; Mori et al., 2017). Ces
changements soulèvent aussi des enjeux socio-économiques majeurs. Par
exemple, comment adapter les stratégies de gestion forestière pour
assurer un approvisionnement durable en bois ? Ou encore, quel est
l'avenir de certaines espèces économiquement et culturellement
importantes, comme l'érable à sucre au Québec ? Comprendre et prédire
les conséquences de ces changements climatiques sur les écosystèmes
forestiers représente donc l'un des grands défis actuels pour la
communauté scientifique (Garcia et al., 2014; Pereira et al., 2010).

\hypertarget{ruxe9ponses-des-uxe9cosystuxe8mes-forestiers-aux-changements-climatiques}{%
\section{Réponses des écosystèmes forestiers aux changements
climatiques}\label{ruxe9ponses-des-uxe9cosystuxe8mes-forestiers-aux-changements-climatiques}}

\hypertarget{duxe9placements-des-aires-de-ruxe9partition}{%
\subsection{Déplacements des aires de
répartition}\label{duxe9placements-des-aires-de-ruxe9partition}}

Des changements de distribution liés au climat ont déjà été observés
pour de nombreuses espèces d'arbres à différentes échelles spatiales,
particulièrement dans les zones de transition où les changements sont
plus facilement détectables (Boulanger et al., 2017; Jump et al., 2009).
Par exemple, à l'échelle locale, Fisichelli et al. (2014) ont observé
une avancée de la régénération d'espèces arbres tempérées dans la forêt
boréale de la région à l'ouest des Grands Lacs et ce processus semblait
être facilité par des températures plus chaudes. Leithead et al. (2010)
ont observé que les trouées causées par la mort des arbres boréaux dans
une forêt du nord de l'Ontario facilitent l'établissement d'espèces
tempérées du sud. Les changements dans la composition forestière ont
aussi été observés dans les écotones altitudinaux sur une période de 40
ans; dans les Montagnes Vertes du Vermont, les arbres tempérés ont
progressé en altitude, conduisant à un déplacement des limites de
l'écotone boréal-tempéré d'environ 100 m (Beckage et al., 2008), tandis
que sur le Mont-Mégantic au sud du Québec, les arbres se sont déplacés
en élévation de près de 30 m en moyenne et les espèces de sous-bois de
près de 40 m (Savage \& Vellend, 2015). Mis ensemble, ces derniers
résultats indiquent qu'un décalage dans la répartition des deux strates
de végétation, et donc un changement de composition, est déjà en train
de se former dû à une différence entre la vitesse de réponse des espèces
de sous-bois et celle des arbres. À l'échelle régionale, Boisvert-Marsh
et al. (2014) et Sittaro et al. (2017) ont montré une migration à
prédominance vers le nord des essences d'arbres à travers le Québec,
avec les gaulis présentant une réponse plus rapide que les arbres
adultes.

\hypertarget{ruxe9ponses-duxe9mographiques}{%
\subsection{Réponses
démographiques}\label{ruxe9ponses-duxe9mographiques}}

Alors que de nombreuses études sur l'impact des changements climatiques
sur les forêts ont tenté de détecter ou de prédire les déplacements des
limites d'aires de répartition des espèces, comparativement peu d'études
ont examiné les changements à long terme des taux démographiques,
e.g.~mortalité, recrutement, croissance, ou ont exploré les facteurs
environnementaux responsables de ces changements (reviewed in Allen et
al., 2010). L'accent mis sur les déplacements des espèces vers les pôles
sous-estime l'empreinte des changements climatiques. Les changements de
température et de précipitations ont des effets directs sur la
croissance, la mortalité et le recrutement des arbres (Vanderwel et al.,
2013; Zhang et al., 2015). Par exemple, les augmentations récentes des
taux de mortalité des arbres dans l'ouest de l'Amérique du Nord ont été
attribuées à des températures élevées et des sécheresses (Peng et al.,
2011; van Mantgem et al., 2009). Or, c'est l'équilibre entre les gains
par la croissance et le recrutement et les pertes par la mortalité qui
détermine, localement, la dynamique des forêts et, régionalement, les
limites d'aires de répartition (Holt et al., 2005). Des changements même
très faibles dans les taux démographiques peuvent modifier le rapport de
force de la compétition interspécifique (Luo \& Chen, 2013; Reich et
al., 2015), de même que la dynamique et la trajectoire de succession des
forêts (Prach \& Walker, 2011), modifiant par conséquent leur structure
et leur composition (Stephenson et al., 2011; van Mantgem et al., 2009).
À long terme, ces changements démographiques agissent donc pour
contrôler les limites géographiques des différents types de forêts (Holt
et al., 2005).

Étant donné l'échelle temporelle à laquelle les changements de
répartition se produisent pour des organismes à longue durée de vie
comme les arbres, comprendre l'influence des conditions abiotiques et
biotiques sur les taux démographiques des populations offre une
meilleure perspective sur la biogéographie des espèces et permet
d'inférer les changements continus à la limite et à l'intérieur de
l'aire de répartition (Schurr et al., 2012; Sexton et al., 2009; Snell
et al., 2014; Thuiller et al., 2013). Pourtant, à l'heure actuelle, il
existe très peu d'informations quantitatives sur l'effet combiné des
changements climatiques et des multiples perturbations forestières sur
ces processus fondamentaux de la dynamique forestière.

\hypertarget{duxe9lais-de-ruxe9ponse-et-duxe9suxe9quilibre}{%
\section{Délais de réponse et
déséquilibre}\label{duxe9lais-de-ruxe9ponse-et-duxe9suxe9quilibre}}

Bien qu'on prévoie un déplacement des niches climatiques des arbres de
plusieurs centaines de kilomètres vers le nord d'ici la fin du siècle
(McKenney et al., 2007), un nombre croissant d'études suggèrent que le
déplacement des arbres en Amérique du Nord ne réussira probablement pas
à suivre le rythme du réchauffement climatique (Sittaro et al., 2017;
Vissault, 2016; Woodall et al., 2013; Zhu et al., 2012). Par exemple,
malgré qu'on observe un déplacement des aires de répartition des arbres
vers le nord, les vitesses de migration des espèces d'arbres au Québec
étaient en moyenne inférieures à 50\% de la vitesse d'avancée
géographique des changements climatiques récents (Sittaro et al., 2017).

Les arbres sont particulièrement susceptibles de montrer de longs délais
de réponse aux changements parce que ces espèces sont sessiles, ont une
faible capacité de dispersion, une longue durée de vie, une croissance
lente et une maturité sexuelle tardive (Iverson \& McKenzie, 2013;
Lenoir \& Svenning, 2015). Ces caractéristiques pourraient expliquer le
haut niveau d'inertie des forêts malgré les changements de climat
(Vissault, 2016). En effet, ces caractéristiques peuvent expliquer
l'absence de colonisation à la limite nord malgré que les conditions
soient devenues favorables et engendrer un crédit de colonisation.
Inversement, les espèces peuvent persister pendant un certain temps dans
un milieu nouvellement inadapté en raison du délai d'extinction ou
peuvent être maintenues grâce à une dynamique source-puit, engendrant
une dette d'extinction (Jackson \& Sax, 2010; Pulliam, 2000; Schurr et
al., 2012). Ainsi, étant donné que l'environnement est dynamique et que
les écosystèmes forestiers sont caractérisés par d'importants délais de
colonisation et d'extinction, les systèmes perturbés ne parviennent
souvent pas à un équilibre statistique sur des échelles temporelles et
spatiales réalistes pour permettre une analyse statique (e.g., SDM). Il
y a donc un décalage entre la niche Hutchinsonienne et la répartition
géographique d'une espèce (Holt, 2009). Pourtant, la majorité des
modèles de distribution d'espèces suppose que les espèces sont en
équilibre avec leur environnement, ignorant la dynamique de transition.

D'importants délais de réponse des forêts aux changements climatiques
sont déjà observables puisque la distribution de plusieurs espèces
d'arbres de l'est de l'Amérique du Nord n'est pas à l'équilibre avec le
climat aux marges de leur aire de répartition, avec davantage de dettes
d'extinction au sud et de crédits de colonisation au nord (Talluto et
al., 2017). Leurs résultats montrent aussi que la vitesse de la
contraction d'aire de répartition dans le sud est plus rapide que
l'expansion dans le nord (Talluto et al., 2017)). Des simulations ont
aussi montré que le décalage entre la niche climatique optimale des
espèces tempérées et leur distribution réalisée ne fera que s'accroître
avec le temps (Vissault, 2016). Cette tension grandissante entre la
distribution réalisée et potentielle des espèces risque d'autant plus de
causer des changements brusques (regime shift) dans les écosystèmes
forestiers suite à une perturbation anthropique ou naturelle (Renwick \&
Rocca, 2015; Vanderwel \& Purves, 2014).

\hypertarget{contraintes-uxe0-la-migration}{%
\section{Contraintes à la
migration}\label{contraintes-uxe0-la-migration}}

\hypertarget{interactions-biotiques}{%
\subsection{Interactions biotiques}\label{interactions-biotiques}}

Alors que le climat est un déterminant majeur de la niche des espèces,
des facteurs non climatiques, tels que les interactions biotiques,
imposent des contraintes supplémentaires à la migration des espèces.
Bien qu'elles répondent de manière indépendante, les espèces ne sont pas
isolées, mais interagissent avec les membres de leur communauté. Il est
généralement admis que les facteurs déterminant la distribution sont
spatialement hiérarchisés, de sorte que le climat régirait la
répartition à l'échelle régionale, alors que les interactions biotiques
seraient plus importantes à l'échelle locale (Soberón, 2007). De plus,
le climat contraindrait la distribution et l'abondance des espèces à
leur limite nord, tandis que le rôle des interactions serait plus
important à la limite sud et à l'intérieur de l'aire de répartition, là
où les conditions environnementales sont plus favorables (Louthan et
al., 2015). Un bon exemple de ce phénomène est la distribution de
l'épinette noire, une espèce ayant une niche écologique très large, mais
dont la distribution au sud est limitée aux sites où la compétition est
faible \textbf{(Loehle 1998)}, comme des sites à drainage très mauvais
ou excessif. Toutefois, avec les changements climatiques, les conditions
favorables se déplacent et forcent de nouvelles interactions à la marge
des aires de distributions (Kissling \& Schleuning, 2015). Par exemple,
à moins qu'un dépérissement massif de la forêt ne se produise, les
espèces tempérées qui migreront dans les forêts boréales devront
s'établir sur des sites qui sont déjà colonisés par d'autres espèces et
devront donc vraisemblablement compétitionner pour les ressources lors
de leur établissement (phénomène appelé l'effet prioritaire; Gilman et
al., 2010). Plus la compétition par les espèces résidentes sera forte,
plus la probabilité de colonisation par les espèces migratrices
diminuera, car ces dernières parviendront difficilement à s'installer
(Cazelles, Mouquet, et al., 2016), d'où l'importance potentielle des
interactions dans la distribution à grande échelle. Une étude de
simulation a d'ailleurs révélé que les taux de migration sont plus
faibles dans les forêts établies que dans les forêts de début de
succession, et lorsque la diversité est grande (Meier et al., 2012).
Ainsi, les espèces de début de succession ont des taux de migration plus
rapides que les espèces de fin de succession puisque ces dernières
colonisent principalement les habitats forestiers déjà colonisés où la
compétition interspécifique est plus élevée Meier et al. (2012){]}. Les
interactions biotiques sont de plus en plus reconnues comme étant un
facteur clé influençant la distribution des espèces à grande échelle
(Blois et al., 2013; Cazelles, Araújo, et al., 2016; Meier et al., 2010;
Svenning et al., 2014; Wisz et al., 2013).

Comme la répartition géographique d'une espèce dépend de nombreux
facteurs environnementaux, ainsi que des limites de dispersion et des
contingences historiques (Godsoe, 2010; Holt, 2009; Pulliam, 2000), il
peut s'avérer difficile sur le plan technique de trouver des preuves de
l'effet des interactions entre espèces sur la distribution. Les limites
physiologiques des espèces (et donc l'hétérogénéité environnementale)
influencent les interactions biotiques puisqu'elles déterminent le pool
d'espèces qui peuvent potentiellement cohabiter à un endroit donné. Les
patrons de cooccurrence des espèces en compétition sont en partie dus au
hasard, déterminés par qui est arrivé le premier et par des facteurs
aléatoires qui donnent un avantage initial. L'influence de l'effet
prioritaire et de l'hétérogénéité environnementale sur la répartition
actuelle des espèces rend donc difficile l'estimation de la force de
compétition à partir des patrons de cooccurrence ; une faible
cooccurrence peut refléter une faible compétition actuelle, mais une
forte compétition dans le passé, et inversement une forte cooccurrence
peut indiquer une faible compétition puisque les espèces coexistent,
mais aussi une forte compétition pour les mêmes ressources.

Bien que plusieurs études sur les déplacements d'aires de répartition
soulignent que les interactions biotiques risquent de réduire le succès
de migration, les preuves empiriques de leurs impacts sur les taux de
migration sont rares et indirectes (Svenning et al., 2014). Les
interactions interspécifiques représentent donc un facteur inconnu clé
dans les études sur le changement climatique. Une étude approfondie des
taux de recrutements et de mortalités pourrait permettre de tester et
quantifier l'importance du rôle joué par les interactions biotiques sur
la dynamique de transition et de migration des arbres.

\hypertarget{propriuxe9tuxe9s-du-sol}{%
\subsection{Propriétés du sol}\label{propriuxe9tuxe9s-du-sol}}

En plus de la compétition interspécifique, les espèces migratrices
coloniseront des sols qui sont déjà développés et qui présentent des
propriétés (e.g.~qualité du drainage, disponibilité en nutriments, pH,
mycorhizes) qui varient localement ou régionalement, lesquelles
pourraient retarder ou contraindre leur établissement (Brown \& Vellend,
2014; Goldblum \& Rigg, 2010; Lafleur et al., 2010). Les forêts dominées
par les conifères au nord où la température moyenne est froide
présentent généralement des sols plus acides et conduisent à une
activité microbienne plus faible et à une décomposition plus lente de la
matière organique que les forêts tempérées décidues plus chaudes du sud
(Goldblum \& Rigg, 2010). Par exemple, Collin et al. (2017) ont montré
que l'acidité du sol forestier sous une canopée dominée par les
conifères affecte négativement les semis de l'érable à sucre via un
débalancement nutritif foliaire, ce qui pourrait donc freiner sa
migration dans la forêt boréale. Toutefois, les espèces forestières, par
leur effet sur la qualité chimique de la litière (C, N, Mg) et sur la
composition des microorganismes du sol, peuvent elles-mêmes modifier les
taux de décomposition de la matière organique et la disponibilité des
éléments nutritifs (Laganière et al., 2010). Les espèces migratrices
pourraient donc influencer leur propre taux d'invasion. Plusieurs études
menées dans le nord-est du Canada ont montré une colonisation rapide des
peuplements résineux par le peuplier faux-tremble après une perturbation
par la coupe forestière ou les feux (Chen et al., 2009; Laquerre et al.,
2009). La présence de cette espèce décidue induit des changements
physicochimiques et accélère les taux de décomposition de la matière
organique (Laganière et al., 2010; Légaré et al., 2005). À leur tour,
ces conditions de sol modifiées pourraient favoriser l'établissement et
la persistance de nouvelles espèces migratrices.

En plus des facteurs endogènes (traits, démographie lente et dispersion
limitée), la compétition par les espèces résidentes et les contraintes
imposées par les propriétés des sols résidents sur les plantes
migratrices sont des facteurs exogènes qui peuvent également contribuer
aux déséquilibres observés entre la niche climatique et la répartition
des espèces, particulièrement par un crédit de colonisation. La
compréhension de l'effet de la compétition et des sols sur les plantes
migratrices est donc essentielle pour prédire la redistribution des
espèces sous le changement climatique.

\hypertarget{interaction-entre-changements-climatiques-et-perturbations}{%
\subsection{Interaction entre changements climatiques et
perturbations}\label{interaction-entre-changements-climatiques-et-perturbations}}

Malgré l'empreinte indéniable des changements climatiques, la réponse
récente des écosystèmes forestiers n'est pas aussi unidirectionnelle que
prévu, car elle dépend de nombreux facteurs qui peuvent interagir entre
eux; ainsi les répercussions à long terme demeurent encore difficiles à
prévoir. Ajoutés aux effets des changements climatiques sur la
performance des arbres, sont les effets des perturbations naturelles à
grande échelle, notamment les feux de forêt et les épidémies d'insectes
(Bergeron et al., 2017; Boulanger et al., 2017; Gauthier et al., 2015;
Keane et al., 2013), qui peuvent déclencher des altérations rapides dans
la succession végétale et par conséquent dans les fonctions des
écosystèmes. De la même façon, les activités forestières peuvent
également interagir fortement avec les impacts liés aux changements
climatiques en modifiant la structure et la composition des forêts
(Bergeron et al., 2017; Boulanger et al., 2017; Scheller \& Mladenoff,
2005). Par exemple, entre 1930 et 2002, la coupe forestière dans une
région à la limite nord des espèces tempérées au Québec a engendré un
changement majeur de composition; près de 40 \% du paysage est passé
d'un couvert coniférien à un couvert mixte et près de 20 \% est devenu
feuillu (Boucher et al. 2006).

Les perturbations, autant naturelles qu'anthropiques, devraient avoir
une forte influence sur la façon dont les forêts répondent aux
changements climatiques car elles peuvent offrir des opportunités de
colonisation, changer le rapport de force de compétition entre les
espèces d'arbres pionnières et de fin de succession et faciliter
l'expansion des espèces tempérées vers le nord capables de profiter des
ouvertures de la canopée (Vanderwel \& Purves, 2014; Woodall et al.,
2013; Xu et al., 2012). Une perturbation combinée aux changements dans
les conditions climatiques peut changer la trajectoire successionnelle
de la forêt et même la faire basculer vers un autre nouvel état altéré
persistant (concept de regime shift), par exemple d'une forêt à
dominance conifèrienne à une forêt mixte, tel qu'observé par Boucher et
al. (2006). Les perturbations telles que les feux ou les coupes
pourraient alors agir comme des accélérateurs possibles de la migration
future de la forêt.

Face aux nombreuses perturbations et étant donné la longue échelle
temporelle des processus de dynamique forestière, il semble
incontournable que les forêts soient de plus en plus en déséquilibre.
Par conséquent, la réponse des forêts aux futurs changements climatiques
dépendra et interagira avec des dynamiques de transition déjà en cours.
Démêler les effets des changements climatiques et ceux des perturbations
naturelles et anthropiques et leurs rétroactions potentielles est
nécessaire à la fois pour informer les modèles prédictifs de
distribution de la biodiversité sous les changements climatiques et pour
élaborer des stratégies de gestion forestière permettant un aménagement
durable des forêts.

\hypertarget{enjeux-et-importances}{%
\section{Enjeux et importances}\label{enjeux-et-importances}}

La question des effets des CC sur la dynamique forestière soulève de
nombreux enjeux. La gestion de la biodiversité est un enjeu
majeur\ldots{} La gestion actuelle repose grandement sur une conception
statique de la biodiversité dans un climat stable. Par exemple,
l'aménagement écosystémique des forêts se base sur des états de
références historiques (Egan \& Howell, 2001), comme les forêts en place
avant la colonisation européenne et l'exploitation industrielle : les
forêts précoloniales ou préindustrielles. ``L'utilisation d'états de
références historiques pour l'aménagement écosystémique comporte des
limites. Dans le contexte de changements climatiques, une utilisation «
stricte » des caractéristiques d'écosystèmes du passé comme états de
référence pourrait aboutir à des écosystèmes forestiers non viables dans
le futur (Choi et al., 2008).'' De plus, l'aménagement forestier repose
également sur des modèle des possibilités forestières, ``lesquelles
correspondent au volume maximum des récoltes annuelles que l'on peut
prélever à perpétuité, sans diminuer la capacité productive du milieu
forestier.'' La calcul de possibilité forestière tient compte de
plusieurs critères tels que la dynamique naturelle des forêts, leur
composition, leur structure d'âge les aires de protection et la
probabilité de perturbation par les feux, les insectes et les maladies.
Cependant, ce calcul fait des prédictions à long terme et la coupe
forestière dépend de ces prédictions. Or, on devine que le changement
rapide du climat risque de bousculer ces prédictions. Par exemple, une
espèce pourrait ne pas se renouveler après coupe. Notre capacité à
prédire les effets futurs des changements climatiques sur la dynamique
forestière dépend de la description et de la compréhension de ses effets
passés et de son interaction avec les perturbations naturelles.

\hypertarget{contexte-climatique-et-uxe9cologique-du-quuxe9bec}{%
\section{Contexte climatique et écologique du
Québec}\label{contexte-climatique-et-uxe9cologique-du-quuxe9bec}}

\hypertarget{climat-du-quuxe9bec}{%
\subsection{Climat du Québec}\label{climat-du-quuxe9bec}}

Le climat du Québec est fortement marqué par le gradient latitudinal de
la température. Ce gradient de chaleur est le facteur le plus
déterminant pour la composition de la végétation du Québec. Ainsi on
aura, du sud vers le nord, un gradient de biodiversité qui reflète
étroitement celui de la température moyenne.

\hypertarget{vuxe9guxe9tation-et-domaines-bioclimatiques}{%
\subsection{Végétation et domaines
bioclimatiques}\label{vuxe9guxe9tation-et-domaines-bioclimatiques}}

Sur une superficie totale de 1 667 712 km\^{}2, ses forêts couvrent 761
100 km\^{}2, soit près de la moitié du territoire. La nordicité de la
forêt québécoise a comme conséquence la dominance des forêts résineuses
sur une grande partie du territoire et la faible diversité en espèces
d'arbres. En raison du fort gradient de température, les types de forêt
sont également structurés latitudinalement.

La forêt boréale occupe environ 72 \% du territoire québécois et sa
dynamique repose sur les feux, les épidémies de tordeuse des bourgeons
de l'épinette, les trouées et les chablis. Elle est composée
majoritairement d'épinette noire et de sapin baumier, mais aussi de pin
gris, de bouleau blanc et de peuplier faux-tremble

La forêt boréale 551 400 km\^{}2, la forêt mélangée 98 600 km2 et la
forêt feuillue 111 100 km2

PRINCIPALES ESSENCES D'ARBRES Forêt boréale : épinette noire, sapin
baumier et bouleau blanc. Forêt mélangée : bouleau jaune et sapin
baumier. Forêt feuillue : érable à sucre et bouleau jaune.

\hypertarget{changements-climatiques-au-quuxe9bec}{%
\subsection{Changements climatiques au
Québec}\label{changements-climatiques-au-quuxe9bec}}

\hypertarget{uxe9tat-des-foruxeats-du-quuxe9bec-muxe9ridional}{%
\subsection{État des forêts du Québec
méridional\ldots{}}\label{uxe9tat-des-foruxeats-du-quuxe9bec-muxe9ridional}}

\hypertarget{objectifs}{%
\section{Objectifs}\label{objectifs}}

Le principal objectif de cette thèse est de comprendre l'influence des
changements climatiques et des perturbations sur les changements à long
terme dans les écosystèmes forestiers tempérés. En utilisant les données
d'inventaires forestiers du Québec méridional de 1970 à 2018, cette
thèse s'articule autour de trois grandes questions :

\begin{enumerate}
\def\labelenumi{(\arabic{enumi})}
\item
  Comment les changements dans les patrons spatio-temporels de mortalité
  et de recrutement des arbres ont-ils influencé la diversité et la
  composition des forêts boréales et tempérées au cours des dernières
  décennies ?
\item
  Et quelle est l'importance relative des facteurs liés au climat, aux
  perturbations humaines (coupe, pollution) et naturelles (épidémie,
  feu) et aux caractéristiques du peuplement qui influencent la
  mortalité des arbres ?
\item
  Est-ce que les interactions compétitives entre les espèces d'arbres
  influencent leur taux de recrutement et de mortalité ? De ces
  questions en découle une autre très intéressante, à savoir quelle est
  l'influence des changements climatiques récents combinée aux effets
  des perturbations sur la trajectoire des communautés forestières.
\end{enumerate}

Chacun de ces objectifs est traité dans un chapitre de cette thèse
(chapitres 2, 3 et 4). Les réponses à ces questions sont essentielles
pour comprendre les relations entre les mécanismes locaux (interactions
entre espèces) et régionaux (contraintes environnementales) qui
sous-tendent les réponses des communautés aux changements
environnementaux. L'étude des variations spatio-temporelles de la
distribution des espèces apportera donc de nouvelles informations très
utiles sur l'importance relative de ces divers mécanismes. Cette étude
permettra ainsi de mettre en évidence le lien entre la dynamique
forestière et les changements climatiques, en tenant compte des
perturbations forestières, des interactions compétitives et des diverses
contraintes à la migration.

\hypertarget{sections-suivantes-uxe0-intuxe9grer-plus-haut}{%
\section{sections suivantes à intégrer plus
haut}\label{sections-suivantes-uxe0-intuxe9grer-plus-haut}}

\hypertarget{chapitre-1-patrons-de-diversituxe9-buxeata-temporelle}{%
\subsection{Chapitre 1 : Patrons de diversité bêta
temporelle}\label{chapitre-1-patrons-de-diversituxe9-buxeata-temporelle}}

L'Homme est aujourd'hui la principale force gouvernant les processus
écologiques faisant entrer la terre dans une nouvelle ère géologique,
l'Anthropocène (Crutzen, 2002). Un nombre croissant de preuves révèle
une perte de biodiversité exceptionnellement rapide au cours des
derniers siècles, ce qui indique qu'une sixième extinction de masse est
déjà en cours (Ceballos et al., 2015). D'ailleurs, Ceballos et al.
(2015) soulignent qu'au-delà des extinctions globales des espèces, la
Terre connaît aussi un énorme épisode de déclin des populations, dont
les conséquences se répercuteront sur les fonctions et les services des
écosystèmes. Malgré tout, des métaanalyses récentes ont montré que bien
souvent, à l'échelle locale, la biodiversité ne diminue pas et peut même
parfois augmenter (Dornelas et al., 2014; Vellend et al., 2013). Bien
que ces résultats aient été vivement critiqués (Newbold et al.~2015;
Gonzalez et al.~2016), il reste clair que la diversité locale (diversité
α) peut montrer des tendances variées, déconnectées des tendances à plus
grande échelle, même face à une extinction de masse à l'échelle globale.
Dans tous les cas, il est généralement admis qu'il y a eu des
changements importants dans la composition des communautés (diversité β;
Vellend et al.~2013; Dornelas et al.~2014; Newbold et al.~2015),
impliquant à la fois des pertes et des gains d'espèces (Wardle et
al.~2011). Ainsi, afin de mieux comprendre l'effet des changements
anthropiques sur la biodiversité, nous devons examiner parallèlement la
diversité α et β, ainsi que les composantes sous-jacentes de ces
changements, les pertes et les gains d'espèces.

Des travaux récents ont attiré l'attention sur le gain en compréhension
lorsque la diversité β est partitionnée en ses composantes sous-jacentes
(Baselga, 2010; Legendre, 2014, 2019; Podani et al., 2013). De telles
analyses permettent de quantifier les contributions de différents
processus écologiques à la diversité β. Legendre \& Salvat (2015) ont
développé une méthode pour partitionner la diversité β temporelle en
composantes de pertes et de gains en espèces, et l'ont appliquée aux
communautés de mollusque se rétablissant après des essais nucléaires.
Cette méthode offre la possibilité de faire le lien entre les
changements de diversité et les changements démographiques dans les
communautés, puisque les pertes et les gains sont en fait des mortalités
et des recrutements lorsque calculés sur des données d'abondance, et des
extinctions et colonisations lorsque calculés sur des données de
présence-absence.

Le chapitre 1 de la thèse vise à répondre à deux objectifs principaux
liés à la fois aux tendances temporelles de la biodiversité dans les
forêts de l'écotone boréal-tempéré au cours des dernières décennies et à
l'application de nouvelles méthodes d'analyse de diversité β temporelle.
Spécifiquement : quelles sont les tendances temporelles de diversité α
et β des forêts ? Comment les forêts ont-elles changé en termes de
mortalités et de recrutements ? Est-ce que ces changements sont
constants pour différents groupes d'espèces et de régions ? Selon mes
hypothèses, il n'y aura pas de tendance temporelle particulière au
niveau de la diversité α. Inversement, il y aura une augmentation de la
diversité β au cours du temps qui sera provoquée principalement par une
augmentation des mortalités, attribuables à l'action concommitante de
multiples perturbations, qui ne sera pas compensée par des recrutements.

En utilisant les données d'inventaires forestiers du Québec méridional
(MFFP, 2016), ces questions seront étudiées, dans un premier temps, en
quantifiant les tendances temporelles de diversité α, mesurée comme un
changement dans la richesse locale, et de diversité β, mesurée comme un
changement dans la composition des communautés. Et dans un deuxième
temps, en analysant les composantes sous-jacentes d'un indice de
diversité β temporelle (TBI\,; Temporal Beta Diversity Index\,; Legendre
\& Salvat 2015), soit les mortalités et les recrutements. En accordant
une attention accrue aux tendances de la diversité β, ce travail pourra
révéler des tendances précédemment imperceptibles sous l'angle de la
diversité α seule et aidera à améliorer notre compréhension des réponses
de la biodiversité forestière aux multiples facteurs de stress
anthropiques qui se sont accélérés au cours des dernières décennies.

\hypertarget{chapitre-2-tendances-et-causes-de-mortalituxe9s-dans-les-foruxeats-tempuxe9ruxe9es}{%
\subsection{Chapitre 2 : Tendances et causes de mortalités dans les
forêts
tempérées}\label{chapitre-2-tendances-et-causes-de-mortalituxe9s-dans-les-foruxeats-tempuxe9ruxe9es}}

La mortalité et le recrutement des arbres sont les moteurs principaux de
la dynamique forestière à long terme et leur variation peut entraîner
des changements marqués dans la composition et la structure des
communautés. Cependant, nous avons actuellement peu d'informations
quantitatives sur la variation géographique de ces taux et l'importance
relative des causes de la mortalité des arbres.

Plusieurs études récentes ont montré une augmentation des taux de
mortalité des arbres dans le temps associée à l'augmentation des
températures et des sécheresses (Allen et al., 2010; Peng et al., 2011;
van Mantgem et al., 2009; van Mantgem \& Stephenson, 2007). Malgré
l'importance indéniable du climat à l'échelle régionale sur ces
tendances, étonnamment peu d'attention a été accordée aux autres causes
possibles de mortalités qui peuvent interagir avec le climat. Par
exemple, Dietze et al.~(2011) ont révélé que les polluants
atmosphériques (particulièrement les dépôts acides) avaient un effet
particulièrement élevé, plus grand que l'effet du climat, sur les taux
de mortalité des arbres des forêts de l'est de l'Amérique du Nord. De
même, les processus endogènes associés au développement des peuplements
forestiers, tels que le stade de succession et la compétition, peuvent
avoir une grande influence sur la dynamique, mais ont été largement
ignorés puisque de nombreuses études sur l'effet des changements
climatiques sur la mortalité excluent de facto les forêts qui ont été
perturbées. Des études dans l'Ouest Canadien ont ainsi montré que
l'effet des changements climatiques sur les tendances temporelles de
mortalité des arbres était nettement plus important dans les jeunes
forêts que dans les forêts matures (Luo \& Chen 2013; Zhang et
al.~2015). La contribution relative de ces facteurs pourrait aussi
varier entre la forêt boréale et la forêt tempérée puisque leur
dynamique naturelle est très différente; la dynamique des forêts
boréales est gouvernée par des perturbations à grandes échelles, comme
les feux, les épidémies et la coupe, tandis que la dynamique des forêts
tempérées est plutôt dominée par des perturbations très locales de type
trouée (Goldblum \& Rigg, 2010). La quantification des contributions
relatives de différentes causes de mortalité des arbres est cruciale non
seulement pour mieux comprendre et anticiper les changements dans la
dynamique forestière, mais aussi pour mieux informer les modèles qui se
basent sur la démographie.

Les taux typiques de mortalité des arbres sont faibles (de l'ordre de
0.1 à 2\% par l'année) de sorte qu'estimer leur variation de manière
fiable requiert de grands échantillons et de longues périodes
d'échantillonnage. Les inventaires forestiers représentent une occasion
unique d'étudier les changements démographiques à long terme. Dans le
chapitre 2, les tendances et les causes des changements démographiques
(mortalités et recrutements) des populations d'arbres au cours des
quatre dernières décennies seront analysées. Je m'intéresse à trois
questions : 1. Est-ce que les taux de mortalité ou de recrutement ont
changé systématiquement dans les forêts du Québec méridional au cours
des quatre dernières décennies ? 2. Si les taux démographiques ont
changé, les changements sont-ils constants entre les différents groupes
d'espèces (boréales, tempérées et pionnières) et entre les régions ? Et
enfin, 3. Quelles sont les causes probables de ces changements
démographiques et quelle est l'importance relative des effets endogènes
(développement du peuplement, succession) et des effets exogènes
(climat, perturbations naturelles et humaines) sur la mortalité
individuelle des arbres? Selon mes hypothèses, avec les changements
climatiques, on observera une augmentation du recrutement des espèces
tempérées particulièrement en zone de forêts mixtes, et, inversement,
une augmentation de la mortalité des espèces boréales. Toutefois, comme
la majorité des mortalités seront principalement causées par des
perturbations directes, comme la coupe et les feux, il y aura aussi une
augmentation des espèces pionnières.

La probabilité de mortalité annuelle sera analysée par un modèle de
régression logistique avec divers prédicteurs environnementaux, incluant
des variables liées aux caractéristiques du peuplement (âge, surface
terrière, densité), des variables climatiques (température annuelle
moyenne, précipitation annuelle totale) et des variables liées aux
perturbations humaines (e.g., coupes forestières) et naturelles
(épidémies d'insecte, feux).

\hypertarget{chapitre-3-influence-des-interactions-compuxe9titives-sur-la-dynamique-de-migration-des-arbres-en-foruxeats-tempuxe9ruxe9es}{%
\subsection{Chapitre 3 : Influence des interactions compétitives sur la
dynamique de migration des arbres en forêts
tempérées}\label{chapitre-3-influence-des-interactions-compuxe9titives-sur-la-dynamique-de-migration-des-arbres-en-foruxeats-tempuxe9ruxe9es}}

Bien qu'un important déplacement des niches climatiques soit anticipé
d'ici la fin du siècle, les approches de modélisation utilisées à ce
jour sont majoritairement incapables de projeter le rythme auquel les
espèces forestières répondront aux changements climatiques, car les
contraintes à la migration sont encore peu connues. Une des hypothèses
importantes avance que la compétition par les espèces résidentes
pourrait freiner l'établissement des espèces migratrices (Svenning et
al.~2014). Cependant, l'effet des interactions biotiques sur la
dynamique d'expansion d'aires de répartition des arbres n'a pas reçu
suffisamment d'attention et jusqu'à maintenant les études empiriques sur
le sujet sont principalement issues d'expériences de transplantation
(Hillerislambers et al.~2013; Brown \& Vellend 2014).

Les perturbations pourraient moduler la vitesse de réponse des forêts
aux changements climatiques en diminuant ou éliminant la compétition par
les espèces résidentes, créant ainsi des opportunités de colonisation
pour les espèces d'arbres tempérés (Xu et al.~2012; Woodall et al.~2013;
Vanderwel \& Purves 2014). Les perturbations pourraient donc accélérer
les changements de composition dans les forêts ou même les faire
basculer d'une dominance conifèrienne à une composition mixte (regime
shift). En plus de pouvoir accélérer la vitesse de transition, les
perturbations pourraient influencer différentiellement les espèces en
raison du compromis compétition-colonisation. Par exemple, les
compétiteurs inférieurs pourraient être favorisés (du moins à court
terme) grâce à leur meilleure capacité de colonisation leur permettant
d'atteindre les milieux récemment perturbés (Gilman et al.~2010). Pour
l'instant, l'effet des perturbations sur les capacités de colonisation
des espèces migratrices en réponse aux changements climatiques a été
étudié surtout par modélisation à l'échelle régionale (Scheller \&
Mladenoff 2005; Vanderwel \& Purves 2014) et empiriquement à l'échelle
locale (Leithead et al.~2010). À l'échelle régionale, Woodall et
al.~(2013) n'ont pas trouvé de différence dans les limites nord de
répartition des semis et des arbres adultes entre les sites perturbés et
non perturbés aux États-Unis, mais ils n'avaient que 5 ans d'intervalle.

S'il y a d'autres contraintes abiotiques que le climat aux limites nord
des aires de répartition des espèces, les projections de migration sous
les changements climatiques qui ignorent ces facteurs pourraient
surestimer l'effet des températures sur l'expansion des aires. Certaines
études expérimentales suggèrent, par exemple, que les propriétés du sol
peuvent freiner la germination et la croissance des semis (Brown \&
Vellend 2014; Eskelinen \& Harrison 2015; Collin et al.~2017). Ainsi,
même si les contraintes climatiques sont relâchées pour une espèce
donnée et que la compétition est éliminée par une perturbation, il est
possible que l'habitat ne soit tout de même pas adéquat (Beauregard \&
De Blois 2014). Ces contraintes biotiques liées à la compétition et
abiotiques liées aux propriétés du sol pourraient donc freiner ou
empêcher la migration des espèces tempérées et contribuer au
déséquilibre entre la répartition géographique et la niche climatique
potentielle des espèces.

Au chapitre 3, je vais évaluer l'importance relative des facteurs non
climatiques sur la dynamique de colonisation et d'extinction des arbres
dans les forêts tempérées, particulièrement à la limite de leur aire de
répartition. Je tenterai de répondre aux questions suivantes : Est-ce
que la compétition (mesurée par un effet densité-dépendant des espèces
résidentes sur l'espèce migratrice) influence les probabilités de
colonisation des arbres ? Ou est-ce plutôt l'effet des propriétés du sol
qui freine la probabilité de colonisation ? Si la compétition est
importante, est-ce que les perturbations permettent d'accélérer le taux
de recrutement des espèces tempérées à la limite nord de leur
distribution ? Selon mes hypothèses, l'effet combiné de la compétition
et des propriétés du sol sur les espèces tempérées migratrices sera
soutenu en marge de leur aire de répartition, mais les perturbations
faciliteront l'établissement de ces espèces en diminuant la compétition.
Aussi, cette diminution de la compétition avantagera différentiellement
les espèces selon leurs traits, et le gain sera plus grand pour les
espèces de début de succession.

Dans ce chapitre, un modèle basé sur la théorie des métapopulations et
des métacommunautés me permettra de modéliser la dynamique d'assemblage
des communautés locales en prenant en considération la manière dont la
compétition, le sol et les perturbations peuvent faciliter ou entraver
la migration des arbres via leurs effets sur la démographie. Le modèle
classique sera étendu de façon à ce que la probabilité de
colonisation/recrutement et d'extinction/mortalité d'une espèce soit
conditionnelle aux variables climatiques (Talluto et al.~2017),
édaphiques et à la composition de la communauté.

\hypertarget{refs}{}
\leavevmode\hypertarget{ref-allen_global_2010}{}%
Allen, C. D., Macalady, A. K., Chenchouni, H., Bachelet, D., McDowell,
N., Vennetier, M., Kitzberger, T., Rigling, A., Breshears, D. D., Hogg,
E. H. (., Gonzalez, P., Fensham, R., Zhang, Z., Castro, J., Demidova,
N., Lim, J.-H., Allard, G., Running, S. W., Semerci, A., \& Cobb, N.
(2010). A global overview of drought and heat-induced tree mortality
reveals emerging climate change risks for forests. \emph{Forest Ecology
and Management}, \emph{259}(4), 660--684.
\url{https://doi.org/10.1016/j.foreco.2009.09.001}

\leavevmode\hypertarget{ref-baselga_partitioning_2010}{}%
Baselga, A. (2010). Partitioning the turnover and nestedness components
of beta diversity. \emph{Global Ecology and Biogeography}, \emph{19}(1),
134--143. \url{https://doi.org/10.1111/j.1466-8238.2009.00490.x}

\leavevmode\hypertarget{ref-beckage_rapid_2008}{}%
Beckage, B., Osborne, B., Gavin, D. G., Pucko, C., Siccama, T., \&
Perkins, T. (2008). A rapid upward shift of a forest ecotone during 40
years of warming in the Green Mountains of Vermont. \emph{Proceedings of
the National Academy of Sciences}, \emph{105}(11), 4197--4202.
\url{https://doi.org/10.1073/pnas.0708921105}

\leavevmode\hypertarget{ref-bellard_impacts_2012}{}%
Bellard, C., Bertelsmeier, C., Leadley, P., Thuiller, W., \& Courchamp,
F. (2012). Impacts of climate change on the future of biodiversity:
Biodiversity and climate change. \emph{Ecology Letters}, \emph{15}(4),
365--377. \url{https://doi.org/10.1111/j.1461-0248.2011.01736.x}

\leavevmode\hypertarget{ref-bergeron_projections_2017}{}%
Bergeron, Y., Irulappa Pillai Vijayakumar, D. B., Ouzennou, H., Raulier,
F., Leduc, A., \& Gauthier, S. (2017). Projections of future forest age
class structure under the influence of fire and harvesting: Implications
for forest management in the boreal forest of eastern Canada.
\emph{Forestry: An International Journal of Forest Research},
\emph{90}(4), 485--495. \url{https://doi.org/10.1093/forestry/cpx022}

\leavevmode\hypertarget{ref-blois_climate_2013}{}%
Blois, J. L., Zarnetske, P. L., Fitzpatrick, M. C., \& Finnegan, S.
(2013). Climate Change and the Past, Present, and Future of Biotic
Interactions. \emph{Science}, \emph{341}(6145), 499--504.
\url{https://doi.org/10.1126/science.1237184}

\leavevmode\hypertarget{ref-boisvert-marsh_shifting_2014}{}%
Boisvert-Marsh, L., Périé, C., \& de Blois, S. (2014). Shifting with
climate? Evidence for recent changes in tree species distribution at
high latitudes. \emph{Ecosphere}, \emph{5}(7), art83.
\url{https://doi.org/10.1890/ES14-00111.1}

\leavevmode\hypertarget{ref-boucher_logging-induced_2006}{}%
Boucher, Y., Arseneault, D., \& Sirois, L. (2006). Logging-induced
change (1930-2002) of a preindustrial landscape at the northern range
limit of northern hardwoods, eastern Canada. \emph{Canadian Journal of
Forest Research}, \emph{36}(2), 505--517.
\url{https://doi.org/10.1139/x05-252}

\leavevmode\hypertarget{ref-boulanger_climate_2017}{}%
Boulanger, Y., Taylor, A. R., Price, D. T., Cyr, D., McGarrigle, E.,
Rammer, W., Sainte-Marie, G., Beaudoin, A., Guindon, L., \& Mansuy, N.
(2017). Climate change impacts on forest landscapes along the Canadian
southern boreal forest transition zone. \emph{Landscape Ecology},
\emph{32}(7), 1415--1431.
\url{https://doi.org/10.1007/s10980-016-0421-7}

\leavevmode\hypertarget{ref-brown_non-climatic_2014}{}%
Brown, C. D., \& Vellend, M. (2014). Non-climatic constraints on upper
elevational plant range expansion under climate change.
\emph{Proceedings of the Royal Society B: Biological Sciences},
\emph{281}(1794), 20141779--20141779.
\url{https://doi.org/10.1098/rspb.2014.1779}

\leavevmode\hypertarget{ref-cazelles_theory_2016}{}%
Cazelles, K., Araújo, M. B., Mouquet, N., \& Gravel, D. (2016). A theory
for species co-occurrence in interaction networks. \emph{Theoretical
Ecology}, \emph{9}(1), 39--48.
\url{https://doi.org/10.1007/s12080-015-0281-9}

\leavevmode\hypertarget{ref-cazelles_integration_2016}{}%
Cazelles, K., Mouquet, N., Mouillot, D., \& Gravel, D. (2016). On the
integration of biotic interaction and environmental constraints at the
biogeographical scale. \emph{Ecography}, \emph{39}(10), 921--931.
\url{https://doi.org/10.1111/ecog.01714}

\leavevmode\hypertarget{ref-ceballos_accelerated_2015}{}%
Ceballos, G., Ehrlich, P. R., Barnosky, A. D., Garcia, A., Pringle, R.
M., \& Palmer, T. M. (2015). Accelerated modern human-induced species
losses: Entering the sixth mass extinction. \emph{Science Advances},
\emph{1}(5), e1400253--e1400253.
\url{https://doi.org/10.1126/sciadv.1400253}

\leavevmode\hypertarget{ref-chen_wildfire_2009}{}%
Chen, H. Y. H., Vasiliauskas, S., Kayahara, G. J., \& Ilisson, T.
(2009). Wildfire promotes broadleaves and species mixture in boreal
forest. \emph{Forest Ecology and Management}, \emph{257}(1), 343--350.
\url{https://doi.org/10.1016/j.foreco.2008.09.022}

\leavevmode\hypertarget{ref-collin_conifer_2017}{}%
Collin, A., Messier, C., \& Bélanger, N. (2017). Conifer Presence May
Negatively Affect Sugar Maple's Ability to Migrate into the Boreal
Forest Through Reduced Foliar Nutritional Status. \emph{Ecosystems},
\emph{20}(4), 701--716. \url{https://doi.org/10.1007/s10021-016-0045-4}

\leavevmode\hypertarget{ref-crutzen_geology_2002}{}%
Crutzen, P. J. (2002). Geology of mankind. \emph{Nature},
\emph{415}(6867), 23--23. \url{https://doi.org/10.1038/415023a}

\leavevmode\hypertarget{ref-dornelas_assemblage_2014}{}%
Dornelas, M., Gotelli, N. J., McGill, B., Shimadzu, H., Moyes, F.,
Sievers, C., \& Magurran, A. E. (2014). Assemblage Time Series Reveal
Biodiversity Change but Not Systematic Loss. \emph{Science},
\emph{344}(6181), 296--299.
\url{https://doi.org/10.1126/science.1248484}

\leavevmode\hypertarget{ref-fisichelli_temperate_2014}{}%
Fisichelli, N. A., Frelich, L. E., \& Reich, P. B. (2014). Temperate
tree expansion into adjacent boreal forest patches facilitated by warmer
temperatures. \emph{Ecography}, \emph{37}(2), 152--161.
\url{https://doi.org/10.1111/j.1600-0587.2013.00197.x}

\leavevmode\hypertarget{ref-garcia_multiple_2014}{}%
Garcia, R. A., Cabeza, M., Rahbek, C., \& Araujo, M. B. (2014). Multiple
Dimensions of Climate Change and Their Implications for Biodiversity.
\emph{Science}, \emph{344}(6183), 1247579--1247579.
\url{https://doi.org/10.1126/science.1247579}

\leavevmode\hypertarget{ref-gauthier_vulnerability_2015}{}%
Gauthier, S., Bernier, P. Y., Boulanger, Y., Guo, J., Guindon, L.,
Beaudoin, A., \& Boucher, D. (2015). Vulnerability of timber supply to
projected changes in fire regime in Canada's managed forests.
\emph{Canadian Journal of Forest Research}, \emph{45}(11), 1439--1447.
\url{https://doi.org/10.1139/cjfr-2015-0079}

\leavevmode\hypertarget{ref-gilman_framework_2010}{}%
Gilman, S. E., Urban, M. C., Tewksbury, J., Gilchrist, G. W., \& Holt,
R. D. (2010). A framework for community interactions under climate
change. \emph{Trends in Ecology \& Evolution}, \emph{25}(6), 325--331.
\url{https://doi.org/10.1016/j.tree.2010.03.002}

\leavevmode\hypertarget{ref-godsoe_i_2010}{}%
Godsoe, W. (2010). I can't define the niche but I know it when I see it:
A formal link between statistical theory and the ecological niche.
\emph{Oikos}, \emph{119}(1), 53--60.
\url{https://doi.org/10.1111/j.1600-0706.2009.17630.x}

\leavevmode\hypertarget{ref-goldblum_deciduous_2010}{}%
Goldblum, D., \& Rigg, L. S. (2010). The Deciduous Forest - Boreal
Forest Ecotone. \emph{Geography Compass}, \emph{4}(7), 701--717.
\url{https://doi.org/10.1111/j.1749-8198.2010.00342.x}

\leavevmode\hypertarget{ref-holt_bringing_2009}{}%
Holt, R. D. (2009). Bringing the Hutchinsonian niche into the 21st
century: Ecological and evolutionary perspectives. \emph{Proceedings of
the National Academy of Sciences}, \emph{106}(Supplement\_2),
19659--19665. \url{https://doi.org/10.1073/pnas.0905137106}

\leavevmode\hypertarget{ref-holt_theoretical_2005}{}%
Holt, R. D., Keitt, T. H., Lewis, M. A., Maurer, B. A., \& Taper, M. L.
(2005). Theoretical models of species' borders: Single species
approaches. \emph{Oikos}, \emph{108}(1), 18--27.
\url{https://doi.org/10.1111/j.0030-1299.2005.13147.x}

\leavevmode\hypertarget{ref-ipcc_climate_2014}{}%
IPCC. (2014). \emph{Climate change 2014: Fifth Assessment Synthesis
Report. Contribution of Working Groups I, II and III to the Fifth
Assessment Report of the Intergovernmental Panel on Climate Change}.
\url{http://ar5-syr.ipcc.ch/}

\leavevmode\hypertarget{ref-iverson_tree-species_2013}{}%
Iverson, L. R., \& McKenzie, D. (2013). Tree-species range shifts in a
changing climate: Detecting, modeling, assisting. \emph{Landscape
Ecology}, \emph{28}(5), 879--889.
\url{https://doi.org/10.1007/s10980-013-9885-x}

\leavevmode\hypertarget{ref-jackson_balancing_2010}{}%
Jackson, S. T., \& Sax, D. F. (2010). Balancing biodiversity in a
changing environment: Extinction debt, immigration credit and species
turnover. \emph{Trends in Ecology \& Evolution}, \emph{25}(3), 153--160.
\url{https://doi.org/10.1016/j.tree.2009.10.001}

\leavevmode\hypertarget{ref-jump_altitude-for-latitude_2009}{}%
Jump, A. S., Mátyás, C., \& Peñuelas, J. (2009). The
altitude-for-latitude disparity in the range retractions of woody
species. \emph{Trends in Ecology \& Evolution}, \emph{24}(12), 694--701.
\url{https://doi.org/10.1016/j.tree.2009.06.007}

\leavevmode\hypertarget{ref-keane_exploring_2013}{}%
Keane, R. E., Cary, G. J., Flannigan, M. D., Parsons, R. A., Davies, I.
D., King, K. J., Li, C., Bradstock, R. A., \& Gill, M. (2013). Exploring
the role of fire, succession, climate, and weather on landscape dynamics
using comparative modeling. \emph{Ecological Modelling}, \emph{266},
172--186. \url{https://doi.org/10.1016/j.ecolmodel.2013.06.020}

\leavevmode\hypertarget{ref-kissling_multispecies_2015}{}%
Kissling, W. D., \& Schleuning, M. (2015). Multispecies interactions
across trophic levels at macroscales: Retrospective and future
directions. \emph{Ecography}, \emph{38}(4), 346--357.
\url{https://doi.org/10.1111/ecog.00819}

\leavevmode\hypertarget{ref-lafleur_response_2010}{}%
Lafleur, B., Paré, D., Munson, A. D., \& Bergeron, Y. (2010). Response
of northeastern North American forests to climate change: Will soil
conditions constrain tree species migration? \emph{Environmental
Reviews}, \emph{18}(NA), 279--289. \url{https://doi.org/10.1139/A10-013}

\leavevmode\hypertarget{ref-laganiere_how_2010}{}%
Laganière, J., Paré, D., \& Bradley, R. L. (2010). How does a tree
species influence litter decomposition? Separating the relative
contribution of litter quality, litter mixing, and forest floor
conditions. \emph{Canadian Journal of Forest Research}, \emph{40}(3),
465--475. \url{https://doi.org/10.1139/X09-208}

\leavevmode\hypertarget{ref-laquerre_augmentation_2009}{}%
Laquerre, S., Leduc, A., \& Harvey, B. D. (2009). Augmentation du
couvert en peuplier faux-tremble dans les pessières noires du nord-ouest
du Québec après coupe totale. \emph{Écoscience}, \emph{16}(4), 483--491.
\url{https://doi.org/10.2980/16-4-3252}

\leavevmode\hypertarget{ref-legendre_interpreting_2014}{}%
Legendre, P. (2014). Interpreting the replacement and richness
difference components of beta diversity. \emph{Global Ecology and
Biogeography}, \emph{23}(11), 1324--1334.
\url{https://doi.org/10.1111/geb.12207}

\leavevmode\hypertarget{ref-legendre_temporal_2019}{}%
Legendre, P. (2019). A temporal beta-diversity index to identify sites
that have changed in exceptional ways in space--time surveys.
\emph{Ecology and Evolution}, \emph{9}(6), 3500--3514.
\url{https://doi.org/10.1002/ece3.4984}

\leavevmode\hypertarget{ref-legendre_thirty-year_2015}{}%
Legendre, P., \& Salvat, B. (2015). Thirty-year recovery of mollusc
communities after nuclear experimentations on Fangataufa atoll (Tuamotu,
French Polynesia). \emph{Proceedings of the Royal Society B: Biological
Sciences}, \emph{282}(1810), 20150750.
\url{https://doi.org/10.1098/rspb.2015.0750}

\leavevmode\hypertarget{ref-leithead_northward_2010}{}%
Leithead, M. D., Anand, M., \& Silva, L. C. R. (2010). Northward
migrating trees establish in treefall gaps at the northern limit of the
temperate--boreal ecotone, Ontario, Canada. \emph{Oecologia},
\emph{164}(4), 1095--1106.
\url{https://doi.org/10.1007/s00442-010-1769-z}

\leavevmode\hypertarget{ref-lenoir_climate-related_2015}{}%
Lenoir, J., \& Svenning, J.-C. (2015). Climate-related range shifts - a
global multidimensional synthesis and new research directions.
\emph{Ecography}, \emph{38}(1), 15--28.
\url{https://doi.org/10.1111/ecog.00967}

\leavevmode\hypertarget{ref-legare_influence_2005}{}%
Légaré, S., Paré, D., \& Bergeron, Y. (2005). Influence of Aspen on
Forest Floor Properties in Black Spruce-dominated Stands. \emph{Plant
and Soil}, \emph{275}(1-2), 207--220.
\url{https://doi.org/10.1007/s11104-005-1482-6}

\leavevmode\hypertarget{ref-louthan_where_2015}{}%
Louthan, A. M., Doak, D. F., \& Angert, A. L. (2015). Where and When do
Species Interactions Set Range Limits? \emph{Trends in Ecology \&
Evolution}, \emph{30}(12), 780--792.
\url{https://doi.org/10.1016/j.tree.2015.09.011}

\leavevmode\hypertarget{ref-luo_observations_2013}{}%
Luo, Y., \& Chen, H. Y. H. (2013). Observations from old forests
underestimate climate change effects on tree mortality. \emph{Nature
Communications}, \emph{4}, 1655.
\url{https://doi.org/10.1038/ncomms2681}

\leavevmode\hypertarget{ref-mckenney_potential_2007}{}%
McKenney, D. W., Pedlar, J. H., Lawrence, K., Campbell, K., \&
Hutchinson, M. F. (2007). Potential Impacts of Climate Change on the
Distribution of North American Trees. \emph{BioScience}, \emph{57}(11),
939--948. \url{https://doi.org/10.1641/B571106}

\leavevmode\hypertarget{ref-meier_biotic_2010}{}%
Meier, E. S., Kienast, F., Pearman, P. B., Svenning, J.-C., Thuiller,
W., Araújo, M. B., Guisan, A., \& Zimmermann, N. E. (2010). Biotic and
abiotic variables show little redundancy in explaining tree species
distributions. \emph{Ecography}, \emph{33}(6), 1038--1048.
\url{https://doi.org/10.1111/j.1600-0587.2010.06229.x}

\leavevmode\hypertarget{ref-meier_climate_2012}{}%
Meier, E. S., Lischke, H., Schmatz, D. R., \& Zimmermann, N. E. (2012).
Climate, competition and connectivity affect future migration and ranges
of European trees: Future migration and ranges of European trees.
\emph{Global Ecology and Biogeography}, \emph{21}(2), 164--178.
\url{https://doi.org/10.1111/j.1466-8238.2011.00669.x}

\leavevmode\hypertarget{ref-mffp_placettes-echantillons_2016}{}%
MFFP. (2016). \emph{Placettes-échantillons permanentes: normes
techniques} (p. 236). Ministère des Forêts de la Faune et des Parcs,
Secteur des forêts, Direction des Inventaires Forestiers.
\url{http://collections.banq.qc.ca/ark:/52327/2748265}

\leavevmode\hypertarget{ref-mitchell_linking_2013}{}%
Mitchell, M. G. E., Bennett, E. M., \& Gonzalez, A. (2013). Linking
Landscape Connectivity and Ecosystem Service Provision: Current
Knowledge and Research Gaps. \emph{Ecosystems}, \emph{16}(5), 894--908.
\url{https://doi.org/10.1007/s10021-013-9647-2}

\leavevmode\hypertarget{ref-mori_biodiversity_2017}{}%
Mori, A. S., Lertzman, K. P., \& Gustafsson, L. (2017). Biodiversity and
ecosystem services in forest ecosystems: A research agenda for applied
forest ecology. \emph{Journal of Applied Ecology}, \emph{54}(1), 12--27.
\url{https://doi.org/10.1111/1365-2664.12669}

\leavevmode\hypertarget{ref-peng_drought-induced_2011}{}%
Peng, C., Ma, Z., Lei, X., Zhu, Q., Chen, H., Wang, W., Liu, S., Li, W.,
Fang, X., \& Zhou, X. (2011). A drought-induced pervasive increase in
tree mortality across Canada's boreal forests. \emph{Nature Climate
Change}, \emph{1}(9), 467--471.
\url{https://doi.org/10.1038/nclimate1293}

\leavevmode\hypertarget{ref-pereira_scenarios_2010}{}%
Pereira, H. M., Leadley, P. W., Proenca, V., Alkemade, R., Scharlemann,
J. P. W., Fernandez-Manjarres, J. F., Araujo, M. B., Balvanera, P.,
Biggs, R., Cheung, W. W. L., Chini, L., Cooper, H. D., Gilman, E. L.,
Guenette, S., Hurtt, G. C., Huntington, H. P., Mace, G. M., Oberdorff,
T., Revenga, C., \ldots{} Walpole, M. (2010). Scenarios for Global
Biodiversity in the 21st Century. \emph{Science}, \emph{330}(6010),
1496--1501. \url{https://doi.org/10.1126/science.1196624}

\leavevmode\hypertarget{ref-podani_general_2013}{}%
Podani, J., Ricotta, C., \& Schmera, D. (2013). A general framework for
analyzing beta diversity, nestedness and related community-level
phenomena based on abundance data. \emph{Ecological Complexity},
\emph{15}, 52--61. \url{https://doi.org/10.1016/j.ecocom.2013.03.002}

\leavevmode\hypertarget{ref-prach_four_2011}{}%
Prach, K., \& Walker, L. R. (2011). Four opportunities for studies of
ecological succession. \emph{Trends in Ecology \& Evolution},
\emph{26}(3), 119--123. \url{https://doi.org/10.1016/j.tree.2010.12.007}

\leavevmode\hypertarget{ref-price_anticipating_2013}{}%
Price, D. T., Alfaro, R. I., Brown, K. J., Flannigan, M. D., Fleming, R.
A., Hogg, E. H., Girardin, M. P., Lakusta, T., Johnston, M., McKenney,
D. W., Pedlar, J. H., Stratton, T., Sturrock, R. N., Thompson, I. D.,
Trofymow, J. A., \& Venier, L. A. (2013). Anticipating the consequences
of climate change for Canada's boreal forest ecosystems.
\emph{Environmental Reviews}, \emph{21}(4), 322--365.
\url{https://doi.org/10.1139/er-2013-0042}

\leavevmode\hypertarget{ref-pulliam_relationship_2000}{}%
Pulliam, H. R. (2000). On the relationship between niche and
distribution. \emph{Ecology Letters}, \emph{3}(4), 349--361.
\url{https://doi.org/10.1046/j.1461-0248.2000.00143.x}

\leavevmode\hypertarget{ref-reich_geographic_2015}{}%
Reich, P. B., Sendall, K. M., Rice, K., Rich, R. L., Stefanski, A.,
Hobbie, S. E., \& Montgomery, R. A. (2015). Geographic range predicts
photosynthetic and growth response to warming in co-occurring tree
species. \emph{Nature Climate Change}, \emph{5}(2), 148--152.
\url{https://doi.org/10.1038/nclimate2497}

\leavevmode\hypertarget{ref-renwick_temporal_2015}{}%
Renwick, K. M., \& Rocca, M. E. (2015). Temporal context affects the
observed rate of climate-driven range shifts in tree species: Importance
of temporal context in tree range shifts. \emph{Global Ecology and
Biogeography}, \emph{24}(1), 44--51.
\url{https://doi.org/10.1111/geb.12240}

\leavevmode\hypertarget{ref-savage_elevational_2015}{}%
Savage, J., \& Vellend, M. (2015). Elevational shifts, biotic
homogenization and time lags in vegetation change during 40 years of
climate warming. \emph{Ecography}, \emph{38}(6), 546--555.
\url{https://doi.org/10.1111/ecog.01131}

\leavevmode\hypertarget{ref-scheller_spatially_2005}{}%
Scheller, R. M., \& Mladenoff, D. J. (2005). A spatially interactive
simulation of climate change, harvesting, wind, and tree species
migration and projected changes to forest composition and biomass in
northern Wisconsin, USA. \emph{Global Change Biology}, \emph{11}(2),
307--321. \url{https://doi.org/10.1111/j.1365-2486.2005.00906.x}

\leavevmode\hypertarget{ref-schurr_how_2012}{}%
Schurr, F. M., Pagel, J., Cabral, J. S., Groeneveld, J., Bykova, O.,
O'Hara, R. B., Hartig, F., Kissling, W. D., Linder, H. P., Midgley, G.
F., Schröder, B., Singer, A., \& Zimmermann, N. E. (2012). How to
understand species' niches and range dynamics: A demographic research
agenda for biogeography: A demographic research agenda for biogeography.
\emph{Journal of Biogeography}, \emph{39}(12), 2146--2162.
\url{https://doi.org/10.1111/j.1365-2699.2012.02737.x}

\leavevmode\hypertarget{ref-sexton_evolution_2009}{}%
Sexton, J. P., McIntyre, P. J., Angert, A. L., \& Rice, K. J. (2009).
Evolution and Ecology of Species Range Limits. \emph{Annual Review of
Ecology, Evolution, and Systematics}, \emph{40}(1), 415--436.
\url{https://doi.org/10.1146/annurev.ecolsys.110308.120317}

\leavevmode\hypertarget{ref-sittaro_tree_2017}{}%
Sittaro, F., Paquette, A., Messier, C., \& Nock, C. A. (2017). Tree
range expansion in eastern North America fails to keep pace with climate
warming at northern range limits. \emph{Global Change Biology},
\emph{23}(8), 3292--3301. \url{https://doi.org/10.1111/gcb.13622}

\leavevmode\hypertarget{ref-snell_using_2014}{}%
Snell, R. S., Huth, A., Nabel, J. E. M. S., Bocedi, G., Travis, J. M.
J., Gravel, D., Bugmann, H., Gutiérrez, A. G., Hickler, T., Higgins, S.
I., Reineking, B., Scherstjanoi, M., Zurbriggen, N., \& Lischke, H.
(2014). Using dynamic vegetation models to simulate plant range shifts.
\emph{Ecography}, \emph{37}(12), 1184--1197.
\url{https://doi.org/10.1111/ecog.00580}

\leavevmode\hypertarget{ref-soberon_grinnellian_2007}{}%
Soberón, J. (2007). Grinnellian and Eltonian niches and geographic
distributions of species. \emph{Ecology Letters}, \emph{10}(12),
1115--1123. \url{https://doi.org/10.1111/j.1461-0248.2007.01107.x}

\leavevmode\hypertarget{ref-stephenson_causes_2011}{}%
Stephenson, N. L., van Mantgem, P. J., Bunn, A. G., Bruner, H., Harmon,
M. E., O'Connell, K. B., Urban, D. L., \& Franklin, J. F. (2011). Causes
and implications of the correlation between forest productivity and tree
mortality rates. \emph{Ecological Monographs}, \emph{81}(4), 527--555.
\url{https://doi.org/10.1890/10-1077.1}

\leavevmode\hypertarget{ref-svenning_influence_2014}{}%
Svenning, J.-C., Gravel, D., Holt, R. D., Schurr, F. M., Thuiller, W.,
Münkemüller, T., Schiffers, K. H., Dullinger, S., Edwards, T. C.,
Hickler, T., Higgins, S. I., Nabel, J. E. M. S., Pagel, J., \& Normand,
S. (2014). The influence of interspecific interactions on species range
expansion rates. \emph{Ecography}, \emph{37}(12), 1198--1209.
\url{https://doi.org/10.1111/j.1600-0587.2013.00574.x}

\leavevmode\hypertarget{ref-talluto_extinction_2017}{}%
Talluto, M. V., Boulangeat, I., Vissault, S., Thuiller, W., \& Gravel,
D. (2017). Extinction debt and colonization credit delay range shifts of
eastern North American trees. \emph{Nature Ecology \& Evolution},
\emph{1}, 0182. \url{https://doi.org/10.1038/s41559-017-0182}

\leavevmode\hypertarget{ref-thuiller_road_2013}{}%
Thuiller, W., Münkemüller, T., Lavergne, S., Mouillot, D., Mouquet, N.,
Schiffers, K., \& Gravel, D. (2013). A road map for integrating
eco-evolutionary processes into biodiversity models. \emph{Ecology
Letters}, \emph{16}, 94--105. \url{https://doi.org/10.1111/ele.12104}

\leavevmode\hypertarget{ref-vanderwel_climate-related_2013}{}%
Vanderwel, M. C., Lyutsarev, V. S., \& Purves, D. W. (2013).
Climate-related variation in mortality and recruitment determine
regional forest-type distributions: Forest distributions from
demography. \emph{Global Ecology and Biogeography}, \emph{22}(11),
1192--1203. \url{https://doi.org/10.1111/geb.12081}

\leavevmode\hypertarget{ref-vanderwel_how_2014}{}%
Vanderwel, M. C., \& Purves, D. W. (2014). How do disturbances and
environmental heterogeneity affect the pace of forest distribution
shifts under climate change? \emph{Ecography}, \emph{37}(1), 10--20.
\url{https://doi.org/10.1111/j.1600-0587.2013.00345.x}

\leavevmode\hypertarget{ref-van_mantgem_apparent_2007}{}%
van Mantgem, P. J., \& Stephenson, N. L. (2007). Apparent climatically
induced increase of tree mortality rates in a temperate forest.
\emph{Ecology Letters}, \emph{10}(10), 909--916.
\url{https://doi.org/10.1111/j.1461-0248.2007.01080.x}

\leavevmode\hypertarget{ref-van_mantgem_widespread_2009}{}%
van Mantgem, P. J., Stephenson, N. L., Byrne, J. C., Daniels, L. D.,
Franklin, J. F., Fule, P. Z., Harmon, M. E., Larson, A. J., Smith, J.
M., Taylor, A. H., \& Veblen, T. T. (2009). Widespread Increase of Tree
Mortality Rates in the Western United States. \emph{Science},
\emph{323}(5913), 521--524.
\url{https://doi.org/10.1126/science.1165000}

\leavevmode\hypertarget{ref-vellend_global_2013}{}%
Vellend, M., Baeten, L., Myers-Smith, I. H., Elmendorf, S. C.,
Beausejour, R., Brown, C. D., De Frenne, P., Verheyen, K., \& Wipf, S.
(2013). Global meta-analysis reveals no net change in local-scale plant
biodiversity over time. \emph{Proceedings of the National Academy of
Sciences}, \emph{110}(48), 19456--19459.
\url{https://doi.org/10.1073/pnas.1312779110}

\leavevmode\hypertarget{ref-vissault_biogeographie_2016}{}%
Vissault, S. (2016). \emph{Biogéographie et dynamique de la forêt
tempérée nordique dans un contexte de changement climatiques.} {[}Master
thesis{]}. Université du Québec à Rimouski.

\leavevmode\hypertarget{ref-wisz_role_2013}{}%
Wisz, M. S., Pottier, J., Kissling, W. D., Pellissier, L., Lenoir, J.,
Damgaard, C. F., Dormann, C. F., Forchhammer, M. C., Grytnes, J.-A.,
Guisan, A., Heikkinen, R. K., Høye, T. T., Kühn, I., Luoto, M.,
Maiorano, L., Nilsson, M.-C., Normand, S., Öckinger, E., Schmidt, N. M.,
\ldots{} Svenning, J.-C. (2013). The role of biotic interactions in
shaping distributions and realised assemblages of species: Implications
for species distribution modelling. \emph{Biological Reviews},
\emph{88}(1), 15--30.
\url{https://doi.org/10.1111/j.1469-185X.2012.00235.x}

\leavevmode\hypertarget{ref-woodall_assessing_2013}{}%
Woodall, C. W., Zhu, K., Westfall, J. A., Oswalt, C. M., D'Amato, A. W.,
Walters, B. F., \& Lintz, H. E. (2013). Assessing the stability of tree
ranges and influence of disturbance in eastern US forests. \emph{Forest
Ecology and Management}, \emph{291}, 172--180.
\url{https://doi.org/10.1016/j.foreco.2012.11.047}

\leavevmode\hypertarget{ref-xu_importance_2012}{}%
Xu, C., Gertner, G. Z., \& Scheller, R. M. (2012). Importance of
colonization and competition in forest landscape response to global
climatic change. \emph{Climatic Change}, \emph{110}(1-2), 53--83.
\url{https://doi.org/10.1007/s10584-011-0098-5}

\leavevmode\hypertarget{ref-zhang_half-century_2015}{}%
Zhang, J., Huang, S., \& He, F. (2015). Half-century evidence from
western Canada shows forest dynamics are primarily driven by competition
followed by climate. \emph{Proceedings of the National Academy of
Sciences}, \emph{112}(13), 4009--4014.
\url{https://doi.org/10.1073/pnas.1420844112}

\leavevmode\hypertarget{ref-zhu_failure_2012}{}%
Zhu, K., Woodall, C. W., \& Clark, J. S. (2012). Failure to migrate:
Lack of tree range expansion in response to climate change. \emph{Global
Change Biology}, \emph{18}(3), 1042--1052.
\url{https://doi.org/10.1111/j.1365-2486.2011.02571.x}

\end{document}


\cleardoublepage


%%%%%%%%%%%%%%%%%%%%%%%%%%%%%%%%%%%%%%%%%%%%%%%%%%%%%%%%%%%%
%%%%%%%%%%%%%%%%%%%%                   %%%%%%%%%%%%%%%%%%%%%
%%%%%%%%%%%%%%%%%%%%  A R T I C L E 1  %%%%%%%%%%%%%%%%%%%%%
%%%%%%%%%%%%%%%%%%%%                   %%%%%%%%%%%%%%%%%%%%%
%%%%%%%%%%%%%%%%%%%%%%%%%%%%%%%%%%%%%%%%%%%%%%%%%%%%%%%%%%%%

\renewcommand\thefigure{1.\arabic{figure}}
\renewcommand\thetable{1.\arabic{table}}

\include{article1/article1}

\cleardoublepage


%%%%%%%%%%%%%%%%%%%%%%%%%%%%%%%%%%%%%%%%%%%%%%%%%%%%%%%%%%%%
%%%%%%%%%%%%%%%%%%%%                   %%%%%%%%%%%%%%%%%%%%%
%%%%%%%%%%%%%%%%%%%%  A R T I C L E 2  %%%%%%%%%%%%%%%%%%%%%
%%%%%%%%%%%%%%%%%%%%                   %%%%%%%%%%%%%%%%%%%%%
%%%%%%%%%%%%%%%%%%%%%%%%%%%%%%%%%%%%%%%%%%%%%%%%%%%%%%%%%%%%

\renewcommand\thefigure{2.\arabic{figure}}
\renewcommand\thetable{2.\arabic{table}}

\include{article2/article2}
%%
%% etc.

 % S'il y a une bibliographie pour tout le document, on peut
 % utiliser les commandes suivantes. À noter que le style est
 % laisser au choix de l'auteur·e. (Il est même possible
 % d'utiliser <natbib>).
 % Il est possible d'avoir une bibliographie pour chaque
 % chapitre. Consulter l'article en exemple pour voir
 % comment faire.


 %%%%%%%%%%%%%%%%%%%%%%%%%%%%%%%%%%%%%%%%%%%%%%%%%%%%%%%%%%%%
 %%%%%%%%%%%%%%%%%%%%                   %%%%%%%%%%%%%%%%%%%%%
 %%%%%%%%%%%%%%%%%%%%  A R T I C L E 3  %%%%%%%%%%%%%%%%%%%%%
 %%%%%%%%%%%%%%%%%%%%                   %%%%%%%%%%%%%%%%%%%%%
 %%%%%%%%%%%%%%%%%%%%%%%%%%%%%%%%%%%%%%%%%%%%%%%%%%%%%%%%%%%%

 \renewcommand\thefigure{3.\arabic{figure}}
 \renewcommand\thetable{3.\arabic{table}}

 \include{article3/article3}


 %%%%%%%%%%%%%%%%%%%%%%%%%%%%%%%%%%%%%%%%%%%%%%%%%%%%%%%%%%%%
 %%%%%%%%%%%%%%%%%%%%                   %%%%%%%%%%%%%%%%%%%%%
 %%%%%%%%%%%%%%%%%%%%  CONCLUSION       %%%%%%%%%%%%%%%%%%%%%
 %%%%%%%%%%%%%%%%%%%%                   %%%%%%%%%%%%%%%%%%%%%
 %%%%%%%%%%%%%%%%%%%%%%%%%%%%%%%%%%%%%%%%%%%%%%%%%%%%%%%%%%%%

 \renewcommand\thefigure{4.\arabic{figure}}
 \renewcommand\thetable{4.\arabic{table}}

 \francais

\chapter*{Conclusion}

\hypertarget{sommaire-des-ruxe9sultats-comprendre-et-observer-les-changements-ruxe9cents-des-foruxeats}{%
\section{Sommaire des résultats: Comprendre et observer les changements
récents des
forêts}\label{sommaire-des-ruxe9sultats-comprendre-et-observer-les-changements-ruxe9cents-des-foruxeats}}

De nombreuses études utilisent les données d'inventaire forestier pour
tenter de prédire l'avenir sous les changements climatiques
\citep{boulanger_climate_2017, perie_dominant_2016, vissault_biogeographie_2016, meier_climate_2012, iverson_estimating_2008, chen_modeling_2002}.
Mais, s'il est possible de faire des prédictions de grands changements
pour 2050, soit dans 30 ans, ne devrait-on pas déjà commencer à
percevoir les premiers signes de ces changements dans les données
cumulées depuis 1970, il y a 50 ans? Quelles informations pouvons-nous
tirer de ces changements récents? De plus, les projections des effets du
changement climatique sur les forêts ont généralement mis l'accent sur
la capacité des espèces à tolérer les augmentations de température et
les sécheresses et à se disperser, mais ils n'ont pas nécessairement
intégré les effets des perturbations. Étant donné l'importance des
perturbations naturelles et de l'exploitation forestière dans la
dynamique des forêts \citep{turner_disturbance_2010}, ces projections du
futur basées sur le changement climatique en isolation sont-elles
réalistes, voire même trompeuses? Si les perturbations interagissent
avec le changement climatique et exacerbent la mortalité des arbres, les
nouvelles politiques d'aménagement qui reposent sur des modèles
incomplets pourraient bien être mal adaptées.

À travers ma thèse de recherche, j'ai donc tenté de répondre à ces
questionnements. Les trois chapitres ont permis de décrire de multiples
aspects de la dynamique des forêts au cours des dernières décennies en
réponse aux effets combinés du changement climatique et des
perturbations.

\hypertarget{ruxe9organisation-des-communautuxe9s}{%
\subsection{Réorganisation des
communautés}\label{ruxe9organisation-des-communautuxe9s}}

Dans le chapitre 1, je me suis intéressée aux moteurs des changements de
composition dans les communautés forestières. Les perturbations (par
exemple, les coupes à blanc, les épidémies d'insectes, les incendies)
sont les principaux facteurs de changement de composition des
communautés forestières, i.e.~la diversité \(\beta\) temporelle, dans
l'écotone tempéré-boréal. Leurs effets laissent une empreinte à long
terme puisque les perturbations historiques sont les plus importantes.
En revanche, les effets les changements de température et de
précipitation sur les changements des communautés forestières sont très
faibles. Sans approfondir, on pourrait en conclure prématurément que le
changement climatique n'a pas influencé la composition des forêts au
cours des dernières décennies.

Malgré la prévalence des perturbations, l'analyse des changements des
traits écologiques de la communauté m'a permis de montrer une
thermophilisation généralisée des communautés à travers le Québec. Cette
thermophilisation correspond à une augmentation des espèces de climat
chaud au détriment des espèces de climat froid dans les communautés.
Dans la région d'étude, ce processus résulte principalement du gain
d'une espèce tempérée, \emph{Acer rubrum}, et de la perte de deux espèce
boréales, \emph{Abies balsamea} et \emph{Picea mariana}. En outre, la
thermophilisation a été plus grande et s'est étendue plus au nord dans
les communautés modérément perturbées que celles qui n'ont pas été
perturbées ou qui ont subi des perturbations majeures. Ainsi, ces
résultats suggèrent que des perturbations modérées, mais non majeures,
pourraient faciliter les gains d'espèces adaptées aux conditions chaudes
sous l'effet du changement climatique.

En terminant ce chapitre, je suis restée avec une question importante en
tête : la thermophilisation actuelle des forêts indique-t-elle un
changement permanent des écosystèmes forestiers ou bien seulement une
dynamique transitoire? Est-ce que les perturbations pourraient favoriser
des changements d'états rapides et permanents? C'est la question sur
laquelle je me suis penchée dans le second chapitre de ma thèse.

\hypertarget{la-dynamique-des-foruxeats}{%
\subsection{La dynamique des forêts}\label{la-dynamique-des-foruxeats}}

Dans le chapitre 2, j'ai analysé la dynamique de transition des forêts
du Québec en utilisant un modèle à quatre états, soit boréal, mixte,
tempéré et pionnier. Cette analyse a d'abord révélé une forte proportion
de transition de l'état pionnier vers boréal, qui signale une
régénération de la forêt boréale anciennement perturbée, et de
transition de l'état mixte vers tempéré, qui suggère une augmentation de
la proportion d'espèces tempérées. Encore une fois, les perturbations
naturelles et anthropiques ressortent comme les moteurs principaux de la
dynamique de transition des forêts au cours des dernières décennies.
Alors que les perturbations majeures déclenchent surtout des transitions
vers l'état pionnier, les perturbations modérées favorisent les
transitions de l'état mixte vers l'état tempéré.

De plus, ces changements dans les probabilités de transition que nous
percevons dans 48 ans de données risquent de se répercuter sur les
futurs patrons forestiers à l'échelle régionale et sur la vitesse de ces
changements. En effet, les perturbations modérées accélèrent le taux de
renouvellement des forêts et, à long terme, favorisent une augmentation
de la proportion de forêts tempérées dans le paysage. Par conséquent,
les perturbations modérées ont le potentiel de catalyser un déplacement
plus rapide de l'écotone boréal-tempéré vers le nord sous le changement
climatique.

Les transitions des forêts mixtes à tempérées résultent surtout de la
mortalité d'une espèce boréale dominante, \emph{Abies balsamea}, et les
trouées ainsi créées permettent la croissance accrue des espèces
tempérées compagnes. Contrairement aux hypothèses avancées dans mon
premier chapitre, le recrutement des espèces tempérées semble jouer un
rôle négligeable dans cette dynamique, mais il se peut que la méthode
d'analyse ne permette pas de détecter la contribution du recrutement.

Que les perturbations entraînent la mortalité de \emph{Abies balsamea}
étaient prévisibles. Et la croissance subséquente des espèces tempérées
compagnes montrent qu'il y a sans doute un changement de rapport des
forces compétitives entre les espèces en lien avec le changement
climatique. Si les transitions reposent davantage sur la mortalité et
que celle-ci n'est pas compensée par le recrutement de nouveaux arbres,
il s'en suit que les forêts mixtes pourraient en fin de compte dépérir.
Le recrutement est une étape essentielle pour assurer la régénération
des forêts et la migration des arbres. Alors que les perturbations
favorisent la thermophilisation des forêts et la transition de
peuplements mixtes à tempérés, quel est leur impact sur le recrutement
des espèces tempérées?

\hypertarget{la-ruxe9guxe9nuxe9ration-des-foruxeats-les-premiers-pas-de-la-migration}{%
\subsection{La régénération des forêts: les premiers pas de la
migration}\label{la-ruxe9guxe9nuxe9ration-des-foruxeats-les-premiers-pas-de-la-migration}}

C'est la question qui a motivée mon troisième et dernier chapitre. Ce
chapitre a mis en lumière de grands déplacements de distribution vers le
nord pour les gaulis (i.e., jeunes arbres entre 1 et 9 cm de diamètre)
de \emph{Acer rubrum} et \emph{Acer saccharum}, deux espèces reconnues
pour leur grande tolérance aux conditions environnementales et aux
perturbations. La distribution de ces deux espèces s'est déplacée vers
le nord peu importe le niveau de perturbations (mineur, modéré ou
majeur). En revanche, la distribution des gaulis de \emph{Betula
alleghaniensis} s'est déplacée vers le nord seulement dans les parcelles
peu ou pas perturbées, alors que dans les parcelles modérément ou
sévèrement perturbées, elle montrait plutôt des signes de contraction
d'aire. Ces résultats soulignent que les futures modifications d'aire de
répartition des arbres peuvent dépendre de la réponse des espèces aux
perturbations.

La distribution des espèces tempérées dans les populations marginales à
leur limite nord est contrainte aux sommets de pente
\citep{goldblum_age_2002, tremblay_potential_2002, barras_supply_1998},
là où le microclimat est plus chaud. Mes analyses révèlent que la
distribution altitudinale des gaulis tend à descendre vers le bas des
pentes, surtout dans les régions à leur limite nord. Ces tendances, bien
que non significatives, peuvent signaler le début d'une migration des
populations marginales qui sont isolées aux sommets des collines. Tel
que suggéré pour la migration postglaciaire des arbres
\citep{mclachlan_molecular_2005}, ces populations marginales pourraient
jouer un rôle très important dans la migration future des arbres en
réponse au changement climatique.

Mes résultats soulignent que le recrutement des gaulis d'espèces
tempérées à leur limite nord est largement favorisé par l'abondance
d'arbres conspécifiques pour la dispersion des graines, mais freiné par
la compétition des espèces boréales. De plus, les conditions de mauvais
drainage (sites hydriques) dans les bas de pente ne sont pas propice à
l'établissement des gaules, tandis que les coupes forestières modérées,
en diminuant la compétition et en libérant les ressources, ont une
influence bénéfique. L'ensemble des résultats de ce chapitre suggère
que, malgré l'effet positif d'une exploitation forestière modérée sur le
recrutement, la prévalence des contraintes locales associées à la
composition des peuplements et à la position topographique risque de
freiner la migration des espèces tempérées vers le nord.

\hypertarget{les-perturbations-catalyseurs-de-changements-dans-les-foruxeats}{%
\section{Les perturbations --- catalyseurs de changements dans les
forêts}\label{les-perturbations-catalyseurs-de-changements-dans-les-foruxeats}}

Dans l'ensemble, mes résultats soulignent le rôle central des
perturbations dans la réponse des forêts face au changement climatique.
En effet, la composition et la structure (Chapitre 1), la dynamique de
transition (Chapitre 2), ainsi que la dynamique de régénération
(Chapitre 3) des forêts de l'écotone boréal-tempéré au Québec sont
principalement contrôlées par les perturbations et leurs effets semblent
interagir avec ceux des changements climatiques. J'ai montré comment les
perturbations accélèrent la réponse des communautés forestières aux
changements climatiques, révélant des synergies qui ont le potentiel de
modifier l'avenir de nos forêts.

\hypertarget{les-perturbations-et-la-dynamique-forestiuxe8re}{%
\subsection{Les perturbations et la dynamique
forestière}\label{les-perturbations-et-la-dynamique-forestiuxe8re}}

Les perturbations font partie intégrante de la dynamique des forêts.
Feux, épidémies d'insectes, chablis, inondations sont des événements de
mortalités aigues et à court terme \citep[\emph{pulse
disturbance};][]{bender_perturbation_1984} qui s'inscrivent dans la
dynamique naturelle des forêts et permettent leur renouvellement
\citep{attiwill_disturbance_1994}. Les espèces forestières sont
généralement bien adaptés à un régime de perturbation naturelle, défini
par la superficie, la sévérité et la fréquence
\citep{turner_disturbance_2010}. Par exemple, dans la forêt boréale, les
cônes sérotineux de \emph{Pinus banksiana} nécessitent des températures
élevées pour s'ouvrir et lui permettent de coloniser rapidement un site
après un feu \citep{burns_silvics_1990}. Dans la forêt tempérée, les
semis et les gaules de \emph{Acer saccharum} peuvent persister longtemps
dans un sous-bois ombragé jusqu'à ce qu'une trouée créée par la chute
des vieux arbres lui permette de rejoindre la canopée rapidement
\citep{burns_silvics_1990}. Ainsi, les communautés forestières ont une
bonne capacité à se rétablir et à revenir à leur état initial suite à
une perturbation importante, i.e.~elles sont résilientes
\citep{gunderson_ecological_2000}.

Les perturbations anthropiques viennent s'ajouter ou se substituer aux
perturbations naturelles pour créer un tout nouveau régime de
perturbations \citep{boucher_land_2014}. Les coupes forestières ont
tendance à homogénéiser et rajeunir le paysage car leur période de
rotation est plus courte, tandis que leur superficie et leur intensité
sont plus homogènes
\citep{mcrae_comparisons_2001, boucher_logging-induced_2006}. Les forêts
présentent une certaine résilience face aux coupes forestières; tant que
le régime de coupes forestières restent dans la plage de variabilité
naturelle des forêts \citep{grondin_have_2018}, la plupart des espèces
peuvent se régénérer après coupe. Toutefois, à long terme, des coupes
répétées peuvent entraîner d'importants changements de composition, par
exemple en favorisant l'augmentation des espèces pionnières intolérantes
à l'ombre, comme \emph{Populus tremuloides}, et des espèces tolérantes
aux perturbations, comme \emph{Acer rubrum}
\citep{danneyrolles_stronger_2019, boucher_logging-induced_2006}.

\hypertarget{le-changement-climatique-uxe9rode-la-ruxe9silience}{%
\subsection{Le changement climatique érode la
résilience}\label{le-changement-climatique-uxe9rode-la-ruxe9silience}}

En plus de ces perturbations de type \emph{pulse}, le climat global
change graduellement et de façon persistante \citep[\emph{press
disturbance};][]{bender_perturbation_1984}, ce qui peut altérer la forme
du bassin d'attraction (Fig. \ref{fig0.4}), rendant les forêts plus
fragiles face aux autres perturbations
\citep{scheffer_catastrophic_2001}. En effet, au fur et à mesure que le
climat se réchauffe, le déséquilibre entre la répartition de certaines
espèces et les conditions climatiques s'agrandit
\citep{talluto_extinction_2017}. À la marge sud de leur répartition,
certaines populations peuvent persister mais sont vouées à l'extinction
puisque les conditions environnementales ne sont plus propices à leur
régénération ou leur survie; on observe une dette d'extinction. À leur
marge nord, de nouveaux habitats sont devenus suffisamment chauds mais
n'ont toujours pas été colonisés en raison de différentes contraintes au
recrutement (e.g., distance, barrière, prédation des graines); on
observe un crédit de colonisation
\citep{jackson_balancing_2010, tilman_habitat_1994}. Les forêts peuvent
ainsi perdre leur capacité à se rétablir, puisque les espèces en place
ne sont plus aussi bien adaptées au climat et pourront être remplacées
advenant une perturbation \citep{johnstone_changing_2016}. Cette perte
de résilience peut entraîner un basculement vers un nouvel état de
l'écosystème \citep{scheffer_catastrophic_2001}.

\hypertarget{interaction-changement-climatique-et-perturbations}{%
\subsection{Interaction changement climatique et
perturbations}\label{interaction-changement-climatique-et-perturbations}}

En théorie, des changements d'états stables peuvent subvenir suivant
deux mécanismes: (1) des changements graduels dans les conditions
environnementales jusqu'à un niveau critique auquel le système
s'effondre soudainement, et (2) des perturbations trop sévères ou en
rafale qui poussent le système hors de son bassin d'attraction
\citep{scheffer_catastrophic_2001}. Par exemple, la grande augmentation
des taux de mortalité des arbres dans l'ouest des États-Unis en réponse
à un stress hydrique grandissant \citep{van_mantgem_apparent_2007}
pourrait être le signe avant-coureur d'un dépérissement massif des
forêts. Dans un autre cas, \citet{payette_shift_2003} a montré que les
impacts cumulés de la coupe forestière, suivie d'une épidémie d'insectes
puis d'un incendie pourraient avoir des effets catastrophiques sur la
régénération des arbres, entraînant la transition d'une pessière dense
en un milieu ouvert dominé par le lichen. Bien que les deux mécanismes
puissent indépendamment mener à une transition rapide d'état, leurs
effets cumulés augmentent le risque de changements rapides de régime
écologique \citep[\emph{regime
shift};][]{scheffer_catastrophic_2001, harris_biological_2018}. Mes
résultats supportent cette hypothèse et montrent que le réchauffement
climatique érode la résilience des forêts mixtes tandis que les
perturbations éliminent les espèces boréales en place et accélèrent le
processus de succession vers davantage d'espèces tempérées adaptées aux
températures plus chaudes (Chapitres 1, 2). Sans perturbation, la grande
inertie des forêts cache la perte de résilience
\citep{johnstone_changing_2016}. Les arbres, ayant une longue durée de
vie, peuvent faire paraître les forêts inébranlables face aux
changements environnementaux même si la niche de régénération est en
train de se déplacer \citep[Chapitre
3;][]{sittaro_tree_2017, boisvertmarsh_divergent_2019}.

\hypertarget{uxe9tats-alternatifs-stables-et-changement-de-ruxe9gime}{%
\subsection{États alternatifs stables et changement de
régime}\label{uxe9tats-alternatifs-stables-et-changement-de-ruxe9gime}}

Suite à une perturbation modérée, la composition des forêts mixtes peut
donc se déplacer rapidement vers une dominance en espèces tempérées qui
sont mieux adaptées aux températures plus chaudes. Le long du gradient
latitudinal, la forêt boréale au nord et la forêt tempérée au sud sont
les seuls états stables possibles (Fig. \ref{fig4.1}a). Toutefois, dans
la zone de transition entre ces deux grands biomes, on trouve la forêt
mixte, un état relativement rare qui change facilement d'un état de
dominance à un autre. Ceci suggère que la forêt mixte représente un état
instable où le système n'est que transitoire. La coexistence des espèces
boréales, tempérées et pionnières dans la forêt mixte s'est maintenue
grâce à l'hétérogénéité des perturbations naturelles combinée à des
différences dans les stratégies de cycle de vie des espèces
\citep{kneeshaw_natural_2007, bouchard_tree_2006} sous un climat donné.
Mon interprétation est que le réchauffement modifie le bassin
d'attraction de l'état mixte et abaisse le seuil critique à franchir
pour se rendre à l'état tempéré(Fig. \ref{fig4.1}b). De plus, le
réchauffement peut aussi rendre le bassin de l'état tempéré plus profond
et celui de l'état boréal moins profond. Ces modifications de la courbe
des états alternatifs stables fragilisent la forêt mixte et la rendent
encore plus vulnérable aux transitions vers l'état tempéré suite à une
perturbation modérée (Fig. \ref{fig4.1}b). Une fois que les espèces
tempérées feuillues deviennent dominantes elles peuvent alors se
maintenir par une boucle de rétroaction positive, par exemple en
modifiant la fertilité des sols, la luminosité du sous-bois, et le
régime de perturbation.

\begin{figure}
\centering
\includegraphics[width=.65\textwidth]{conclusion/figures/etat_alternatif2.png}
\caption[Représentation conceptuelle des états alternatifs stables le long du gradient latitudinal]{Mise à jour de ma représentation conceptuelle des états alternatifs stables le long du gradient latitudinal sans changement climatique (a) et avec changement climatique (b). Dans ce schéma avec la courbe, la boule caractérise l'état de l'écosystème à un instant donné, le paysage correspond à l'ensemble des états dans lesquels l’écosystème peut se retrouver, les vallées sont les bassins d'attraction des équilibres stables, et les sommets des collines sont les équilibres instables. Le schéma rectangulaire représente le paysage correspondant. Au sud du gradient latitudinal au de température, un seul état stable existe, la forêt tempérée, boule rouge. Au nord du gradient, l'état stable dominant est la forêt boréale, boule bleue. Ce sont des états stables dynamiques; les perturbations peuvent faire déplacer la boule dans son bassin d'attraction et elle peut même passer par l'état pionnier (un état transitoire non représenté sur la courbe, mais représenté par des carrés jaunes dans le paysage). La forêt étant habituellement résiliente aux perturbations, elle retourne ensuite vers son état initial. Au centre, dans la zone d'écotone, la forêt mixte, boule verte, serait un état transitoire entre les deux états stables dominants qui est maintenue grâce à la dynamique naturelle de perturbations. Le changement du climat (b) peut provoquer un changement de la forme du paysage de différentes façons: (1) en abaissant le seuil pour passer de l'état mixte à tempéré, (2) en creusant et en élargissant le bassin d'attraction de l'état tempéré et (3) en rendant moins profond le bassin d'attraction de l'état boréal. Ces modifications font que la boule dégringole plus souvent de la vallée mixte à la vallée sous l'effet d'une perturbation.}
\label{fig4.1}
\end{figure}

Contrairement aux perturbations modérées, les perturbations majeures
détruisent toutes la communauté en place et poussent le système vers
l'état pionnier, i.e. des peuplements dominés par des espèces
intolérantes à l'ombre, comme \emph{Betula papyrifera} et \emph{Populus
tremuloides}, ou bien avec pas ou très peu d'arbres. Leur effet à long
terme est difficile à prévoir à partir des données d'inventaire puisque
les systèmes n'ont sans doute pas eu le temps de revenir à un état
stable. En effet, l'état pionnier est en général un état transitoire. Il
est donc fort probable que la majorité des forêts soient encore en train
de se déplacer vers leur état d'équilibre. Toutefois, il se pourrait que
certaines forêts se soient effondrées définitivement. Par exemple, les
peuplements de \emph{Populus tremuloides} dans les pessières noires
représentent normalement un état de transition, mais, sous l'effet des
coupes forestières, ces peuplements sont en expansion et semblent se
maintenir \citep{grondin_les_2003}.

\hypertarget{ruxe9silience-et-capacituxe9-adaptative}{%
\subsection{Résilience et capacité
adaptative}\label{ruxe9silience-et-capacituxe9-adaptative}}

La biodiversité est un élément clé de la résilience et la capacité
adaptative des forêts
\citep{messier_functional_2019, filotas_viewing_2014}. Alors que la
résilience permet à une forêt de retrouver sa structure et ses fonctions
d'origine, la capacité adaptative lui permet de diverger d'un état
antérieur qui était mal adapté aux conditions environnementales
\citep{messier_managing_2013, filotas_viewing_2014}. Bien que moins
résilientes, les forêts mixtes montrent une bonne capacité adaptative
face au changement climatique puisqu'elles arrivent à se réorganiser de
manière à ajuster leur composition aux nouvelles conditions
environnementales \citep{messier_functional_2019, filotas_viewing_2014}.
Mais qu'en est-il des forêts boréales pures? Contrairement aux forêts
mixtes, celles-ci montrent peu de changements de composition en réponse
au réchauffement climatique. En effet, alors qu'il y a eu très peu de
transitions vers l'état mixte et pas de thermophilisation des
communautés, on a plutôt observé une dynamique de remplacement entre les
états pionnier et boréal (Chapitres 1 et 2). Comme les forêts du nord du
Québec sont très pauvres en espèces, étant largement dominées par
\emph{Picea mariana} et \emph{Abies balsamea}, elles ont moins de
ressources pour faire face aux changements récents et futurs et ce qui
pourrait limiter leur capacité à s'ajuster et s'éloigner d'un état
possiblement mal adapté. De plus, on prévoit que le climat de la forêt
boréale de l'est de l'Amérique du Nord devrait ressembler à celui de la
forêt tempérée d'ici la fin du siècle \citep{gauthier_boreal_2015}.
Toutefois, la migration des espèces tempérées dans ces régions semble
limitée par plusieurs facteurs non-climatiques, notamment leur capacité
de dispersion, la compétition par les espèces boréales, ainsi que les
conditions édaphiques \citep[Chapitre
3;][]{solarik_priority_2019, carteron_soil_2020}. Ainsi, si le
réchauffement continue de s'accentuer et que les espèces tempérées ne
parviennent pas à coloniser les régions boréales, on peut se demander
comment se transformeront les sapinières et les pessières du Québec.

La fréquence et la gravité des perturbations naturelles, telles que les
incendies, les épidémies d'insectes, les sécheresses et les vagues de
chaleur, devraient augmenter dans de nombreuses régions du monde
\citep{seidl_forest_2017, bergeron_past_2006}. À la lumière de mes
résultats, cela pourrait conduire à des changements majeurs dans la
composition des forêts au cours des prochaines décennies et
potentiellement à des modifications permanentes des états forestiers.
Cependant, si les perturbations deviennent trop fréquentes et trop
intenses, les forêts pourraient basculer vers une dominance en espèces
pionnières de début de succession. Des comportements non-linéaires dans
les réponses des écosystèmes forestiers impliquent que de nombreuses
projections sous-estiment probablement l'ampleur des changements futurs
de la biodiversité \citep{scheffer_catastrophic_2001}. Une telle
conclusion souligne l'importance de tenir compte de l'effet synergique
des perturbations et du changement climatique dans les stratégies de
gestion forestière ainsi que dans les modèles de prédiction.

\hypertarget{des-foruxeats-en-transformation-des-individus-aux-biomes}{%
\section{Des forêts en transformation: des individus aux
biomes}\label{des-foruxeats-en-transformation-des-individus-aux-biomes}}

Les effets des changements environnementaux se répercutent à chaque
niveau d'organisation de la biodiversité, se transmettant des individus
jusqu'au biome, en passant par les populations, les communautés et les
écosystèmes \citep{bellard_impacts_2012, parmesan_globally_2003}. En
effet, conformément avec les concepts de la science des systèmes
complexes, les changements démographiques au bas de la hiérarchie
peuvent faire émerger, par des processus d'organisation autonome, des
réorganisations massives à l'échelle régionale. De plus, les
interactions entre n'importe quel niveau de la hiérarchie peuvent donner
naissance à des dynamiques non-linéaires, comme les changements de
régime écologique \citep[Fig.
\ref{fig4.2};][]{filotas_viewing_2014, messier_managing_2013}. Les
résultats présentés dans mes trois chapitres de thèse permettent de bien
illustrer ces processus ascendants et en interaction par lequel les
forêts de l'écotone boréal-tempéré sont en train de se transformer.
L'interprétation de mes résultats sous la perspective des systèmes
complexes permet de mieux comprendre la réponse des forêts sous l'effet
combiné du changement climatique et des perturbations.

\begin{figure}
\centering
\includegraphics[width=.9\textwidth]{conclusion/figures/complex.png}
\caption[Représentation conceptuelle des effets des changements environnementaux sur les différents niveaux d'organisation biologique]{Représentation conceptuelle des effets des changements environnementaux sur les forêts sous l'angle d'un système complexe. Les changements dans la dynamique des forêts sont transférés de façon ascendante entre les différents niveaux d'organisation biologique. Des interactions et des rétroactions ont lieu entre les entités à l'intérieur et à travers les échelles spatiales, temporelles et hiérarchiques. Les entités qui interagissent à un niveau donnent naissance à des entités émergentes de niveau supérieur, dont l'existence, à son tour, affecte le comportement des entités de niveau inférieur. Schéma issu de Messier \emph{et al.} 2013.}
\label{fig4.2}
\end{figure}

\hypertarget{changements-duxe9mographiques}{%
\subsection{Changements
démographiques}\label{changements-duxe9mographiques}}

Dans un premier temps, le réchauffement climatique favorise le
recrutement, la survie et la croissance des espèces tempérées à leur
limite nord \citep[Chapitre
3;][]{fisichelli_temperate_2014, boisvertmarsh_divergent_2019, peng_drought-induced_2011, goldblum_tree_2005, grundmann_impact_2011, bolte_understory_2014},
ce qui leur confère un avantage compétitif par rapport aux espèces
boréales. Mais, en l'absence de perturbations, les arbres poussent
lentement, leurs taux de mortalité et de recrutement sont faibles et la
compétition pour l'espace et la lumière est intense. Ainsi, le
renouvellement de la communauté est très lent.

Des perturbations modérées peuvent cependant éliminer les individus des
espèces résidentes. Dans la zone d'étude, les perturbations naturelles
et anthropiques ont provoqué une mortalité disproportionnée d'une espèce
dominante dans les forêts mixtes (Chapitre 2). En effet, \emph{Abies
balsamea} a subi une mortalité massive suite aux grandes épidémies de
tordeuse des bourgeons de l'épinette dans les années 1967-1992
\citep{duchesne_population_2008}. De plus, cette espèce est aussi la
plus coupée au Québec. Les trouées dans la canopée résultant de la perte
de cette espèce boréale ubiquiste et abondante ont probablement permis
de réduire la compétition et libérer des ressources, favorisant ensuite
la croissance rapide des espèces tempérées compagnes, telles que
\emph{Acer saccharum}, \emph{A. rubrum} et \emph{Betula alleghaniensis}
(Chapitre 2). De plus, alors que les perturbations naturelles semblent
avoir un effet plutôt délétère, les coupes partielles favorisent une
hausse du recrutement de ces espèces tempérées à leur limite nord
(Chapitre 3).

\hypertarget{dynamique-de-population}{%
\subsection{Dynamique de population}\label{dynamique-de-population}}

Le réchauffement et les perturbations peuvent donc exercer leur
influence sur la dynamique de population des espèces par le biais de
plusieurs processus démographiques, notamment la reproduction, le
recrutement, la croissance et la mortalité. Ces changements à l'échelle
des individus et des sites s'accumulent dans le temps et dans l'espace
pour altérer l'abondance et l'aire de répartition des populations
\citep{holt_theoretical_2005}. Si les effets sur les individus d'une
espèce sont généralement négatifs, le taux de croissance de la
population diminuera; certaines populations pourraient être amenées à
disparaître localement et, dans le pire des cas, régionalement. À
l'inverse, lorsque les effets sur les taux démographiques d'une espèce
sont positifs, les populations grandissent, ce qui peut entraîner une
augmentation locale de l'abondance et une expansion régionale de l'aire
de répartition.

Ces effets sur le fitness des espèces ne sont pas aléatoires, mais sont
déterminés par leurs tolérances physiologiques, leurs stratégies
d'histoire de vie et leurs capacités de dispersion
\citep{aubin_traits_2016, estrada_species_2015}. Ces caractéristiques
écologiques spécifiques sont à l'origine de la grande variabilité dans
les réponses des arbres face aux changements environnementaux. Alors que
le réchauffement peut favoriser le taux de croissance des populations
d'espèces qui sont limitées par les températures très froides
\citep{de_frenne_microclimate_2013, devictor_birds_2008}, les
perturbations devraient promouvoir les espèces opportunistes,
intolérantes à l'ombre, avec une bonne capacité de dispersion et de
reproduction végétative \citep{danneyrolles_stronger_2019}. Par exemple,
\emph{Acer rubrum}, dont la distribution au nord est en partie limitée
par un faible niveau de reproduction sexuée
\citep{tremblay_potential_2002}, est reconnue comme une espèce
super-généraliste et opportuniste, capable de coloniser rapidement des
sites variés après une perturbation
\citep{abrams_red_1998, fei_rapid_2009}. Ces caractéristiques pourraient
donc expliquer le grand succès de \emph{Acer rubrum} sous l'effet
combiné du réchauffement et des perturbations. En revanche, certaines
espèces limitées par la dispersion, comme \emph{Tilia americana},
limitées à un habitat spécifique, comme \emph{Acer saccharinum}, ou
encore intolérante aux perturbations, comme \emph{Tsuga canadensis},
pourrait ne pas bénéficier des opportunités de recrutement et de
croissance suivant des perturbations, et ce, même si le climat devient
plus favorable.

De plus, la sensibilité aux variations climatiques peut varier entre les
différents processus démographiques \citep{niinemets_responses_2010}.
Par exemple, comme la régénération est souvent plus sensible que la
survie des adultes aux stress hydrique ou thermique
\citep{niinemets_responses_2010}, une population peut persister pendant
des décennies ou des siècles sur un site donné, même si les conditions
climatiques sont devenues défavorables \citep{talluto_extinction_2017}.
Pour cette raison, \citet{jump_altitude-for-latitude_2009} ont suggéré
que l'expansion à la limite nord de la distribution d'une espèce sera
plus rapide que les changements à la limite sud puisque la reproduction
et le recrutement sont plus sensibles aux changements environnementaux
que la mortalité des individus établis. Dans les forêts du Québec, les
données montrent effectivement une augmentation rapide du recrutement de
plusieurs espèces tempérées (Chapitre 3). Toutefois, les changements
découlant de la mortalité des espèces boréales étaient plus rapides
puisque ce processus n'était pas contrôlé par un stress climatique mais
surtout par des perturbations directes (e.g., coupe, feu, épidémie).
Ainsi, les perturbations risquent d'accélérer la contraction de l'aire
de répartition des espèces boréales, tandis que l'effet sur l'expansion
des espèces tempérées à leur limite nord pourrait être beaucoup plus
lent en raison des contraintes au recrutement (Chapitre 3).

\hypertarget{dynamique-des-communautuxe9s}{%
\subsection{Dynamique des
communautés}\label{dynamique-des-communautuxe9s}}

Ces changements démographiques se traduisent donc en pertes et en gains
d'espèces à l'échelle locale et influencent la composition et la
structure des communautés forestières. Les gains en espèces tempérées
combinés aux pertes en espèces boréales ont entraîné une
thermophilisation de nombreuses communautés au Québec, particulièrement
suivant des perturbations modérées (Chapitre 1). J'ai pu identifier
trois mécanismes qui contribuent conjointement aux changements de
composition en altérant la trajectoire de succession après perturbation:
(1) la mortalité d'une espèce dominante; (2) le relâchement de la
croissance des espèces compagnes (Chapitre 2); et (3) le recrutement
accru des gaules des espèces compagnes et présentes dans le voisinage
(Chapitre 3).

Les conséquences au niveau des communautés des changements d'aires de
répartition spécifiques aux espèces pourraient mener à la formation de
communautés sans analogues, i.e.~des communautés dans lesquelles
coexistent des espèces dans des combinaisons historiquement inconnues
\citep{williams_novel_2007}. Dans la zone de transition entre le biome
boréal et tempéré, l'effet combiné du réchauffement et des perturbations
semblent particulièrement favorables à une poignée d'espèces seulement,
avec en première position \emph{Acer rubrum} (Chapitre 1-3). Les modèles
de distribution d'espèces sous l'effet du changement climatique
prédisent un grand potentiel d'augmentation de la richesse au Québec
\citep{berteaux_changements_2014}. Toutefois, si seul un nombre
restreint d'espèces tempérées prospèrent sous ce nouveau régime de
perturbations anthropiques, il est fort probable que les forêts de
l'écotone ne connaîtront pas un enrichissement, mais plutôt un
appauvrissement associé à l'expansion d'une ou quelques espèces. De
plus, que deviendront ces forêts si cette réorganisation se fait au
détriment des espèces boréales actuellement dominantes, notamment
\emph{Abies balsamea} et \emph{Picea spp.}? Le déclin de ces espèces
résineuses risque de s'accentuer dans les prochaines décennies dans les
domaines de la sapinière en raison du réchauffement climatique
\citep{dorangeville_beneficial_2018}. Le remplacement de ces espèces
résineuses par des espèces feuillues pourrait avoir des conséquences
économiques importantes. Par exemple, l'expansion de \emph{Acer rubrum}
pourrait compromettre l'approvisionnement en espèces résineuses à grande
valeur commerciale dans la forêt mixte. De plus, les pratiques
sylvicoles devront être révisées pour s'adapter à la nouvelle réalité
puisqu'elles sont élaborées en fonction de la composition et de la
dynamique naturelle des forêts et dépendent de la régénération naturelle
\citep{pinna_amenagement_2009}.

\hypertarget{duxe9placement-des-grands-biomes-forestiers}{%
\subsection{Déplacement des grands biomes
forestiers}\label{duxe9placement-des-grands-biomes-forestiers}}

Les effets cumulés des changements à l'échelle des individus, des
populations et des communautés, peuvent finalement provoquer des
changements de régime (\emph{regime shift}), dans lesquels le
déséquilibre pousse rapidement le système dans un nouvel état
\citep{scheffer_catastrophic_2001}. Ces changements de régime se
traduisent par un déplacement des grands biomes forestiers à l'échelle
régionale, notamment une expansion de la forêt tempérée au détriment de
la forêt mixte (Chapitre 2). Cette réorganisation régionale de la
composition des forêts peut interagir avec le fonctionnement des
écosystèmes à l'échelle locale et les processus à l'échelle globale
\citep[\emph{cross-scale
interaction};][]{peters_crossscale_2007, messier_managing_2013}. Ces
changements de fonctionnement risquent d'être d'autant plus grand à
l'écotone puisque les espèces feuillues et les espèces résineuses
présentent des différences importantes sur le plan de leurs
caractéristiques et fonctions écologiques
\citep{wardle_terrestrial_2011}. L'enfeuillement des forêts mixtes
pourrait par exemple influencer localement la qualité de la litière, le
taux de décomposition de la matière organique et la composition des
microorganismes du sol
\citep{laganiere_how_2010, legare_influence_2005}. Les changements dans
la composition, la structure d'abondance et la distribution spatiale des
forêts affecteront également la survie et la distribution des nombreuses
espèces de mammifères, d'oiseaux et d'insectes qui en dépendent pour
s'abriter et se nourrir \citep{friggens_effects_2018}. Comme les espèces
feuillues sont moins inflammables et moins sensibles aux insectes
ravageurs que les conifères, leur augmentation dans le paysage forestier
peut modifier le régime régional de perturbations
\citep{terrier_potential_2013, mffp_insectes_2018}. À long terme,
l'expansion du biome tempéré au détriment des forêts mixtes et des
forêts boréales pourrait avoir un effet sur le climat global en
augmentant la séquestration du carbone \citep{thurner_carbon_2014} ainsi
que l'albédo \citep{anderson_biophysical_2011}.

\hypertarget{lamuxe9nagement-forestier-dans-un-monde-en-changement-et-incertain}{%
\section{L'aménagement forestier dans un monde en changement et
incertain}\label{lamuxe9nagement-forestier-dans-un-monde-en-changement-et-incertain}}

Les effets multiples des changements globaux sur la dynamique forestière
soulèvent un défi majeur pour l'aménagement de nos forêts. Face à la
rapidité et à l'incertitude de ces changements, nos pratiques visant à
contrôler et prédire l'évolution de nos forêts risquent d'être
contreproductive \citep{puettmann_critique_2009}. Étant donné que les
coupes forestières ont une influence majeure sur la composition
(Chapitre 1), la dynamique (Chapitre 2) et la régénération des forêts
(Chapitre 3) et interagissent avec le changement du climat, il est
évident que les futures politiques d'aménagement auront un rôle
fondamental à jouer pour aider les forêts à s'adapter rapidement et
faire face aux changements globaux.

Le réchauffement que nous avons connu jusqu'à présent n'est que mineur
par rapport à ce qui est attendu d'ici la fin du siècle. Néanmoins, tel
que mis en évidence dans l'ensemble de ma thèse, de grandes
transformations sont déjà évidentes à toutes les échelles de
l'organisation biologique. Avec l'accélération des changements
environnementaux et l'inertie inhérente de la dynamique forestière, le
déséquilibre ne pourra que s'accentuer et les réponses des forêts
dépendront des dynamiques transitoires déjà en cours. Actuellement, les
principes fondamentaux de l'estimation de la productivité des
peuplements forestiers reposent sur des taux de mortalité et de
croissance prévisibles sous un climat constant. En aménagement
forestier, on présume que le climat est stable et que les forêts sont à
équilibre. À court terme, ces hypothèses sont valables. Mais, à long
terme, elles sont particulièrement problématiques dans le contexte du
changement climatique. En effet, comme la dynamique forestière n'est pas
à l'équilibre \citep{talluto_extinction_2017}, la trajectoire de
succession et la niche de régénération sont facilement altérées sous
l'effet combiné du réchauffement et des perturbations (Chapitre 2, 3).
Les prédictions issues de calculs qui supposent l'équilibre pourraient à
la fois sur-estimer la capacité de régénération et de croissance de
certaines espèces et sous-estimer la mortalité associée à des extrêmes
climatiques, menant ainsi à une planification mal adaptée. Par
conséquent, les activités de gestion doivent non seulement anticiper le
changement, mais aussi reconnaître que les systèmes actuels ont déjà été
transformés et sont en train de se transformer davantage. L'importance
de ce point a été bien exprimée par \citet{seastedt_management_2008} :

\begin{quote}
In managing novel ecosystems, the point is not to think outside the box
but to recognize that the box has moved, and in the 21st century, the
box will continue to move more rapidly.
\end{quote}

\hypertarget{uxe9volution-de-lamuxe9nagement-au-quuxe9bec}{%
\subsection{Évolution de l'aménagement au
Québec}\label{uxe9volution-de-lamuxe9nagement-au-quuxe9bec}}

Une importante volonté de gestion durable des forêts s'est développée
dans les dernières décennies à travers le monde. Jusqu'à la fin du
XX\textsuperscript{e} siècle, les modèles de gestion ont une vision
utililariste de la forêt et sont centrés sur la production
\citep{kuuluvainen_natural_2012}. Depuis les années 1990, la foresterie
a évolué vers un aménagement qui intègre davantage de critères
écologiques et sociaux \citep{messier_managing_2013}. Au Québec, dans la
foulée du documentaire \emph{L'erreur boréale}, sorti en 1999, une
grande réflexion s'est amorcée sur l'exploitation de la forêt publique.
Pour répondre aux inquiétudes de la population, la Commission d'étude
sur la gestion de la forêt publique québécoise est formée en 2003 et
dépose un rapport en 2004 qui fait état de la situation et formule de
nombreuses recommandations pour améliorer et moderniser la gestion des
forêts
\citep{commission_detude_sur_la_gestion_de_la_foret_publique_quebecoise_commission_2004}.
En réponse à ces recommandations, le Québec se dote la Loi sur
l'aménagement durable du territoire forestier, en vigueur depuis 2013,
qui promeut un aménagement écosystémique. L'aménagement écosystémique
s'inspire des patrons spatio-temporels générés par les perturbations
naturelles, qui servent d'états de référence, afin de maintenir les
écosystèmes dans leur plage de variabilité naturelle historique, et
ainsi réduire les écarts entre la forêt naturelle et aménagée
\citep{vaillancourt_implementation_2009, attiwill_disturbance_1994}.

L'aménagement écosystémique représente une grande avancée car il intègre
un ensemble d'objectifs sociaux et écologiques plus larges et reconnait
l'importance de la biodiversité et des processus écologiques qui
influencent la dynamique forestière
\citep{messier_functional_2019, kuuluvainen_forest_2009}. Cette nouvelle
approche de gestion présente néanmoins une lacune majeure; ces pratiques
de gestion ne sont pas conçues pour faire face au rythme rapide des
changements globaux et à l'incertitude croissante qui en découle
\citep{messier_dealing_2016, millar_climate_2007}. En effet, des
pratiques de gestion qui visent à maintenir la composition et la
structure des forêts historiques de référence vont devenir de plus en
plus difficile à mettre en oeuvre
\citep{duveneck_measuring_2016, boulanger_climate_2019} et ne permettent
pas nécessairement d'améliorer la capacité des écosystèmes à s'adapter à
un nouvel ensemble de conditions environnementales
\citep{seastedt_management_2008}. Bien que l'utilisation stricte des
conditions de référence historiques deviendra contre-productive en tant
qu'objectifs spécifiques, les informations historiques documentant la
dynamique naturelle des écosystèmes forestiers seront essentielles pour
mieux comprendre les dynamiques du futur \citep{harris_ecological_2006}.

\hypertarget{amuxe9nager-les-foruxeats-pour-augmenter-leur-ruxe9silience-et-leur-capacituxe9-adaptative}{%
\subsection{Aménager les forêts pour augmenter leur résilience et leur
capacité
adaptative}\label{amuxe9nager-les-foruxeats-pour-augmenter-leur-ruxe9silience-et-leur-capacituxe9-adaptative}}

La solution privilégiée pour faire face au changement climatique est
d'accroître la résilience et la capacité adaptative des forêts
\citep{messier_managing_2013, seastedt_management_2008}. Les
perturbations naturelles et les variations climatiques sont inévitables
mais en développant une grande diversité, les forêts auront les outils
nécessaires pour se réorganiser et s'adapter à des conditions futures
sans précédent \citep{messier_dealing_2016}. S'appuyant sur l'hypothèse
d'assurance \citep[de l'anglais \emph{insurance
hypothesis};][]{yachi_biodiversity_1999}, l'idée est de favoriser la
diversité génétique, spécifique, fonctionnelle et structurale dans les
forêts afin d'augmenter les chances que certaines espèces continueront à
assurer le fonctionnement de l'écosystème même si d'autres
disparaissent.

Pour favoriser la capacité adaptative des forêts, il faut avant tout
maintenir la diversité naturelle que l'on trouve à toutes les échelles
spatiales, du peuplement jusqu'au biome, de manière à garder les options
d'adaptation qui existent déjà. Par exemple, contrairement aux forêts
boréales, la diversité des forêts mixtes leur confère une meilleure
capacité d'adaptation; elles ont plus de trajectoires possibles pour
réagir aux changements (Chapitre 2). En effet, suite à une perturbation,
des trajectoires diversifiées peuvent émerger dans les peuplements
présentant une hétérogénéité en termes de structure, d'âge, de tolérance
physiologique et de stratégies d'histoire de vie. Dans certains cas, il
sera peut-être nécessaire de cultiver activement la capacité adaptative
des écosystèmes grâce à l'aménagement. Ce principe pourrait devenir
important dans les forêts boréales étant donné leur composition très
homogène et leur très grande inertie face au changement climatique
(Chapitres 1, 2). Pour l'instant, les espèces tempérées semblent avoir
de la difficulté à s'établir naturellement en forêt boréale (Chapitre
3). Les plantations d'enrichissement pourraient alors s'avérer utiles
pour faciliter la migration des espèces tempérées vers le nord et
assurer la résilience des forêts \citep{duveneck_measuring_2016}. La
création d'îlots de forêts mixtes sur les sommets de collines dans les
forêts boréales assurerait la présence de semenciers de différentes
espèces capables de coloniser rapidement les sites après perturbation
lorsque les conditions seront adéquates (Chapitre 3). Finalement, étant
donné les interactions entre échelles, les changements de régime
écologique et la variabilité des réponses des espèces, il devient de
plus en plus clair qu'on ne peut forcer un peuplement à se développer
dans une direction précise prédéterminée en fonction de nos besoins en
bois \citep{puettmann_critique_2009}. Des recherches récentes
encouragent donc à revoir la planification de façon à avoir des
objectifs plus larges et plus flexibles qui permettent un ensemble de
différentes trajectoires futures à l'échelle régionale
\citep{messier_dealing_2016, puettmann_critique_2009}.

Cette idée de favoriser la diversité pour assurer la résilience est déjà
prise en compte dans l'aménagement écosystémique et constitue donc une
porte d'entrée à l'adaptation aux changements climatiques
\citep{samuel_foret_2011}. Toutefois, il faut éviter de mettre tous nos
oeufs dans le même panier. Les modèles de projections climatiques nous
ont informé d'un risque croissant de vagues de chaleur et de sécheresses
\citep{ipcc_climate_2014}. Par conséquent, il apparaît logique de mettre
l'emphase sur la promotion des espèces qui résistent à la sécheresse.
Mais, dans un contexte d'incertitude, cette stratégie ne suffit pas
puisqu'il est possible que ce ne soit pas la sécheresse qui causera le
plus grand stress aux forêts, mais plutôt l'augmentation de la fréquence
des feux, l'arrivée de nouveaux insectes ravageurs ou encore les
variations de températures printanières. De plus, tous ces facteurs de
risque peuvent interagir entre eux et mener à des changements rapides et
inattendus des écosystèmes forestiers. La grande incertitude associée
aux prédictions des effets attendus des changements climatiques doit
être intégrée dans la gestion forestière de façon à prendre en compte du
large éventail de vulnérabilités \citep{messier_dealing_2016}. Pour
permettre à l'écosystème de résister ou s'adapter à ces multiples
facteurs de stress, les politiques de sélection des espèces d'arbres
pourraient, par exemple, promouvoir le mélange d'espèces ayant des
caractéristiques fonctionnelles diverses, allant des tolérances
physiologiques (aux feux, aux ravageurs, à la sécheresse), aux modes de
régénération (e.g., banque de graines, cônes sérotineux, reproduction
végétative) et aux stratégies de croissance
\citep{messier_functional_2019, puettmann_critique_2009}.

\hypertarget{au-deluxe0-des-foruxeats-aplanir-la-courbe}{%
\section{Au-delà des forêts --- Aplanir la
courbe}\label{au-deluxe0-des-foruxeats-aplanir-la-courbe}}

\begin{quote}
Aujourd'hui, il faut revoir nos approches de prévention, de préparation
et d'intervention. Des plans particuliers orientés sur un aléa ou une
conséquence ont fait leurs temps. Nous gérons des conséquences multiples
et des effets domino à grandes échelles, qui s'enchaînent de manière de
plus en plus rapprochée.
\end{quote}

\begin{itemize}
\tightlist
\item
  L'Actualité, 26 avril 2020
\end{itemize}

Cet extrait de L'Actualité traitait de la gestion des risques associés
aux inondations printanières en pleine pandémie de COVID-19. Au cours
des premiers mois de 2020, et au moment où j'écris ces lignes, les
gouvernements du monde entier ont adopté des mesures draconiennes pour
tenter d'atténuer la menace de la COVID-19. Le message a été clair et
bien compris par l'ensemble de la population: il faut aplanir la courbe
en ralentissant le rythme de propagation de la maladie pour éviter de
dépasser la capacité des hôpitaux à traiter les malades.

Ce même concept s'applique également à la crise climatique: la
limitation du réchauffement climatique donnerait aux sociétés et aux
écosystèmes une plus grande marge de manoeuvre pour s'adapter sans
dépasser la capacité de support de la Terre et des systèmes humains. En
d'autres mots, si nous n'agissons pas maintenant pour décarboniser notre
économie, le climat mondial continuera de se dérégler et la
multiplication des catastrophes environnementales dépassera notre
capacité à les gérer. Le casse-tête de la gestion des inondations
pendant la pandémie de COVID-19 à Fort McMurray est un bon exemple des
difficultés à gérer plusieurs catastrophes en même temps. Les
catastrophes naturelles en cascade sont un autre exemple. Au cours des
dernières décennies, l'ouest du Canada a subi une épidémie sans
précédent de dendroctone du pin ponderosa, une grave sécheresse
(2001-2003) et des saisons de feux extrêmes
\citep{michaelian_massive_2011, williamson_climate_2009}. Ces
perturbations peuvent apparaître soudainement et avoir des effets
synergiques qui excédent la capacité du gouvernement à contrôler les
dommages.

Le récent rapport spécial du GIEC a conclu que limiter le réchauffement
climatique à 1.5 \(^{\circ}\)C est possible mais exigera des transitions
rapides et radicales dans tous les aspects de la société
\citep{ipcc_summary_2018}. La réponse à la pandémie montre que nous
pouvons mettre en place rapidement des mesures d'urgences radicales. Par
contre, cette crise sanitaire nous montre aussi qu'il est préférable
d'adopter des mesures préventives afin d'atténuer les changements
climatiques plutôt que d'attendre passivement de frapper un mur et être
forcé de vivre dans l'état d'urgence avec des mesures extrêmes.


 %\cleardoublepage


% \bibliographystyle{abbrvnat}
% \bibliographystyle{plain-fr}
\bibliographystyle{apalike-uqam}
\bibliography{references.bib}


 % Pour les annexes :


 % Les annexes se font comme les chapitres. Le fichier
 % commence par \francais ou \anglais et ensuite
 % \chapter{..}. Le reste est parreil à un chapitre normal.
%%%%%%%%%%%%%%%%%%%%%%%%%%%%%%%%%%%%%%%%%%%%%%%%%%%%%%%%%%%%
%%%%%%%%%%%%%%%%%%%%                   %%%%%%%%%%%%%%%%%%%%%
%%%%%%%%%%%%%%%%%%%%   A N N E X E     %%%%%%%%%%%%%%%%%%%%%
%%%%%%%%%%%%%%%%%%%%                   %%%%%%%%%%%%%%%%%%%%%
%%%%%%%%%%%%%%%%%%%%%%%%%%%%%%%%%%%%%%%%%%%%%%%%%%%%%%%%%%%%

%%%%%% Annexe article 1 %%%%

\appendix

\renewcommand\thefigure{A.\arabic{figure}}
\renewcommand\thetable{A.\arabic{table}}

% Options for packages loaded elsewhere
\PassOptionsToPackage{unicode}{hyperref}
\PassOptionsToPackage{hyphens}{url}
%
\documentclass[
]{article}
\usepackage{lmodern}
\usepackage{amssymb,amsmath}
\usepackage{ifxetex,ifluatex}
\ifnum 0\ifxetex 1\fi\ifluatex 1\fi=0 % if pdftex
  \usepackage[T1]{fontenc}
  \usepackage[utf8]{inputenc}
  \usepackage{textcomp} % provide euro and other symbols
\else % if luatex or xetex
  \usepackage{unicode-math}
  \defaultfontfeatures{Scale=MatchLowercase}
  \defaultfontfeatures[\rmfamily]{Ligatures=TeX,Scale=1}
\fi
% Use upquote if available, for straight quotes in verbatim environments
\IfFileExists{upquote.sty}{\usepackage{upquote}}{}
\IfFileExists{microtype.sty}{% use microtype if available
  \usepackage[]{microtype}
  \UseMicrotypeSet[protrusion]{basicmath} % disable protrusion for tt fonts
}{}
\makeatletter
\@ifundefined{KOMAClassName}{% if non-KOMA class
  \IfFileExists{parskip.sty}{%
    \usepackage{parskip}
  }{% else
    \setlength{\parindent}{0pt}
    \setlength{\parskip}{6pt plus 2pt minus 1pt}}
}{% if KOMA class
  \KOMAoptions{parskip=half}}
\makeatother
\usepackage{xcolor}
\IfFileExists{xurl.sty}{\usepackage{xurl}}{} % add URL line breaks if available
\IfFileExists{bookmark.sty}{\usepackage{bookmark}}{\usepackage{hyperref}}
\hypersetup{
  pdftitle={Supplementary Information},
  hidelinks,
  pdfcreator={LaTeX via pandoc}}
\urlstyle{same} % disable monospaced font for URLs
\usepackage[margin=1in]{geometry}
\usepackage{longtable,booktabs}
% Correct order of tables after \paragraph or \subparagraph
\usepackage{etoolbox}
\makeatletter
\patchcmd\longtable{\par}{\if@noskipsec\mbox{}\fi\par}{}{}
\makeatother
% Allow footnotes in longtable head/foot
\IfFileExists{footnotehyper.sty}{\usepackage{footnotehyper}}{\usepackage{footnote}}
\makesavenoteenv{longtable}
\usepackage{graphicx,grffile}
\makeatletter
\def\maxwidth{\ifdim\Gin@nat@width>\linewidth\linewidth\else\Gin@nat@width\fi}
\def\maxheight{\ifdim\Gin@nat@height>\textheight\textheight\else\Gin@nat@height\fi}
\makeatother
% Scale images if necessary, so that they will not overflow the page
% margins by default, and it is still possible to overwrite the defaults
% using explicit options in \includegraphics[width, height, ...]{}
\setkeys{Gin}{width=\maxwidth,height=\maxheight,keepaspectratio}
% Set default figure placement to htbp
\makeatletter
\def\fps@figure{htbp}
\makeatother
\setlength{\emergencystretch}{3em} % prevent overfull lines
\providecommand{\tightlist}{%
  \setlength{\itemsep}{0pt}\setlength{\parskip}{0pt}}
\setcounter{secnumdepth}{-\maxdimen} % remove section numbering
\usepackage{setspace}
\setstretch{1,5}
\usepackage{lineno}
\usepackage{lscape}
\linenumbers

\title{Supplementary Information}
\date{}

\begin{document}
\maketitle

\hypertarget{supplementary-tables}{%
\section{Supplementary Tables}\label{supplementary-tables}}

\textbf{Table S1}. List of species included in the analyses and their
traits. The species groups were defined using their trait values and
knowledge of species ecology. Temperate species have temperature indices
above 4.25, and boreal species below 4.25. Pioneer species have shade
tolerance below 2.6 and are generally found in disturbed habitats.

\begin{longtable}[]{@{}lllll@{}}
\toprule
Species name & Vernacular name & Group & Shade tolerance & Temperature
index\tabularnewline
\midrule
\endhead
Abies balsamea & Balsam fir & Boreal & 5.0 & 3.16\tabularnewline
Acer pensylvanicum & Striped maple & Temperate & 3.5 &
5.22\tabularnewline
Acer rubrum & Red maple & Temperate & 3.4 & 9.28\tabularnewline
Acer saccharinum & Silver maple & Temperate & 3.6 & 9.97\tabularnewline
Acer saccharum & Sugar maple & Temperate & 4.8 & 6.93\tabularnewline
Acer spicatum & Mountain maple & Temperate & 3.3 & 4.52\tabularnewline
Alnus incana & Speckled alder & Boreal & 1 & 1.22\tabularnewline
Amelanchier sp. & Serviceberry & Temperate & 3.4 & 9.40\tabularnewline
Betula alleghaniensis & Yellow birch & Temperate & 3.2 &
4.49\tabularnewline
Betula papyrifera & White birch & Pioneer & 1.5 & 3.69\tabularnewline
Betula populifolia & Grey birch & Pioneer & 1.5 & 5.58\tabularnewline
Carpinus caroliniana & Blue beech & Temperate & 4.6 &
15.90\tabularnewline
Carya cordiformis & Bitternut hickory & Temperate & 2.1 &
11.06\tabularnewline
Fagus grandifolia & American beech & Temperate & 4.8 &
8.46\tabularnewline
Fraxinus americana & White ash & Temperate & 2.5 & 9.54\tabularnewline
Fraxinus nigra & Black ash & Temperate & 3 & 4.92\tabularnewline
Fraxinus pennsylvanica & Red ash & Temperate & 3.1 &
11.86\tabularnewline
Juglans cinerea & Butternut & Temperate & 1.9 & 8.10\tabularnewline
Larix laricina & Tamarack & Boreal & 1 & 3.92\tabularnewline
Malus sp. & Crab apple & Temperate & 2.2 & 7.96\tabularnewline
Ostrya virginiana & Ironwood & Temperate & 4.6 & 8.91\tabularnewline
Picea glauca & White spruce & Boreal & 4.2 & 3.08\tabularnewline
Picea mariana & Black spruce & Boreal & 4.1 & 1.68\tabularnewline
Picea rubens & Red spruce & Temperate & 4.4 & 4.26\tabularnewline
Pinus banksiana & Jack pine & Boreal & 1.4 & 2.99\tabularnewline
Pinus resinosa & Red pine & Temperate & 1.9 & 5.54\tabularnewline
Pinus strobus & Eastern white pine & Temperate & 3.2 &
6.85\tabularnewline
Populus balsamifera & Balsam poplar & Pioneer & 1.3 &
4.25\tabularnewline
Populus deltoides & Cottonwood & Pioneer & 1.8 & 8.12\tabularnewline
Populus grandidentata & Large tooth aspen & Pioneer & 1.2 &
6.14\tabularnewline
Populus tremuloides & Trembling aspen & Pioneer & 1.2 &
4.22\tabularnewline
Prunus pensylvanica & Pin cherry & Pioneer & 1 & 4.01\tabularnewline
Prunus serotina & Black cherry & Temperate & 2.5 & 4.69\tabularnewline
Prunus virginiana & Chokecherry & Temperate & 2.6 & 7.79\tabularnewline
Quercus alba & White oak & Temperate & 2.9 & 12.95\tabularnewline
Quercus bicolor & Swamp white oak & Temperate & 3 & 9.51\tabularnewline
Quercus macrocarpa & Bur oak & Temperate & 2.7 & 6.72\tabularnewline
Quercus rubra & Red oak & Temperate & 2.8 & 9.67\tabularnewline
Salix sp. & Willow & Pioneer & 1.5 & 13.32\tabularnewline
Sorbus sp. & Mountain-ash & Pioneer & 2.6 & 2.31\tabularnewline
Thuja occidentalis & White cedar & Temperate & 3.5 & 4.30\tabularnewline
Tilia americana & Basswood & Temperate & 4 & 5.34\tabularnewline
Tsuga canadensis & Eastern hemlock & Temperate & 4.8 &
6.87\tabularnewline
Ulmus americana & American elm & Temperate & 3.1 & 10.67\tabularnewline
Ulmus rubra & Red elm & Temperate & 3.3 & 12.37\tabularnewline
Ulmus thomasii & Rock elm & Temperate & 3.2 & 7.80\tabularnewline
\bottomrule
\end{longtable}

\pagebreak

\textbf{Table S2}. 21 original disturbance types and their
reclassification into natural disturbances and harvest, with three
levels of intensity. Sites with tree planting were excluded from the
study.

\begin{longtable}[]{@{}llll@{}}
\toprule
& Original disturbance types & Reclassification & Disturbance
level\tabularnewline
\midrule
\endhead
1 & Improvement cutting & Harvest & Moderate\tabularnewline
2 & Strip cutting & Harvest & Moderate\tabularnewline
3 & Checkerboard~clear-cutting & Harvest & Moderate\tabularnewline
4 & Diameter-limit cutting & Harvest & Moderate\tabularnewline
5 & Selection cutting & Harvest & Moderate\tabularnewline
6 & Partial cutting & Harvest & Moderate\tabularnewline
7 & Diameter-limit cutting with crop tree release & Harvest &
Moderate\tabularnewline
8 & Commercial~thinning & Harvest & Moderate\tabularnewline
9 & Partial cutting with light outbreak & Harvest &
Moderate\tabularnewline
10 & Partial burn & Natural & Moderate\tabularnewline
11 & Light outbreak & Natural & Moderate\tabularnewline
12 & Partial windfall & Natural & Moderate\tabularnewline
13 & Partial ice storm & Natural & Moderate\tabularnewline
14 & Partial decline & Natural & Moderate\tabularnewline
15 & Final strip cutting & Harvest & Major\tabularnewline
16 & Harvesting~with protection of regeneration & Harvest &
Major\tabularnewline
17 & Clearcutting & Harvest & Major\tabularnewline
18 & Total burn & Natural & Major\tabularnewline
19 & Severe outbreak & Natural & Major\tabularnewline
20 & Total windfall & Natural & Major\tabularnewline
21 & Total decline & Natural & Major\tabularnewline
- & Seeding & Plantation & x\tabularnewline
- & Planting & Plantation & x\tabularnewline
- & Planting bare-rooted seedlings & Plantation & x\tabularnewline
- & Container planting & Plantation & x\tabularnewline
\bottomrule
\end{longtable}

\pagebreak

\textbf{Table S3}. List of R packages used for analyses.

\begin{longtable}[]{@{}lll@{}}
\toprule
\begin{minipage}[b]{0.15\columnwidth}\raggedright
Packages\strut
\end{minipage} & \begin{minipage}[b]{0.48\columnwidth}\raggedright
Uses\strut
\end{minipage} & \begin{minipage}[b]{0.29\columnwidth}\raggedright
References\strut
\end{minipage}\tabularnewline
\midrule
\endhead
\begin{minipage}[t]{0.15\columnwidth}\raggedright
adespatial\strut
\end{minipage} & \begin{minipage}[t]{0.48\columnwidth}\raggedright
Forward selection (\texttt{forward.sel}), temporal beta diversity
(\texttt{tbi})\strut
\end{minipage} & \begin{minipage}[t]{0.29\columnwidth}\raggedright
Dray et al. (2018)\strut
\end{minipage}\tabularnewline
\begin{minipage}[t]{0.15\columnwidth}\raggedright
FD\strut
\end{minipage} & \begin{minipage}[t]{0.48\columnwidth}\raggedright
Functional composition (\texttt{functcomp})\strut
\end{minipage} & \begin{minipage}[t]{0.29\columnwidth}\raggedright
Laliberté et al. (2014)\strut
\end{minipage}\tabularnewline
\begin{minipage}[t]{0.15\columnwidth}\raggedright
raster\strut
\end{minipage} & \begin{minipage}[t]{0.48\columnwidth}\raggedright
Manipulation of spatial data\strut
\end{minipage} & \begin{minipage}[t]{0.29\columnwidth}\raggedright
Hijmans (2018)\strut
\end{minipage}\tabularnewline
\begin{minipage}[t]{0.15\columnwidth}\raggedright
sf\strut
\end{minipage} & \begin{minipage}[t]{0.48\columnwidth}\raggedright
Manipulation of spatial data\strut
\end{minipage} & \begin{minipage}[t]{0.29\columnwidth}\raggedright
Pebesma (2018)\strut
\end{minipage}\tabularnewline
\begin{minipage}[t]{0.15\columnwidth}\raggedright
stats\strut
\end{minipage} & \begin{minipage}[t]{0.48\columnwidth}\raggedright
Linear regressions (\texttt{lm})\strut
\end{minipage} & \begin{minipage}[t]{0.29\columnwidth}\raggedright
R Core Team (2018)\strut
\end{minipage}\tabularnewline
\begin{minipage}[t]{0.15\columnwidth}\raggedright
vegan\strut
\end{minipage} & \begin{minipage}[t]{0.48\columnwidth}\raggedright
Variation partitioning (\texttt{varpart})\strut
\end{minipage} & \begin{minipage}[t]{0.29\columnwidth}\raggedright
Oksanen et al. (2019)\strut
\end{minipage}\tabularnewline
\begin{minipage}[t]{0.15\columnwidth}\raggedright
zoo\strut
\end{minipage} & \begin{minipage}[t]{0.48\columnwidth}\raggedright
Rolling average (\texttt{rollmean})\strut
\end{minipage} & \begin{minipage}[t]{0.29\columnwidth}\raggedright
Zeileis \& Grothendieck (2005)\strut
\end{minipage}\tabularnewline
\bottomrule
\end{longtable}

\hypertarget{supplementary-figures}{%
\section{Supplementary Figures}\label{supplementary-figures}}

\includegraphics[width=3in,height=\textheight]{ms/figures/figS1_clim_trend.pdf}

\textbf{Figure S1}. Temporal trends in growing season temperatures
(top), total growing season precipitation (middle) and annual climate
moisture index (bottom). Grey lines represent averaged climate values
across the 6281 studied forest plots. Straight black lines show the
fitted least-squared linear regression lines.

\pagebreak

\includegraphics[width=4.5in,height=\textheight]{ms/figures/figS2_disturb.pdf}

\textbf{Figure S2}. Frequency of forest plots by disturbance type
(natural disturbances and harvest), level of intensity (minor, moderate,
major) and timing (old refers to disturbances that occurred before the
study period whereas recent disturbances occurred during the study
period). The three columns in each disturbance type sum to \emph{n} =
6281 forest plots, but many forest plots have been disturbed by more
than one type of disturbance during the study period.

\pagebreak

\includegraphics[width=6.6in,height=\textheight]{ms/figures/figS3_beta_calculus.pdf}

\textbf{Figure S3.} Equations to compute the temporal ß diversity index,
as well as its components, using the Ružička coefficient for abundance
data (a) and an example (b) where the tree composition of a single
forest plot is compared between two surveys, \(t_1\) and \(t_2\). In the
example, each of the \(n\) species is represented by a symbol. The
symbols in yellow represent the abundance of a species that is common to
the two survey (component A; note that it can be different individuals
of the same species). The symbols in red represent the abundance of a
species that is lost between \(t_1\) and \(t_2\) (component B). The
symbols in blue represent the abundance of a species that is gained
between \(t_1\) and \(t_2\) (component C). In this example,
\(A = 4 + 3 + 2 = 9\), \(B = 2\), and \(C = 3 + 1 = 4\), therefore
\(\beta = 2+4/(9+2+4) = 0.4\).

\pagebreak

\includegraphics[width=6.6in,height=\textheight]{ms/figures/figS4_ternary.pdf}

\textbf{Figure S4}. Triangular diagrams of gains and losses in tree
abundance by bioclimatic domains and disturbance levels. Each point
represents a forest plot and the large black point represents the
centroid. At the upper tip of the triangle, similarity is high (ß = 0;
blue colors). At the base of the triangle, dissimilarity is high (ß =
1). On the left, forests in red are dominated by losses, while on the
right, forests in green are dominated by gains. The similar
distributions of gain and loss values in the ternary diagrams suggests
that there is no major difference in temporal ß diversity patterns among
domains.

\pagebreak

\includegraphics[width=6.6in,height=\textheight]{ms/figures/figS5_spchange.pdf}

\textbf{Figure S5.} Species temporal changes for Québec forests and for
each bioclimatic domain. Changes in species occurrence (left) and
species abundance (right). Only the species occupying more than 20 plots
are shown. The bars represent the mean changes across the study area,
while the colored points represent the mean changes by bioclimatic
domain. Stars represent the levels of the significance of the
\emph{p}-value (* \emph{p} \textless{} 0.05; ** \emph{p} \textless{}
0.01; *** \emph{p} \textless{} 0.001) associated with Wilcoxon
signed-rank tests used to determine whether individual species changes
in occurrence and abundance were significant. An increase in occurrence
indicates that the species has spread regionally, while an increase in
abundance indicates that the species has spread locally and/or
regionally. Letters next to species names correspond to (T)emperate;
(P)ioneer and (B)oreal species.

\pagebreak

\includegraphics[width=6.6in,height=\textheight]{ms/figures/figS6_CTIvsGains.pdf}

\textbf{Figure S6.} Relations between change in Community Temperature
Index (\(\Delta\)CTI) and gains in temperate (top), gains in pioneer
(middle) and losses in boreal species (bottom). In each panel, the slope
and adjusted \(R^2\) of a linear regression model are shown.

\pagebreak

\hypertarget{references}{%
\section*{References}\label{references}}
\addcontentsline{toc}{section}{References}

\hypertarget{refs}{}
\leavevmode\hypertarget{ref-dray_adespatial_2018}{}%
Dray, S., Bauman, D., Blanchet, G., Borcard, D., Clappe, S., Guenard,
G., Jombart, T., Larocque, G., Legendre, P., Madi, N., \& Wagner, H. H.
(2018). \emph{Adespatial: Multivariate Multiscale Spatial Analysis}.
\url{https://CRAN.R-project.org/package=adespatial}

\leavevmode\hypertarget{ref-hijmans_raster_2018}{}%
Hijmans, R. J. (2018). \emph{Raster: Geographic Data Analysis and
Modeling}. \url{https://CRAN.R-project.org/package=raster}

\leavevmode\hypertarget{ref-laliberte_fd_2014}{}%
Laliberté, E., Legendre, P., \& Shipley, B. (2014). \emph{FD: Measuring
functional diversity from multiple traits, and other tools for
functional ecology}.

\leavevmode\hypertarget{ref-oksanen_vegan_2019}{}%
Oksanen, J., Blanchet, F. G., Friendly, M., Kindt, R., Legendre, P.,
McGlinn, D., Minchin, P. R., O'Hara, R. B., Simpson, G. L., Solymos, P.,
Stevens, M. H. H., Szoecs, E., \& Wagner, H. (2019). \emph{Vegan:
Community Ecology Package}.
\url{https://CRAN.R-project.org/package=vegan}

\leavevmode\hypertarget{ref-pebesma_simple_2018}{}%
Pebesma, E. (2018). Simple Features for R: Standardized Support for
Spatial Vector Data. \emph{The R Journal}.
\url{https://journal.r-project.org/archive/2018/RJ-2018-009/index.html}

\leavevmode\hypertarget{ref-r_core_team_r_2018}{}%
R Core Team. (2018). \emph{R: A Language and Environment for Statistical
Computing}. R Foundation for Statistical Computing.
\url{https://www.R-project.org/}

\leavevmode\hypertarget{ref-zeileis_zoo_2005}{}%
Zeileis, A., \& Grothendieck, G. (2005). Zoo: S3 Infrastructure for
Regular and Irregular Time Series. \emph{Journal of Statistical
Software}, \emph{14}(6), 1--27.
\url{https://doi.org/10.18637/jss.v014.i06}

\end{document}


%%%% Annexe article 2 %%%%



\renewcommand\thefigure{B.\arabic{figure}}
\renewcommand\thetable{B.\arabic{table}}

\include{article2/annexe2}

%%%% Annexe article 3 %%%%



\renewcommand\thefigure{C.\arabic{figure}}
\renewcommand\thetable{C.\arabic{table}}

\include{article3/annexe3}

\end{document}

\endinput
%%
%% End of file `gabaritTPA.tex'.
