\anglais

\chapter*{Abstract}

The main objective of my thesis was to understand and quantify the
combined effect of recent climate change and disturbances on forest
community dynamics in the boreal-temperate ecotone over the past
decades. To do so, I analysed the evolution of forest composition,
transition dynamics and recruitment dynamics in Quebec from 1970 to
2018.

In the first chapter, I showed that natural and anthropogenic
disturbances are the main drivers of forest compositional change,
i.e.~temporal \(\beta\) diversity. Despite the prevalence of
disturbances, analysis of community ecological traits revealed a
thermophilization of forests across Québec, i.e.~an increase of
warm-adapted species at the expense of cold-adapted species. This
thermophilization was further amplified by moderate disturbances,
leading to a new question: if disturbances can favour such a community
reorganization, could they catalyse a permanent shift to alternative
states?

The second chapter provided answers to this question with an analysis of
forest transition dynamics in Québec based on a four-state model,
i.e.~boreal, mixed, temperate and pioneer. Transition dynamics are
primarily influenced by disturbances and secondarily by climate and
edaphic conditions. Major disturbances mainly trigger transitions to the
pioneer state, while moderate disturbances promote transitions from
mixed to temperate states. In the long run, moderate disturbances may
catalyse a faster northward shift of the temperate-boreal ecotone under
climate change. However, contrary to my expectations, temperate species
recruitment played a negligible role in this dynamic compared to
mortality and growth processes.

The first two chapters were focussing on the evolution of mature tree
communities. However, to understand forest dynamics, a detailed analysis
of regeneration dynamics is necessary. Such a study was conducted in the
third chapter where I highlighted northward shifts of several kilometres
for the saplings of \emph{Acer rubrum}, \emph{Acer saccharum} and
\emph{Betula alleghaniensis} in undisturbed forests. However, under the
influence of moderate disturbances, only \emph{Acer spp.} had migrated
and there were no shifts under the influence of major disturbances. In
contrast, I found no evidence of migration for the saplings of
\emph{Fagus grandifolia}. Although partial cutting increased recruitment
success of all four species, it did not result in larger northward range
shifts, presumably because recruitment was constrained by short-distance
dispersal, strong competition by boreal species, and unfavourable
edaphic conditions.

Overall, my results highlighted that forest communities in the
temperate-boreal ecotone are already changing in response to recent
climate warming and that disturbances are accelerating this response.
While climate warming erodes the resilience of mixed forests,
disturbances remove resident boreal species, thereby accelerating the
successional process and facilitating the establishment of temperate
species.

\vspace{1cm}

\textbf{Keywords}

Beta diversity, Boreal-temperate ecotone, Climate change, Community
ecology, Equilibrium and transient dynamics, Forests, Natural
disturbances and logging, Québec Forest Inventory, Thermophilization,
Transition dynamics, Tree migration, Tree sapling recruitment.
