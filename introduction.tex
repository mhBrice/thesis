\francais

\chapter*{Introduction}

\hypertarget{introduction-guxe9nuxe9rale}{%
\section{Introduction générale}\label{introduction-guxe9nuxe9rale}}

L'humain est aujourd'hui une force prédominante gouvernant les processus
écologiques, amenant de nombreux chercheurs à suggérer que le système
terrestre a basculé dans une nouvelle ère géologique, l'Anthropocène
\citep{crutzen_geology_2002}. Depuis environ un siècle, les activités
humaines ont largement perturbé l'équilibre dynamique des cycles
naturels. Le développement des sociétés occidentales s'est basé sur
l'industrialisation et l'exploitation des ressources, résultant
notamment en un changement d'utilisation des sols associé à la
fragmentation et la dégradation des habitats, ainsi qu'à un relargage
massif de gaz à effet de serre (e.g.~CO2, SOx, CH4, NOx\ldots) dans
l'atmosphère. Les changements environnementaux récents se caractérisent
par leur vitesse et leur intensité. La recherche scientifique
contemporaine s'intéresse à comprendre, évaluer et prédire l'impact de
ces perturbations sur les écosystèmes et les communautés (McGill et al.,
2015; Root et al., 2003; Sala et al., 2000; Vellend et al., 2017).

\hypertarget{changements-climatiques}{%
\section{Changements climatiques}\label{changements-climatiques}}

Le réchauffement climatique mesuré sur l'ensemble de la planète durant
les dernières décennies est sans équivoque, et la responsabilité de
l'humain par l'émission de gaz à effet de serre (abrégé GES par la
suite) est clairement établie \citep{ipcc_climate_2014}. Des projections
récentes des changements climatiques indiquent que les températures
moyennes mondiales pourraient augmenter de 2.6 à 4.8\(^{\circ}\)C d'ici
la fin du XXIe siècle dans le nord-est de l'Amérique du Nord, s'il n'y a
pas de progrès sur le contrôle des émissions de GES anthropiques
\citep{ipcc_climate_2014}. Le climat étant un déterminant important de
la distribution des espèces, de telles augmentations de température
auront un impact majeur sur la structure et les fonctions de tous les
écosystèmes \citep{bellard_impacts_2012, gauthier_vulnerability_2015}.

Selon les prévisions, les changements de température et de régimes de
précipitation devraient déplacer les niches climatiques optimales de
nombreuses espèces d'arbres vers le nord sur des centaines de kilomètres
\citep{mckenney_potential_2007} ou plus haut en altitude d'une centaine
de mètres \citep{jump_altitude-for-latitude_2009}, modifiant la
composition, la structure et la diversité forestières
\citep{price_anticipating_2013, reich_geographic_2015}. Or, de tels
changements dans les forêts peuvent avoir des répercussions
environnementales considérables sur les fonctions et les services des
écosystèmes, tels que l'approvisionnement en bois et en produits
forestiers non ligneux, le stockage du carbone, le cycle des nutriments,
la purification de l'air et de l'eau et le maintien d'habitats pour la
faune et la flore \citep{mitchell_linking_2013, mori_biodiversity_2017}.
Ces changements soulèvent aussi des enjeux socio-économiques majeurs.
Par exemple, comment adapter les stratégies de gestion forestière pour
assurer un approvisionnement durable en bois ? Ou encore, quel est
l'avenir de certaines espèces économiquement et culturellement
importantes, comme l'érable à sucre au Québec ? Comprendre et prédire
les conséquences de ces changements climatiques sur les écosystèmes
forestiers représente donc l'un des grands défis actuels pour la
communauté scientifique
\citep{pereira_scenarios_2010, garcia_multiple_2014}.

\hypertarget{ruxe9ponses-des-uxe9cosystuxe8mes-forestiers-aux-changements-climatiques}{%
\section{Réponses des écosystèmes forestiers aux changements
climatiques}\label{ruxe9ponses-des-uxe9cosystuxe8mes-forestiers-aux-changements-climatiques}}

\hypertarget{duxe9placements-des-aires-de-ruxe9partition}{%
\subsection{Déplacements des aires de
répartition}\label{duxe9placements-des-aires-de-ruxe9partition}}

Des changements de distribution liés au climat ont déjà été observés
pour de nombreuses espèces d'arbres à différentes échelles spatiales,
particulièrement dans les zones de transition où les changements sont
plus facilement détectables
\citep{jump_altitude-for-latitude_2009, boulanger_climate_2017}. Par
exemple, à l'échelle locale, \citet{fisichelli_temperate_2014} ont
observé une avancée de la régénération d'espèces arbres tempérées dans
la forêt boréale de la région à l'ouest des Grands Lacs et ce processus
semblait être facilité par des températures plus chaudes.
\citet{leithead_northward_2010} ont observé que les trouées causées par
la mort des arbres boréaux dans une forêt du nord de l'Ontario
facilitent l'établissement d'espèces tempérées du sud. Les changements
dans la composition forestière ont aussi été observés dans les écotones
altitudinaux sur une période de 40 ans; dans les Montagnes Vertes du
Vermont, les arbres tempérés ont progressé en altitude, conduisant à un
déplacement des limites de l'écotone boréal-tempéré d'environ 100 m
\citep{beckage_rapid_2008}, tandis que sur le Mont-Mégantic au sud du
Québec, les arbres se sont déplacés en élévation de près de 30 m en
moyenne et les espèces de sous-bois de près de 40 m
\citep{savage_elevational_2015}. Mis ensemble, ces derniers résultats
indiquent qu'un décalage dans la répartition des deux strates de
végétation, et donc un changement de composition, est déjà en train de
se former dû à une différence entre la vitesse de réponse des espèces de
sous-bois et celle des arbres. À l'échelle régionale,
\citet{boisvert-marsh_shifting_2014} et \citet{sittaro_tree_2017} ont
montré une migration à prédominance vers le nord des essences d'arbres à
travers le Québec, avec les gaulis présentant une réponse plus rapide
que les arbres adultes.

\hypertarget{ruxe9ponses-duxe9mographiques}{%
\subsection{Réponses
démographiques}\label{ruxe9ponses-duxe9mographiques}}

Alors que de nombreuses études sur l'impact des changements climatiques
sur les forêts ont tenté de détecter ou de prédire les déplacements des
limites d'aires de répartition des espèces, comparativement peu d'études
ont examiné les changements à long terme des taux démographiques,
e.g.~mortalité, recrutement, croissance, ou ont exploré les facteurs
environnementaux responsables de ces changements \citep[reviewed
in][]{allen_global_2010}. L'accent mis sur les déplacements des espèces
vers les pôles sous-estime l'empreinte des changements climatiques. Les
changements de température et de précipitations ont des effets directs
sur la croissance, la mortalité et le recrutement des arbres
\citep{vanderwel_climate-related_2013, zhang_half-century_2015}. Par
exemple, les augmentations récentes des taux de mortalité des arbres
dans l'ouest de l'Amérique du Nord ont été attribuées à des températures
élevées et des sécheresses
\citep{van_mantgem_widespread_2009, peng_drought-induced_2011}. Or,
c'est l'équilibre entre les gains par la croissance et le recrutement et
les pertes par la mortalité qui détermine, localement, la dynamique des
forêts et, régionalement, les limites d'aires de répartition
\citep{holt_theoretical_2005}. Des changements même très faibles dans
les taux démographiques peuvent modifier le rapport de force de la
compétition interspécifique
\citep{luo_observations_2013, reich_geographic_2015}, de même que la
dynamique et la trajectoire de succession des forêts
\citep{prach_four_2011}, modifiant par conséquent leur structure et leur
composition \citep{van_mantgem_widespread_2009, stephenson_causes_2011}.
À long terme, ces changements démographiques agissent donc pour
contrôler les limites géographiques des différents types de forêts
\citep{holt_theoretical_2005}.

Étant donné l'échelle temporelle à laquelle les changements de
répartition se produisent pour des organismes à longue durée de vie
comme les arbres, comprendre l'influence des conditions abiotiques et
biotiques sur les taux démographiques des populations offre une
meilleure perspective sur la biogéographie des espèces et permet
d'inférer les changements continus à la limite et à l'intérieur de
l'aire de répartition
\citep{sexton_evolution_2009, schurr_how_2012, thuiller_road_2013, snell_using_2014}.
Pourtant, à l'heure actuelle, il existe très peu d'informations
quantitatives sur l'effet combiné des changements climatiques et des
multiples perturbations forestières sur ces processus fondamentaux de la
dynamique forestière.

\hypertarget{duxe9lais-de-ruxe9ponse-et-duxe9suxe9quilibre}{%
\section{Délais de réponse et
déséquilibre}\label{duxe9lais-de-ruxe9ponse-et-duxe9suxe9quilibre}}

Bien qu'on prévoie un déplacement des niches climatiques des arbres de
plusieurs centaines de kilomètres vers le nord d'ici la fin du siècle
\citep{mckenney_potential_2007}, un nombre croissant d'études suggèrent
que le déplacement des arbres en Amérique du Nord ne réussira
probablement pas à suivre le rythme du réchauffement climatique
\citep{zhu_failure_2012, woodall_assessing_2013, vissault_biogeographie_2016, sittaro_tree_2017}.
Par exemple, malgré qu'on observe un déplacement des aires de
répartition des arbres vers le nord, les vitesses de migration des
espèces d'arbres au Québec étaient en moyenne inférieures à 50\% de la
vitesse d'avancée géographique des changements climatiques récents
\citep{sittaro_tree_2017}.

Les arbres sont particulièrement susceptibles de montrer de longs délais
de réponse aux changements parce que ces espèces sont sessiles, ont une
faible capacité de dispersion, une longue durée de vie, une croissance
lente et une maturité sexuelle tardive
\citep{iverson_tree-species_2013, lenoir_climate-related_2015}. Ces
caractéristiques pourraient expliquer le haut niveau d'inertie des
forêts malgré les changements de climat
\citep{vissault_biogeographie_2016}. En effet, ces caractéristiques
peuvent expliquer l'absence de colonisation à la limite nord malgré que
les conditions soient devenues favorables et engendrer un crédit de
colonisation. Inversement, les espèces peuvent persister pendant un
certain temps dans un milieu nouvellement inadapté en raison du délai
d'extinction ou peuvent être maintenues grâce à une dynamique
source-puit, engendrant une dette d'extinction
\citep{pulliam_relationship_2000, jackson_balancing_2010, schurr_how_2012}.
Ainsi, étant donné que l'environnement est dynamique et que les
écosystèmes forestiers sont caractérisés par d'importants délais de
colonisation et d'extinction, les systèmes perturbés ne parviennent
souvent pas à un équilibre statistique sur des échelles temporelles et
spatiales réalistes pour permettre une analyse statique (e.g., SDM). Il
y a donc un décalage entre la niche Hutchinsonienne et la répartition
géographique d'une espèce \citep{holt_bringing_2009}. Pourtant, la
majorité des modèles de distribution d'espèces suppose que les espèces
sont en équilibre avec leur environnement, ignorant la dynamique de
transition.

D'importants délais de réponse des forêts aux changements climatiques
sont déjà observables puisque la distribution de plusieurs espèces
d'arbres de l'est de l'Amérique du Nord n'est pas à l'équilibre avec le
climat aux marges de leur aire de répartition, avec davantage de dettes
d'extinction au sud et de crédits de colonisation au nord
\citep{talluto_extinction_2017}. Leurs résultats montrent aussi que la
vitesse de la contraction d'aire de répartition dans le sud est plus
rapide que l'expansion dans le nord \citep{talluto_extinction_2017}).
Des simulations ont aussi montré que le décalage entre la niche
climatique optimale des espèces tempérées et leur distribution réalisée
ne fera que s'accroître avec le temps
\citep{vissault_biogeographie_2016}. Cette tension grandissante entre la
distribution réalisée et potentielle des espèces risque d'autant plus de
causer des changements brusques (regime shift) dans les écosystèmes
forestiers suite à une perturbation anthropique ou naturelle
\citep{vanderwel_how_2014, renwick_temporal_2015}.

\hypertarget{contraintes-uxe0-la-migration}{%
\section{Contraintes à la
migration}\label{contraintes-uxe0-la-migration}}

\hypertarget{interactions-biotiques}{%
\subsection{Interactions biotiques}\label{interactions-biotiques}}

Alors que le climat est un déterminant majeur de la niche des espèces,
des facteurs non climatiques, tels que les interactions biotiques,
imposent des contraintes supplémentaires à la migration des espèces.
Bien qu'elles répondent de manière indépendante, les espèces ne sont pas
isolées, mais interagissent avec les membres de leur communauté. Il est
généralement admis que les facteurs déterminant la distribution sont
spatialement hiérarchisés, de sorte que le climat régirait la
répartition à l'échelle régionale, alors que les interactions biotiques
seraient plus importantes à l'échelle locale
\citep{soberon_grinnellian_2007}. De plus, le climat contraindrait la
distribution et l'abondance des espèces à leur limite nord, tandis que
le rôle des interactions serait plus important à la limite sud et à
l'intérieur de l'aire de répartition, là où les conditions
environnementales sont plus favorables \citep{louthan_where_2015}. Un
bon exemple de ce phénomène est la distribution de l'épinette noire, une
espèce ayant une niche écologique très large, mais dont la distribution
au sud est limitée aux sites où la compétition est faible
\textbf{(Loehle 1998)}, comme des sites à drainage très mauvais ou
excessif. Toutefois, avec les changements climatiques, les conditions
favorables se déplacent et forcent de nouvelles interactions à la marge
des aires de distributions \citep{kissling_multispecies_2015}. Par
exemple, à moins qu'un dépérissement massif de la forêt ne se produise,
les espèces tempérées qui migreront dans les forêts boréales devront
s'établir sur des sites qui sont déjà colonisés par d'autres espèces et
devront donc vraisemblablement compétitionner pour les ressources lors
de leur établissement \citep[phénomène appelé l'effet
prioritaire;][]{gilman_framework_2010}. Plus la compétition par les
espèces résidentes sera forte, plus la probabilité de colonisation par
les espèces migratrices diminuera, car ces dernières parviendront
difficilement à s'installer \citep{cazelles_integration_2016}, d'où
l'importance potentielle des interactions dans la distribution à grande
échelle. Une étude de simulation a d'ailleurs révélé que les taux de
migration sont plus faibles dans les forêts établies que dans les forêts
de début de succession, et lorsque la diversité est grande
\citep{meier_climate_2012}. Ainsi, les espèces de début de succession
ont des taux de migration plus rapides que les espèces de fin de
succession puisque ces dernières colonisent principalement les habitats
forestiers déjà colonisés où la compétition interspécifique est plus
élevée \citet{meier_climate_2012}{]}. Les interactions biotiques sont de
plus en plus reconnues comme étant un facteur clé influençant la
distribution des espèces à grande échelle
\citep{meier_biotic_2010, blois_climate_2013, wisz_role_2013, svenning_influence_2014, cazelles_theory_2016}.

Comme la répartition géographique d'une espèce dépend de nombreux
facteurs environnementaux, ainsi que des limites de dispersion et des
contingences historiques
\citep{pulliam_relationship_2000, holt_bringing_2009, godsoe_i_2010}, il
peut s'avérer difficile sur le plan technique de trouver des preuves de
l'effet des interactions entre espèces sur la distribution. Les limites
physiologiques des espèces (et donc l'hétérogénéité environnementale)
influencent les interactions biotiques puisqu'elles déterminent le pool
d'espèces qui peuvent potentiellement cohabiter à un endroit donné. Les
patrons de cooccurrence des espèces en compétition sont en partie dus au
hasard, déterminés par qui est arrivé le premier et par des facteurs
aléatoires qui donnent un avantage initial. L'influence de l'effet
prioritaire et de l'hétérogénéité environnementale sur la répartition
actuelle des espèces rend donc difficile l'estimation de la force de
compétition à partir des patrons de cooccurrence ; une faible
cooccurrence peut refléter une faible compétition actuelle, mais une
forte compétition dans le passé, et inversement une forte cooccurrence
peut indiquer une faible compétition puisque les espèces coexistent,
mais aussi une forte compétition pour les mêmes ressources.

Bien que plusieurs études sur les déplacements d'aires de répartition
soulignent que les interactions biotiques risquent de réduire le succès
de migration, les preuves empiriques de leurs impacts sur les taux de
migration sont rares et indirectes \citep{svenning_influence_2014}. Les
interactions interspécifiques représentent donc un facteur inconnu clé
dans les études sur le changement climatique. Une étude approfondie des
taux de recrutements et de mortalités pourrait permettre de tester et
quantifier l'importance du rôle joué par les interactions biotiques sur
la dynamique de transition et de migration des arbres.

\hypertarget{propriuxe9tuxe9s-du-sol}{%
\subsection{Propriétés du sol}\label{propriuxe9tuxe9s-du-sol}}

En plus de la compétition interspécifique, les espèces migratrices
coloniseront des sols qui sont déjà développés et qui présentent des
propriétés (e.g.~qualité du drainage, disponibilité en nutriments, pH,
mycorhizes) qui varient localement ou régionalement, lesquelles
pourraient retarder ou contraindre leur établissement
\citep{goldblum_deciduous_2010, lafleur_response_2010, brown_non-climatic_2014}.
Les forêts dominées par les conifères au nord où la température moyenne
est froide présentent généralement des sols plus acides et conduisent à
une activité microbienne plus faible et à une décomposition plus lente
de la matière organique que les forêts tempérées décidues plus chaudes
du sud \citep{goldblum_deciduous_2010}. Par exemple,
\citet{collin_conifer_2017} ont montré que l'acidité du sol forestier
sous une canopée dominée par les conifères affecte négativement les
semis de l'érable à sucre via un débalancement nutritif foliaire, ce qui
pourrait donc freiner sa migration dans la forêt boréale. Toutefois, les
espèces forestières, par leur effet sur la qualité chimique de la
litière (C, N, Mg) et sur la composition des microorganismes du sol,
peuvent elles-mêmes modifier les taux de décomposition de la matière
organique et la disponibilité des éléments nutritifs
\citep{laganiere_how_2010}. Les espèces migratrices pourraient donc
influencer leur propre taux d'invasion. Plusieurs études menées dans le
nord-est du Canada ont montré une colonisation rapide des peuplements
résineux par le peuplier faux-tremble après une perturbation par la
coupe forestière ou les feux
\citep{chen_wildfire_2009, laquerre_augmentation_2009}. La présence de
cette espèce décidue induit des changements physicochimiques et accélère
les taux de décomposition de la matière organique
\citep{legare_influence_2005, laganiere_how_2010}. À leur tour, ces
conditions de sol modifiées pourraient favoriser l'établissement et la
persistance de nouvelles espèces migratrices.

En plus des facteurs endogènes (traits, démographie lente et dispersion
limitée), la compétition par les espèces résidentes et les contraintes
imposées par les propriétés des sols résidents sur les plantes
migratrices sont des facteurs exogènes qui peuvent également contribuer
aux déséquilibres observés entre la niche climatique et la répartition
des espèces, particulièrement par un crédit de colonisation. La
compréhension de l'effet de la compétition et des sols sur les plantes
migratrices est donc essentielle pour prédire la redistribution des
espèces sous le changement climatique.

\hypertarget{interaction-entre-changements-climatiques-et-perturbations}{%
\subsection{Interaction entre changements climatiques et
perturbations}\label{interaction-entre-changements-climatiques-et-perturbations}}

Malgré l'empreinte indéniable des changements climatiques, la réponse
récente des écosystèmes forestiers n'est pas aussi unidirectionnelle que
prévu, car elle dépend de nombreux facteurs qui peuvent interagir entre
eux; ainsi les répercussions à long terme demeurent encore difficiles à
prévoir. Ajoutés aux effets des changements climatiques sur la
performance des arbres, sont les effets des perturbations naturelles à
grande échelle, notamment les feux de forêt et les épidémies d'insectes
\citep{keane_exploring_2013, gauthier_vulnerability_2015, bergeron_projections_2017, boulanger_climate_2017},
qui peuvent déclencher des altérations rapides dans la succession
végétale et par conséquent dans les fonctions des écosystèmes. De la
même façon, les activités forestières peuvent également interagir
fortement avec les impacts liés aux changements climatiques en modifiant
la structure et la composition des forêts
\citep{scheller_spatially_2005, bergeron_projections_2017, boulanger_climate_2017}.
Par exemple, entre 1930 et 2002, la coupe forestière dans une région à
la limite nord des espèces tempérées au Québec a engendré un changement
majeur de composition; près de 40 \% du paysage est passé d'un couvert
coniférien à un couvert mixte et près de 20 \% est devenu feuillu
(Boucher et al. 2006).

Les perturbations, autant naturelles qu'anthropiques, devraient avoir
une forte influence sur la façon dont les forêts répondent aux
changements climatiques car elles peuvent offrir des opportunités de
colonisation, changer le rapport de force de compétition entre les
espèces d'arbres pionnières et de fin de succession et faciliter
l'expansion des espèces tempérées vers le nord capables de profiter des
ouvertures de la canopée
\citep{xu_importance_2012, woodall_assessing_2013, vanderwel_how_2014}.
Une perturbation combinée aux changements dans les conditions
climatiques peut changer la trajectoire successionnelle de la forêt et
même la faire basculer vers un autre nouvel état altéré persistant
(concept de regime shift), par exemple d'une forêt à dominance
conifèrienne à une forêt mixte, tel qu'observé par
\citet{boucher_logging-induced_2006}. Les perturbations telles que les
feux ou les coupes pourraient alors agir comme des accélérateurs
possibles de la migration future de la forêt.

Face aux nombreuses perturbations et étant donné la longue échelle
temporelle des processus de dynamique forestière, il semble
incontournable que les forêts soient de plus en plus en déséquilibre.
Par conséquent, la réponse des forêts aux futurs changements climatiques
dépendra et interagira avec des dynamiques de transition déjà en cours.
Démêler les effets des changements climatiques et ceux des perturbations
naturelles et anthropiques et leurs rétroactions potentielles est
nécessaire à la fois pour informer les modèles prédictifs de
distribution de la biodiversité sous les changements climatiques et pour
élaborer des stratégies de gestion forestière permettant un aménagement
durable des forêts.

\hypertarget{enjeux-et-importances}{%
\section{Enjeux et importances}\label{enjeux-et-importances}}

La question des effets des CC sur la dynamique forestière soulève de
nombreux enjeux. La gestion de la biodiversité est un enjeu
majeur\ldots{} La gestion actuelle repose grandement sur une conception
statique de la biodiversité dans un climat stable. Par exemple,
l'aménagement écosystémique des forêts se base sur des états de
références historiques (Egan \& Howell, 2001), comme les forêts en place
avant la colonisation européenne et l'exploitation industrielle : les
forêts précoloniales ou préindustrielles. ``L'utilisation d'états de
références historiques pour l'aménagement écosystémique comporte des
limites. Dans le contexte de changements climatiques, une utilisation «
stricte » des caractéristiques d'écosystèmes du passé comme états de
référence pourrait aboutir à des écosystèmes forestiers non viables dans
le futur (Choi et al., 2008).'' De plus, l'aménagement forestier repose
également sur des modèle des possibilités forestières, ``lesquelles
correspondent au volume maximum des récoltes annuelles que l'on peut
prélever à perpétuité, sans diminuer la capacité productive du milieu
forestier.'' La calcul de possibilité forestière tient compte de
plusieurs critères tels que la dynamique naturelle des forêts, leur
composition, leur structure d'âge les aires de protection et la
probabilité de perturbation par les feux, les insectes et les maladies.
Cependant, ce calcul fait des prédictions à long terme et la coupe
forestière dépend de ces prédictions. Or, on devine que le changement
rapide du climat risque de bousculer ces prédictions. Par exemple, une
espèce pourrait ne pas se renouveler après coupe. Notre capacité à
prédire les effets futurs des changements climatiques sur la dynamique
forestière dépend de la description et de la compréhension de ses effets
passés et de son interaction avec les perturbations naturelles.

\hypertarget{contexte-climatique-et-uxe9cologique-du-quuxe9bec}{%
\section{Contexte climatique et écologique du
Québec}\label{contexte-climatique-et-uxe9cologique-du-quuxe9bec}}

\hypertarget{climat-du-quuxe9bec}{%
\subsection{Climat du Québec}\label{climat-du-quuxe9bec}}

Le climat du Québec est fortement marqué par le gradient latitudinal de
la température. Ce gradient de chaleur est le facteur le plus
déterminant pour la composition de la végétation du Québec. Ainsi on
aura, du sud vers le nord, un gradient de biodiversité qui reflète
étroitement celui de la température moyenne.

\hypertarget{vuxe9guxe9tation-et-domaines-bioclimatiques}{%
\subsection{Végétation et domaines
bioclimatiques}\label{vuxe9guxe9tation-et-domaines-bioclimatiques}}

Sur une superficie totale de 1 667 712 km\^{}2, ses forêts couvrent 761
100 km\^{}2, soit près de la moitié du territoire. La nordicité de la
forêt québécoise a comme conséquence la dominance des forêts résineuses
sur une grande partie du territoire et la faible diversité en espèces
d'arbres. En raison du fort gradient de température, les types de forêt
sont également structurés latitudinalement.

La forêt boréale occupe environ 72 \% du territoire québécois et sa
dynamique repose sur les feux, les épidémies de tordeuse des bourgeons
de l'épinette, les trouées et les chablis. Elle est composée
majoritairement d'épinette noire et de sapin baumier, mais aussi de pin
gris, de bouleau blanc et de peuplier faux-tremble

La forêt boréale 551 400 km\^{}2, la forêt mélangée 98 600 km2 et la
forêt feuillue 111 100 km2

PRINCIPALES ESSENCES D'ARBRES Forêt boréale : épinette noire, sapin
baumier et bouleau blanc. Forêt mélangée : bouleau jaune et sapin
baumier. Forêt feuillue : érable à sucre et bouleau jaune.

\hypertarget{changements-climatiques-au-quuxe9bec}{%
\subsection{Changements climatiques au
Québec}\label{changements-climatiques-au-quuxe9bec}}

\hypertarget{uxe9tat-des-foruxeats-du-quuxe9bec-muxe9ridional}{%
\subsection{État des forêts du Québec
méridional\ldots{}}\label{uxe9tat-des-foruxeats-du-quuxe9bec-muxe9ridional}}

\hypertarget{objectifs}{%
\section{Objectifs}\label{objectifs}}

Le principal objectif de cette thèse est de comprendre l'influence des
changements climatiques et des perturbations sur les changements à long
terme dans les écosystèmes forestiers tempérés. En utilisant les données
d'inventaires forestiers du Québec méridional de 1970 à 2018, cette
thèse s'articule autour de trois grandes questions :

\begin{enumerate}
\def\labelenumi{(\arabic{enumi})}
\item
  Comment les changements dans les patrons spatio-temporels de mortalité
  et de recrutement des arbres ont-ils influencé la diversité et la
  composition des forêts boréales et tempérées au cours des dernières
  décennies ?
\item
  Et quelle est l'importance relative des facteurs liés au climat, aux
  perturbations humaines (coupe, pollution) et naturelles (épidémie,
  feu) et aux caractéristiques du peuplement qui influencent la
  mortalité des arbres ?
\item
  Est-ce que les interactions compétitives entre les espèces d'arbres
  influencent leur taux de recrutement et de mortalité ? De ces
  questions en découle une autre très intéressante, à savoir quelle est
  l'influence des changements climatiques récents combinée aux effets
  des perturbations sur la trajectoire des communautés forestières.
\end{enumerate}

Chacun de ces objectifs est traité dans un chapitre de cette thèse
(chapitres 2, 3 et 4). Les réponses à ces questions sont essentielles
pour comprendre les relations entre les mécanismes locaux (interactions
entre espèces) et régionaux (contraintes environnementales) qui
sous-tendent les réponses des communautés aux changements
environnementaux. L'étude des variations spatio-temporelles de la
distribution des espèces apportera donc de nouvelles informations très
utiles sur l'importance relative de ces divers mécanismes. Cette étude
permettra ainsi de mettre en évidence le lien entre la dynamique
forestière et les changements climatiques, en tenant compte des
perturbations forestières, des interactions compétitives et des diverses
contraintes à la migration.

\hypertarget{sections-suivantes-uxe0-intuxe9grer-plus-haut}{%
\section{sections suivantes à intégrer plus
haut}\label{sections-suivantes-uxe0-intuxe9grer-plus-haut}}

\hypertarget{chapitre-1-patrons-de-diversituxe9-buxeata-temporelle}{%
\subsection{Chapitre 1 : Patrons de diversité bêta
temporelle}\label{chapitre-1-patrons-de-diversituxe9-buxeata-temporelle}}

L'Homme est aujourd'hui la principale force gouvernant les processus
écologiques faisant entrer la terre dans une nouvelle ère géologique,
l'Anthropocène \citep{crutzen_geology_2002}. Un nombre croissant de
preuves révèle une perte de biodiversité exceptionnellement rapide au
cours des derniers siècles, ce qui indique qu'une sixième extinction de
masse est déjà en cours \citep{ceballos_accelerated_2015}. D'ailleurs,
\citet{ceballos_accelerated_2015} soulignent qu'au-delà des extinctions
globales des espèces, la Terre connaît aussi un énorme épisode de déclin
des populations, dont les conséquences se répercuteront sur les
fonctions et les services des écosystèmes. Malgré tout, des métaanalyses
récentes ont montré que bien souvent, à l'échelle locale, la
biodiversité ne diminue pas et peut même parfois augmenter
\citep{vellend_global_2013, dornelas_assemblage_2014}. Bien que ces
résultats aient été vivement critiqués (Newbold et al.~2015; Gonzalez et
al.~2016), il reste clair que la diversité locale (diversité \(\alpha\))
peut montrer des tendances variées, déconnectées des tendances à plus
grande échelle, même face à une extinction de masse à l'échelle globale.
Dans tous les cas, il est généralement admis qu'il y a eu des
changements importants dans la composition des communautés (diversité
\(\beta\); Vellend et al.~2013; Dornelas et al.~2014; Newbold et
al.~2015), impliquant à la fois des pertes et des gains d'espèces
(Wardle et al.~2011). Ainsi, afin de mieux comprendre l'effet des
changements anthropiques sur la biodiversité, nous devons examiner
parallèlement la diversité \(\alpha\) et \(\beta\), ainsi que les
composantes sous-jacentes de ces changements, les pertes et les gains
d'espèces.

Des travaux récents ont attiré l'attention sur le gain en compréhension
lorsque la diversité \(\beta\) est partitionnée en ses composantes
sous-jacentes
\citep{baselga_partitioning_2010, podani_general_2013, legendre_interpreting_2014, legendre_temporal_2019}.
De telles analyses permettent de quantifier les contributions de
différents processus écologiques à la diversité \(\beta\).
\citet{legendre_thirty-year_2015} ont développé une méthode pour
partitionner la diversité \(\beta\) temporelle en composantes de pertes
et de gains en espèces, et l'ont appliquée aux communautés de mollusque
se rétablissant après des essais nucléaires. Cette méthode offre la
possibilité de faire le lien entre les changements de diversité et les
changements démographiques dans les communautés, puisque les pertes et
les gains sont en fait des mortalités et des recrutements lorsque
calculés sur des données d'abondance, et des extinctions et
colonisations lorsque calculés sur des données de présence-absence.

Le chapitre 1 de la thèse vise à répondre à deux objectifs principaux
liés à la fois aux tendances temporelles de la biodiversité dans les
forêts de l'écotone boréal-tempéré au cours des dernières décennies et à
l'application de nouvelles méthodes d'analyse de diversité \(\beta\)
temporelle. Spécifiquement : quelles sont les tendances temporelles de
diversité \(\alpha\) et \(\beta\) des forêts ? Comment les forêts
ont-elles changé en termes de mortalités et de recrutements ? Est-ce que
ces changements sont constants pour différents groupes d'espèces et de
régions ? Selon mes hypothèses, il n'y aura pas de tendance temporelle
particulière au niveau de la diversité \(\alpha\). Inversement, il y
aura une augmentation de la diversité \(\beta\) au cours du temps qui
sera provoquée principalement par une augmentation des mortalités,
attribuables à l'action concommitante de multiples perturbations, qui ne
sera pas compensée par des recrutements.

En utilisant les données d'inventaires forestiers du Québec méridional
\citep{mffp_placettes-echantillons_2016}, ces questions seront étudiées,
dans un premier temps, en quantifiant les tendances temporelles de
diversité \(\alpha\), mesurée comme un changement dans la richesse
locale, et de diversité \(\beta\), mesurée comme un changement dans la
composition des communautés. Et dans un deuxième temps, en analysant les
composantes sous-jacentes d'un indice de diversité \(\beta\) temporelle
(TBI\,; Temporal Beta Diversity Index\,; Legendre \& Salvat 2015), soit
les mortalités et les recrutements. En accordant une attention accrue
aux tendances de la diversité \(\beta\), ce travail pourra révéler des
tendances précédemment imperceptibles sous l'angle de la diversité
\(\alpha\) seule et aidera à améliorer notre compréhension des réponses
de la biodiversité forestière aux multiples facteurs de stress
anthropiques qui se sont accélérés au cours des dernières décennies.

\hypertarget{chapitre-2-tendances-et-causes-de-mortalituxe9s-dans-les-foruxeats-tempuxe9ruxe9es}{%
\subsection{Chapitre 2 : Tendances et causes de mortalités dans les
forêts
tempérées}\label{chapitre-2-tendances-et-causes-de-mortalituxe9s-dans-les-foruxeats-tempuxe9ruxe9es}}

La mortalité et le recrutement des arbres sont les moteurs principaux de
la dynamique forestière à long terme et leur variation peut entraîner
des changements marqués dans la composition et la structure des
communautés. Cependant, nous avons actuellement peu d'informations
quantitatives sur la variation géographique de ces taux et l'importance
relative des causes de la mortalité des arbres.

Plusieurs études récentes ont montré une augmentation des taux de
mortalité des arbres dans le temps associée à l'augmentation des
températures et des sécheresses
\citep{van_mantgem_apparent_2007, van_mantgem_widespread_2009, allen_global_2010, peng_drought-induced_2011}.
Malgré l'importance indéniable du climat à l'échelle régionale sur ces
tendances, étonnamment peu d'attention a été accordée aux autres causes
possibles de mortalités qui peuvent interagir avec le climat. Par
exemple, Dietze et al.~(2011) ont révélé que les polluants
atmosphériques (particulièrement les dépôts acides) avaient un effet
particulièrement élevé, plus grand que l'effet du climat, sur les taux
de mortalité des arbres des forêts de l'est de l'Amérique du Nord. De
même, les processus endogènes associés au développement des peuplements
forestiers, tels que le stade de succession et la compétition, peuvent
avoir une grande influence sur la dynamique, mais ont été largement
ignorés puisque de nombreuses études sur l'effet des changements
climatiques sur la mortalité excluent de facto les forêts qui ont été
perturbées. Des études dans l'Ouest Canadien ont ainsi montré que
l'effet des changements climatiques sur les tendances temporelles de
mortalité des arbres était nettement plus important dans les jeunes
forêts que dans les forêts matures (Luo \& Chen 2013; Zhang et
al.~2015). La contribution relative de ces facteurs pourrait aussi
varier entre la forêt boréale et la forêt tempérée puisque leur
dynamique naturelle est très différente; la dynamique des forêts
boréales est gouvernée par des perturbations à grandes échelles, comme
les feux, les épidémies et la coupe, tandis que la dynamique des forêts
tempérées est plutôt dominée par des perturbations très locales de type
trouée \citep{goldblum_deciduous_2010}. La quantification des
contributions relatives de différentes causes de mortalité des arbres
est cruciale non seulement pour mieux comprendre et anticiper les
changements dans la dynamique forestière, mais aussi pour mieux informer
les modèles qui se basent sur la démographie.

Les taux typiques de mortalité des arbres sont faibles (de l'ordre de
0.1 à 2\% par l'année) de sorte qu'estimer leur variation de manière
fiable requiert de grands échantillons et de longues périodes
d'échantillonnage. Les inventaires forestiers représentent une occasion
unique d'étudier les changements démographiques à long terme. Dans le
chapitre 2, les tendances et les causes des changements démographiques
(mortalités et recrutements) des populations d'arbres au cours des
quatre dernières décennies seront analysées. Je m'intéresse à trois
questions : 1. Est-ce que les taux de mortalité ou de recrutement ont
changé systématiquement dans les forêts du Québec méridional au cours
des quatre dernières décennies ? 2. Si les taux démographiques ont
changé, les changements sont-ils constants entre les différents groupes
d'espèces (boréales, tempérées et pionnières) et entre les régions ? Et
enfin, 3. Quelles sont les causes probables de ces changements
démographiques et quelle est l'importance relative des effets endogènes
(développement du peuplement, succession) et des effets exogènes
(climat, perturbations naturelles et humaines) sur la mortalité
individuelle des arbres? Selon mes hypothèses, avec les changements
climatiques, on observera une augmentation du recrutement des espèces
tempérées particulièrement en zone de forêts mixtes, et, inversement,
une augmentation de la mortalité des espèces boréales. Toutefois, comme
la majorité des mortalités seront principalement causées par des
perturbations directes, comme la coupe et les feux, il y aura aussi une
augmentation des espèces pionnières.

La probabilité de mortalité annuelle sera analysée par un modèle de
régression logistique avec divers prédicteurs environnementaux, incluant
des variables liées aux caractéristiques du peuplement (âge, surface
terrière, densité), des variables climatiques (température annuelle
moyenne, précipitation annuelle totale) et des variables liées aux
perturbations humaines (e.g., coupes forestières) et naturelles
(épidémies d'insecte, feux).

\hypertarget{chapitre-3-influence-des-interactions-compuxe9titives-sur-la-dynamique-de-migration-des-arbres-en-foruxeats-tempuxe9ruxe9es}{%
\subsection{Chapitre 3 : Influence des interactions compétitives sur la
dynamique de migration des arbres en forêts
tempérées}\label{chapitre-3-influence-des-interactions-compuxe9titives-sur-la-dynamique-de-migration-des-arbres-en-foruxeats-tempuxe9ruxe9es}}

Bien qu'un important déplacement des niches climatiques soit anticipé
d'ici la fin du siècle, les approches de modélisation utilisées à ce
jour sont majoritairement incapables de projeter le rythme auquel les
espèces forestières répondront aux changements climatiques, car les
contraintes à la migration sont encore peu connues. Une des hypothèses
importantes avance que la compétition par les espèces résidentes
pourrait freiner l'établissement des espèces migratrices (Svenning et
al.~2014). Cependant, l'effet des interactions biotiques sur la
dynamique d'expansion d'aires de répartition des arbres n'a pas reçu
suffisamment d'attention et jusqu'à maintenant les études empiriques sur
le sujet sont principalement issues d'expériences de transplantation
(Hillerislambers et al.~2013; Brown \& Vellend 2014).

Les perturbations pourraient moduler la vitesse de réponse des forêts
aux changements climatiques en diminuant ou éliminant la compétition par
les espèces résidentes, créant ainsi des opportunités de colonisation
pour les espèces d'arbres tempérés (Xu et al.~2012; Woodall et al.~2013;
Vanderwel \& Purves 2014). Les perturbations pourraient donc accélérer
les changements de composition dans les forêts ou même les faire
basculer d'une dominance conifèrienne à une composition mixte (regime
shift). En plus de pouvoir accélérer la vitesse de transition, les
perturbations pourraient influencer différentiellement les espèces en
raison du compromis compétition-colonisation. Par exemple, les
compétiteurs inférieurs pourraient être favorisés (du moins à court
terme) grâce à leur meilleure capacité de colonisation leur permettant
d'atteindre les milieux récemment perturbés (Gilman et al.~2010). Pour
l'instant, l'effet des perturbations sur les capacités de colonisation
des espèces migratrices en réponse aux changements climatiques a été
étudié surtout par modélisation à l'échelle régionale (Scheller \&
Mladenoff 2005; Vanderwel \& Purves 2014) et empiriquement à l'échelle
locale (Leithead et al.~2010). À l'échelle régionale, Woodall et
al.~(2013) n'ont pas trouvé de différence dans les limites nord de
répartition des semis et des arbres adultes entre les sites perturbés et
non perturbés aux États-Unis, mais ils n'avaient que 5 ans d'intervalle.

S'il y a d'autres contraintes abiotiques que le climat aux limites nord
des aires de répartition des espèces, les projections de migration sous
les changements climatiques qui ignorent ces facteurs pourraient
surestimer l'effet des températures sur l'expansion des aires. Certaines
études expérimentales suggèrent, par exemple, que les propriétés du sol
peuvent freiner la germination et la croissance des semis (Brown \&
Vellend 2014; Eskelinen \& Harrison 2015; Collin et al.~2017). Ainsi,
même si les contraintes climatiques sont relâchées pour une espèce
donnée et que la compétition est éliminée par une perturbation, il est
possible que l'habitat ne soit tout de même pas adéquat (Beauregard \&
De Blois 2014). Ces contraintes biotiques liées à la compétition et
abiotiques liées aux propriétés du sol pourraient donc freiner ou
empêcher la migration des espèces tempérées et contribuer au
déséquilibre entre la répartition géographique et la niche climatique
potentielle des espèces.

Au chapitre 3, je vais évaluer l'importance relative des facteurs non
climatiques sur la dynamique de colonisation et d'extinction des arbres
dans les forêts tempérées, particulièrement à la limite de leur aire de
répartition. Je tenterai de répondre aux questions suivantes : Est-ce
que la compétition (mesurée par un effet densité-dépendant des espèces
résidentes sur l'espèce migratrice) influence les probabilités de
colonisation des arbres ? Ou est-ce plutôt l'effet des propriétés du sol
qui freine la probabilité de colonisation ? Si la compétition est
importante, est-ce que les perturbations permettent d'accélérer le taux
de recrutement des espèces tempérées à la limite nord de leur
distribution ? Selon mes hypothèses, l'effet combiné de la compétition
et des propriétés du sol sur les espèces tempérées migratrices sera
soutenu en marge de leur aire de répartition, mais les perturbations
faciliteront l'établissement de ces espèces en diminuant la compétition.
Aussi, cette diminution de la compétition avantagera différentiellement
les espèces selon leurs traits, et le gain sera plus grand pour les
espèces de début de succession.

Dans ce chapitre, un modèle basé sur la théorie des métapopulations et
des métacommunautés me permettra de modéliser la dynamique d'assemblage
des communautés locales en prenant en considération la manière dont la
compétition, le sol et les perturbations peuvent faciliter ou entraver
la migration des arbres via leurs effets sur la démographie. Le modèle
classique sera étendu de façon à ce que la probabilité de
colonisation/recrutement et d'extinction/mortalité d'une espèce soit
conditionnelle aux variables climatiques (Talluto et al.~2017),
édaphiques et à la composition de la communauté.

\bibliographystyle{apalike-uqam}
\bibliography{references.bib}
