\francais

\chapter*{Remerciements}

Une thèse de doctorat c'est loin d'être un long fleuve tranquille. C'est
plutôt une longue expédition sur une rivière en eau vive ou sur un
sentier de randonnée en montagne. Mon parcours de doctorat a commencé
dans un épais brouillard mais, grâce à vous, j'ai fini par retrouver mon
chemin et ma passion, la forêt. La suite de mon parcours a aussi été
parsemée de nombreux obstacles, de remises en question, d'angoisses et
de doutes, mais aussi et surtout de belles découvertes, beaucoup
d'apprentissage et des rencontres marquantes qui m'ont gardée motivée
tout au long de ces quatre années.

Bien qu'empreint de beaucoup de solitude, ce doctorat n'est pas une
réalisation individuelle. Par leurs idées, leurs conseils ou leur
support moral, plusieurs personnes ont contribué de près ou de loin à
l'achèvement de cette thèse.

Je souhaite tout d'abord remercier mon directeur de recherche, Pierre
Legendre, et ma co-directrice, Marie-Josée Fortin, pour m'avoir soutenue
et encouragée tout au long de ce projet. Vous avez su me donner une
grande liberté dans la réalisation de mon projet tout en me guidant
lorsque j'en avais besoin. Pierre, ton éternel et intarissable passion
pour la recherche et l'écologie numérique m'impressionne et me motive.
Tes conseils et tes encouragements m'ont permis d'aller plus loin.
Marie-Josée, sans toi, cette thèse de doctorat n'aurait peut-être pas vu
le jour. Tu as su m'encourager et me remettre sur le droit chemin quand
j'étais perdue. Merci à tous les deux, vous m'avez amenée exactement là
où je dois être.

Un grand merci à Pierre Grondin pour avoir fait de mon stage à la
Direction de la Recherche forestière une expérience enrichissante et
riche en rencontres. Les nombreux projets et les discussions passionnées
sur la forêt ne font certainement que commencer. Un merci également à
Dominique Gravel pour m'avoir accueillie dans ton laboratoire à
l'Université de Sherbrooke et m'avoir offert de précieux conseils. Merci
à Stéphanie Pellerin et Monique Poulin, mes directrices de maîtrise.
Vous m'avez non seulement formée à la recherche mais vous m'avez aussi
donné le goût de continuer dans cette voie.

À mes fidèles amies, Corinne, Mélanie, Liliane, Amélie et, même si vous
êtes loin, Aurélie et Caroline, un grand merci. Merci pour toutes ces
belles soirées entre filles qui m'ont permis de me changer les idées et
de penser à autres choses qu'au travail. Malgré que nos chemins aient
divergé après le secondaire, notre amitié, toujours aussi forte, nous
garde unies. J'aurais aimé célébrer ce moment avec vous, mais ça a l'air
qu'il y a un petit virus dans les parages, alors il faudra être
patientes\ldots{} Corinne, merci pour cette grande aventure de rafting
au Grand Canyon! Trois semaines en plein air sans ordinateur, sans
courriel, tu ne peux pas savoir comment c'était merveilleux.

Merci à Steve, David et Catherine (et Liam et Alissa) pour ces
inoubliables semaines travaille-vacance au chalet. On a pu s'amuser,
discuter de science, se plaindre de la science, s'entraîner, cuisiner,
faire du pédalo, etc. Tout ça en travaillant sur nos projets respectifs.
Ces moments ont rendu la thèse tellement plus agréable. Annabelle et
Philippe, mes loyaux partenaires dans tous les cours de bio. Ensemble,
on a fait grandir notre passion pour la biodiversité et on a réussi
passer à travers le cycle de Krebs.

Grand-maman Flora, si tu n'existais pas, il faudrait t'inventer. Ton
énergie et ta force de caractère m'inspireront toujours. Tu vas être
contente d'apprendre que j'ai enfin fini l'école! Et ne t'inquiète pas
pour moi, je suis heureuse et en santé même si je ne mange pas de
viande! Feu grand-maman Gabrielle, ta sagesse et ta bienveillance me
manquent terriblement.

Je remercie tout particulièrement ma famille. Maman, papa, Gabriel (et
Chloé et bébé Marion), Camille (et Alice) et la Fifouille, votre soutien
inconditionnel tout au long de ma vie a rendu possible l'achèvement de
ce travail. Merci pour votre présence et votre confiance presque absolue
en moi. Vous m'avez accompagnée et soutenue à travers toutes les
épreuves de la vie, et le doctorat n'en est qu'un exemple. Mes chers
parents, merci pour nos voyages inoubliables en camping et nos
innombrables randonnées pédestres à travers l'Amérique du Nord. Vous
m'avez transmis votre passion pour la Nature; c'est donc grâce à (ou à
cause de) vous si j'ai choisi la voie de l'écologie. Fifouille, une
chance que tu étais à mes côtés (sur mes genoux ou sur mon clavier) pour
égayer mes journées.

Kevin, je ne te remercierai certainement jamais assez. Depuis notre
rencontre dans un cours de statistique bayésienne, tu éclaires mes
journées comme la belle lumière d'un coucher de soleil\ldots{} Les 600km
qui nous séparaient ne nous ont pas empêchés d'être ensemble (un grand
merci à Via Rail). Tu as été à la fois mon plus grand collaborateur et
mon technicien en informatique. Tu as supporté et enduré mon chouinage
et mon instabilité jusqu'aux dernières étapes de ce projet. Je n'ose
même pas imaginer ce qu'aurait été mon doctorat sans toi! Merci pour ton
calme, ta douceur, tes blagues et tes encouragements. On peut maintenant
déménager dans une nouvelle ville et entreprendre des nouveaux projets,
et surtout commencer notre vie à deux.
