\francais

\chapter*{Avant-propos}

Cette thèse est rédigée sous forme d'articles scientifiques, qui forment
le c\oe{}ur du document, tandis que l'introduction et la conclusion
générales viennent compléter l'histoire. Deux des articles sont déjà
publiés ou acceptés à des revues avec comité de lecture, et un article
reste en préparation pour soumission.

\begin{enumerate}
\def\labelenumi{\arabic{enumi}.}
\item
  Brice, M.H., Cazelles, K., Legendre, P., \& Fortin, M. (2019).
  Disturbances amplify tree community responses to climate change in the
  temperate--boreal ecotone. Global Ecology and Biogeography, 28(11),
  1668--1681. \url{https://doi.org/10.1111/geb.12971}
\item
  Brice, M.H., Vissault, S., Vieira, W., Gravel, D., Legendre, P., \&
  Fortin, M. Moderate disturbances accelerate forest transition dynamics
  under climate change in the temperate-boreal ecotone of eastern North
  America. Global Change Biology.
  \url{https://doi.org/10.1111/gcb.15143}
\item
  Brice, M.H., Chalumeau, A., Grondin, P., Fortin, M.J \& Legendre, P.
  Northern range shifts of temperate tree saplings in Québec: the role
  of climate, stand composition, soils and disturbances on recruitment
  dynamics. (En préparation)
\end{enumerate}

J'ai élaboré les idées derrière les trois études, collecté les données,
effectué les analyses et rédigé les manuscrits. Mes directeurs Pierre
Legendre et Marie-Josée Fortin ont aussi suivi chaque étape de cette
thèse, ont contribué à l'amélioration des manuscrits. À travers ma
thèse, j'ai eu la chance de collaborer avec plusieurs chercheurs,
Aurélie Chalumeau, Dominique Gravel, Kevin Cazelles, Pierre Grondin,
Steve Vissault et Willian Vieira. Ceux-ci ont contribué de manière
substantielle à la version finale des manuscrits et sont donc inclus en
tant que co-auteurs.

\newpage

\noindent
\textbf{\large Accessibilité et reproductibilité de mes travaux}

L'ensemble des données et des scripts utilisés dans le cadre de ma thèse
sont disponibles librement. Ainsi, les codes nécessaires à la mise en
forme des bases de données climatiques et des données d'inventaires
forestiers sont accessibles à travers le dépôt:
\url{https://github.com/mhBrice/Quebec_data}. Ce dépôt permet de
nettoyer et mettre en forme les données utilisées dans mes trois
chapitres de thèse. Ensuite, chaque chapitre a un dépôt indépendant qui
contient toutes les données nettoyées ainsi que les scripts R pour
reproduire les analyses et les figures. Les scripts pour reproduire les
analyses du chapitre 1 sont disponibles en ligne à l'adresse:
\url{https://github.com/mhBrice/thermophilization}. Les scripts du
chapitre 2 peuvent être consultées en ligne à l'adresse:
\url{https://github.com/mhBrice/transition}. En enfin, le dépôt du
chapitre 3 se trouve à l'adresse:
\url{https://github.com/mhBrice/recrutement}.
