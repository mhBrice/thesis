\francais

\chapter*{Conclusion}

\hypertarget{sommaire-des-ruxe9sultats-comprendre-et-observer-les-changements-ruxe9cents-des-foruxeats}{%
\section{Sommaire des résultats: Comprendre et observer les changements
récents des
forêts}\label{sommaire-des-ruxe9sultats-comprendre-et-observer-les-changements-ruxe9cents-des-foruxeats}}

De nombreuses études utilisent les données d'inventaire forestier pour
tenter de prédire l'avenir sous les changements climatiques
\citep{boulanger_climate_2017, perie_dominant_2016, vissault_biogeographie_2016, meier_climate_2012, iverson_estimating_2008, chen_modeling_2002}.
Mais, s'il est possible de faire des prédictions de grands changements
pour 2050, soit dans 30 ans, ne devrait-on pas déjà commencer à
percevoir les premiers signes de ces changements dans les données
cumulées depuis 1970, il y a 50 ans? Quelles informations pouvons-nous
tirer de ces changements récents? De plus, les projections des effets du
changement climatique sur les forêts ont généralement mis l'accent sur
la capacité des espèces à tolérer les augmentations de température et
les sécheresses et à se disperser, mais ils n'ont pas nécessairement
intégré les effets des perturbations. Étant donné l'importance des
perturbations naturelles et de l'exploitation forestière dans la
dynamique des forêts \citep{turner_disturbance_2010}, ces projections du
futur basées sur le changement climatique en isolation sont-elles
réalistes, voire même trompeuses? Si les perturbations interagissent
avec le changement climatique et exacerbent la mortalité des arbres, les
nouvelles politiques d'aménagement qui reposent sur des modèles
incomplets pourraient bien être mal adaptées.

À travers ma thèse de recherche, j'ai donc tenté de répondre à ces
questionnements. Les trois chapitres ont permis de décrire de multiples
aspects de la dynamique des forêts au cours des dernières décennies en
réponse aux effets combinés du changement climatique et des
perturbations.

\hypertarget{ruxe9organisation-des-communautuxe9s}{%
\subsection{Réorganisation des
communautés}\label{ruxe9organisation-des-communautuxe9s}}

Dans le chapitre 1, je me suis intéressée aux moteurs des changements de
composition dans les communautés forestières. Les perturbations (par
exemple, les coupes à blanc, les épidémies d'insectes, les incendies)
sont les principaux facteurs de changement de composition des
communautés forestières, i.e.~la diversité \(\beta\) temporelle, dans
l'écotone tempéré-boréal. Leurs effets laissent une empreinte à long
terme puisque les perturbations historiques sont les plus importantes.
En revanche, les effets les changements de température et de
précipitation sur les changements des communautés forestières sont très
faibles. Sans approfondir, on pourrait en conclure prématurément que le
changement climatique n'a pas influencé la composition des forêts au
cours des dernières décennies.

Malgré la prévalence des perturbations, l'analyse des changements des
traits écologiques de la communauté m'a permis de montrer une
thermophilisation généralisée des communautés à travers le Québec. Cette
thermophilisation correspond à une augmentation des espèces de climat
chaud au détriment des espèces de climat froid dans les communautés.
Dans la région d'étude, ce processus résulte principalement du gain
d'une espèce tempérée, \emph{Acer rubrum}, et de la perte de deux espèce
boréales, \emph{Abies balsamea} et \emph{Picea mariana}. En outre, la
thermophilisation a été plus grande et s'est étendue plus au nord dans
les communautés modérément perturbées que celles qui n'ont pas été
perturbées ou qui ont subi des perturbations majeures. Ainsi, ces
résultats suggèrent que des perturbations modérées, mais non majeures,
pourraient faciliter les gains d'espèces adaptées aux conditions chaudes
sous l'effet du changement climatique.

En terminant ce chapitre, je suis restée avec une question importante en
tête : la thermophilisation actuelle des forêts indique-t-elle un
changement permanent des écosystèmes forestiers ou bien seulement une
dynamique transitoire? Est-ce que les perturbations pourraient favoriser
des changements d'états rapides et permanents? C'est la question sur
laquelle je me suis penchée dans le second chapitre de ma thèse.

\hypertarget{la-dynamique-des-foruxeats}{%
\subsection{La dynamique des forêts}\label{la-dynamique-des-foruxeats}}

Dans le chapitre 2, j'ai analysé la dynamique de transition des forêts
du Québec en utilisant un modèle à quatre états, soit boréal, mixte,
tempéré et pionnier. Cette analyse a d'abord révélé une forte proportion
de transition de l'état pionnier vers boréal, qui signale une
régénération de la forêt boréale anciennement perturbée, et de
transition de l'état mixte vers tempéré, qui suggère une augmentation de
la proportion d'espèces tempérées. Encore une fois, les perturbations
naturelles et anthropiques ressortent comme les moteurs principaux de la
dynamique de transition des forêts au cours des dernières décennies.
Alors que les perturbations majeures déclenchent surtout des transitions
vers l'état pionnier, les perturbations modérées favorisent les
transitions de l'état mixte vers l'état tempéré.

De plus, ces changements dans les probabilités de transition que nous
percevons dans 48 ans de données risquent de se répercuter sur les
futurs patrons forestiers à l'échelle régionale et sur la vitesse de ces
changements. En effet, les perturbations modérées accélèrent le taux de
renouvellement des forêts et, à long terme, favorisent une augmentation
de la proportion de forêts tempérées dans le paysage. Par conséquent,
les perturbations modérées ont le potentiel de catalyser un déplacement
plus rapide de l'écotone boréal-tempéré vers le nord sous le changement
climatique.

Les transitions des forêts mixtes à tempérées résultent surtout de la
mortalité d'une espèce boréale dominante, \emph{Abies balsamea}, et les
trouées ainsi créées permettent la croissance accrue des espèces
tempérées compagnes. Contrairement aux hypothèses avancées dans mon
premier chapitre, le recrutement des espèces tempérées semble jouer un
rôle négligeable dans cette dynamique, mais il se peut que la méthode
d'analyse ne permette pas de détecter la contribution du recrutement.

Que les perturbations entraînent la mortalité de \emph{Abies balsamea}
étaient prévisibles. Et la croissance subséquente des espèces tempérées
compagnes montrent qu'il y a sans doute un changement de rapport des
forces compétitives entre les espèces en lien avec le changement
climatique. Si les transitions reposent davantage sur la mortalité et
que celle-ci n'est pas compensée par le recrutement de nouveaux arbres,
il s'en suit que les forêts mixtes pourraient en fin de compte dépérir.
Le recrutement est une étape essentielle pour assurer la régénération
des forêts et la migration des arbres. Alors que les perturbations
favorisent la thermophilisation des forêts et la transition de
peuplements mixtes à tempérés, quel est leur impact sur le recrutement
des espèces tempérées?

\hypertarget{la-ruxe9guxe9nuxe9ration-des-foruxeats-les-premiers-pas-de-la-migration}{%
\subsection{La régénération des forêts: les premiers pas de la
migration}\label{la-ruxe9guxe9nuxe9ration-des-foruxeats-les-premiers-pas-de-la-migration}}

C'est la question qui a motivée mon troisième et dernier chapitre. Ce
chapitre a mis en lumière de grands déplacements de distribution vers le
nord pour les gaulis (i.e., jeunes arbres entre 1 et 9 cm de diamètre)
de \emph{Acer rubrum} et \emph{Acer saccharum}, deux espèces reconnues
pour leur grande tolérance aux conditions environnementales et aux
perturbations. La distribution de ces deux espèces s'est déplacée vers
le nord peu importe le niveau de perturbations (mineur, modéré ou
majeur). En revanche, la distribution des gaulis de \emph{Betula
alleghaniensis} s'est déplacée vers le nord seulement dans les parcelles
peu ou pas perturbées, alors que dans les parcelles modérément ou
sévèrement perturbées, elle montrait plutôt des signes de contraction
d'aire. Ces résultats soulignent que les futures modifications d'aire de
répartition des arbres peuvent dépendre de la réponse des espèces aux
perturbations.

La distribution des espèces tempérées dans les populations marginales à
leur limite nord est contrainte aux sommets de pente
\citep{goldblum_age_2002, tremblay_potential_2002, barras_supply_1998},
là où le microclimat est plus chaud. Mes analyses révèlent que la
distribution altitudinale des gaulis tend à descendre vers le bas des
pentes, surtout dans les régions à leur limite nord. Ces tendances, bien
que non significatives, peuvent signaler le début d'une migration des
populations marginales qui sont isolées aux sommets des collines. Tel
que suggéré pour la migration postglaciaire des arbres
\citep{mclachlan_molecular_2005}, ces populations marginales pourraient
jouer un rôle très important dans la migration future des arbres en
réponse au changement climatique.

Mes résultats soulignent que le recrutement des gaulis d'espèces
tempérées à leur limite nord est largement favorisé par l'abondance
d'arbres conspécifiques pour la dispersion des graines, mais freiné par
la compétition des espèces boréales. De plus, les conditions de mauvais
drainage (sites hydriques) dans les bas de pente ne sont pas propice à
l'établissement des gaules, tandis que les coupes forestières modérées,
en diminuant la compétition et en libérant les ressources, ont une
influence bénéfique. L'ensemble des résultats de ce chapitre suggère
que, malgré l'effet positif d'une exploitation forestière modérée sur le
recrutement, la prévalence des contraintes locales associées à la
composition des peuplements et à la position topographique risque de
freiner la migration des espèces tempérées vers le nord.

\hypertarget{les-perturbations-catalyseurs-de-changements-dans-les-foruxeats}{%
\section{Les perturbations --- catalyseurs de changements dans les
forêts}\label{les-perturbations-catalyseurs-de-changements-dans-les-foruxeats}}

Dans l'ensemble, mes résultats soulignent le rôle central des
perturbations dans la réponse des forêts face au changement climatique.
En effet, la composition et la structure (Chapitre 1), la dynamique de
transition (Chapitre 2), ainsi que la dynamique de régénération
(Chapitre 3) des forêts de l'écotone boréal-tempéré au Québec sont
principalement contrôlées par les perturbations et leurs effets semblent
interagir avec ceux des changements climatiques. J'ai montré comment les
perturbations accélèrent la réponse des communautés forestières aux
changements climatiques, révélant des synergies qui ont le potentiel de
modifier l'avenir de nos forêts.

\hypertarget{les-perturbations-et-la-dynamique-forestiuxe8re}{%
\subsection{Les perturbations et la dynamique
forestière}\label{les-perturbations-et-la-dynamique-forestiuxe8re}}

Les perturbations font partie intégrante de la dynamique des forêts.
Feux, épidémies d'insectes, chablis, inondations sont des événements de
mortalités aigues et à court terme \citep[\emph{pulse
disturbance};][]{bender_perturbation_1984} qui s'inscrivent dans la
dynamique naturelle des forêts et permettent leur renouvellement
\citep{attiwill_disturbance_1994}. Les espèces forestières sont
généralement bien adaptés à un régime de perturbation naturelle, défini
par la superficie, la sévérité et la fréquence
\citep{turner_disturbance_2010}. Par exemple, dans la forêt boréale, les
cônes sérotineux de \emph{Pinus banksiana} nécessitent des températures
élevées pour s'ouvrir et lui permettent de coloniser rapidement un site
après un feu \citep{burns_silvics_1990}. Dans la forêt tempérée, les
semis et les gaules de \emph{Acer saccharum} peuvent persister longtemps
dans un sous-bois ombragé jusqu'à ce qu'une trouée créée par la chute
des vieux arbres lui permette de rejoindre la canopée rapidement
\citep{burns_silvics_1990}. Ainsi, les communautés forestières ont une
bonne capacité à se rétablir et à revenir à leur état initial suite à
une perturbation importante, i.e.~elles sont résilientes
\citep{gunderson_ecological_2000}.

Les perturbations anthropiques viennent s'ajouter ou se substituer aux
perturbations naturelles pour créer un tout nouveau régime de
perturbations \citep{boucher_land_2014}. Les coupes forestières ont
tendance à homogénéiser et rajeunir le paysage car leur période de
rotation est plus courte, tandis que leur superficie et leur intensité
sont plus homogènes
\citep{mcrae_comparisons_2001, boucher_logging-induced_2006}. Les forêts
présentent une certaine résilience face aux coupes forestières; tant que
le régime de coupes forestières restent dans la plage de variabilité
naturelle des forêts \citep{grondin_have_2018}, la plupart des espèces
peuvent se régénérer après coupe. Toutefois, à long terme, des coupes
répétées peuvent entraîner d'importants changements de composition, par
exemple en favorisant l'augmentation des espèces pionnières intolérantes
à l'ombre, comme \emph{Populus tremuloides}, et des espèces tolérantes
aux perturbations, comme \emph{Acer rubrum}
\citep{danneyrolles_stronger_2019, boucher_logging-induced_2006}.

\hypertarget{le-changement-climatique-uxe9rode-la-ruxe9silience}{%
\subsection{Le changement climatique érode la
résilience}\label{le-changement-climatique-uxe9rode-la-ruxe9silience}}

En plus de ces perturbations de type \emph{pulse}, le climat global
change graduellement et de façon persistante \citep[\emph{press
disturbance};][]{bender_perturbation_1984}, ce qui peut altérer la forme
du bassin d'attraction (Fig. \ref{fig0.4}), rendant les forêts plus
fragiles face aux autres perturbations
\citep{scheffer_catastrophic_2001}. En effet, au fur et à mesure que le
climat se réchauffe, le déséquilibre entre la répartition de certaines
espèces et les conditions climatiques s'agrandit
\citep{talluto_extinction_2017}. À la marge sud de leur répartition,
certaines populations peuvent persister mais sont vouées à l'extinction
puisque les conditions environnementales ne sont plus propices à leur
régénération ou leur survie; on observe une dette d'extinction. À leur
marge nord, de nouveaux habitats sont devenus suffisamment chauds mais
n'ont toujours pas été colonisés en raison de différentes contraintes au
recrutement (e.g., distance, barrière, prédation des graines); on
observe un crédit de colonisation
\citep{jackson_balancing_2010, tilman_habitat_1994}. Les forêts peuvent
ainsi perdre leur capacité à se rétablir, puisque les espèces en place
ne sont plus aussi bien adaptées au climat et pourront être remplacées
advenant une perturbation \citep{johnstone_changing_2016}. Cette perte
de résilience peut entraîner un basculement vers un nouvel état de
l'écosystème \citep{scheffer_catastrophic_2001}.

\hypertarget{interaction-changement-climatique-et-perturbations}{%
\subsection{Interaction changement climatique et
perturbations}\label{interaction-changement-climatique-et-perturbations}}

En théorie, des changements d'états stables peuvent subvenir suivant
deux mécanismes: (1) des changements graduels dans les conditions
environnementales jusqu'à un niveau critique auquel le système
s'effondre soudainement, et (2) des perturbations trop sévères ou en
rafale qui poussent le système hors de son bassin d'attraction
\citep{scheffer_catastrophic_2001}. Par exemple, la grande augmentation
des taux de mortalité des arbres dans l'ouest des États-Unis en réponse
à un stress hydrique grandissant \citep{van_mantgem_apparent_2007}
pourrait être le signe avant-coureur d'un dépérissement massif des
forêts. Dans un autre cas, \citet{payette_shift_2003} a montré que les
impacts cumulés de la coupe forestière, suivie d'une épidémie d'insectes
puis d'un incendie pourraient avoir des effets catastrophiques sur la
régénération des arbres, entraînant la transition d'une pessière dense
en un milieu ouvert dominé par le lichen. Bien que les deux mécanismes
puissent indépendamment mener à une transition rapide d'état, leurs
effets cumulés augmentent le risque de changements rapides de régime
écologique \citep[\emph{regime
shift};][]{scheffer_catastrophic_2001, harris_biological_2018}. Mes
résultats supportent cette hypothèse et montrent que le réchauffement
climatique érode la résilience des forêts mixtes tandis que les
perturbations éliminent les espèces boréales en place et accélèrent le
processus de succession vers davantage d'espèces tempérées adaptées aux
températures plus chaudes (Chapitres 1, 2). Sans perturbation, la grande
inertie des forêts cache la perte de résilience
\citep{johnstone_changing_2016}. Les arbres, ayant une longue durée de
vie, peuvent faire paraître les forêts inébranlables face aux
changements environnementaux même si la niche de régénération est en
train de se déplacer \citep[Chapitre
3;][]{sittaro_tree_2017, boisvertmarsh_divergent_2019}.

\hypertarget{uxe9tats-alternatifs-stables-et-changement-de-ruxe9gime}{%
\subsection{États alternatifs stables et changement de
régime}\label{uxe9tats-alternatifs-stables-et-changement-de-ruxe9gime}}

Suite à une perturbation modérée, la composition des forêts mixtes peut
donc se déplacer rapidement vers une dominance en espèces tempérées qui
sont mieux adaptées aux températures plus chaudes. Le long du gradient
latitudinal, la forêt boréale au nord et la forêt tempérée au sud sont
les seuls états stables possibles (Fig. \ref{fig4.1}a). Toutefois, dans
la zone de transition entre ces deux grands biomes, on trouve la forêt
mixte, un état relativement rare qui change facilement d'un état de
dominance à un autre. Ceci suggère que la forêt mixte représente un état
instable où le système n'est que transitoire. La coexistence des espèces
boréales, tempérées et pionnières dans la forêt mixte s'est maintenue
grâce à l'hétérogénéité des perturbations naturelles combinée à des
différences dans les stratégies de cycle de vie des espèces
\citep{kneeshaw_natural_2007, bouchard_tree_2006} sous un climat donné.
Mon interprétation est que le réchauffement modifie le bassin
d'attraction de l'état mixte et abaisse le seuil critique à franchir
pour se rendre à l'état tempéré(Fig. \ref{fig4.1}b). De plus, le
réchauffement peut aussi rendre le bassin de l'état tempéré plus profond
et celui de l'état boréal moins profond. Ces modifications de la courbe
des états alternatifs stables fragilisent la forêt mixte et la rendent
encore plus vulnérable aux transitions vers l'état tempéré suite à une
perturbation modérée (Fig. \ref{fig4.1}b). Une fois que les espèces
tempérées feuillues deviennent dominantes elles peuvent alors se
maintenir par une boucle de rétroaction positive, par exemple en
modifiant la fertilité des sols, la luminosité du sous-bois, et le
régime de perturbation.

\begin{figure}
\centering
\includegraphics[width=.65\textwidth]{conclusion/figures/etat_alternatif2.png}
\caption[Représentation conceptuelle des états alternatifs stables le long du gradient latitudinal]{Mise à jour de ma représentation conceptuelle des états alternatifs stables le long du gradient latitudinal sans changement climatique (a) et avec changement climatique (b). Dans ce schéma avec la courbe, la boule caractérise l'état de l'écosystème à un instant donné, le paysage correspond à l'ensemble des états dans lesquels l’écosystème peut se retrouver, les vallées sont les bassins d'attraction des équilibres stables, et les sommets des collines sont les équilibres instables. Le schéma rectangulaire représente le paysage correspondant. Au sud du gradient latitudinal au de température, un seul état stable existe, la forêt tempérée, boule rouge. Au nord du gradient, l'état stable dominant est la forêt boréale, boule bleue. Ce sont des états stables dynamiques; les perturbations peuvent faire déplacer la boule dans son bassin d'attraction et elle peut même passer par l'état pionnier (un état transitoire non représenté sur la courbe, mais représenté par des carrés jaunes dans le paysage). La forêt étant habituellement résiliente aux perturbations, elle retourne ensuite vers son état initial. Au centre, dans la zone d'écotone, la forêt mixte, boule verte, serait un état transitoire entre les deux états stables dominants qui est maintenue grâce à la dynamique naturelle de perturbations. Le changement du climat (b) peut provoquer un changement de la forme du paysage de différentes façons: (1) en abaissant le seuil pour passer de l'état mixte à tempéré, (2) en creusant et en élargissant le bassin d'attraction de l'état tempéré et (3) en rendant moins profond le bassin d'attraction de l'état boréal. Ces modifications font que la boule dégringole plus souvent de la vallée mixte à la vallée sous l'effet d'une perturbation.}
\label{fig4.1}
\end{figure}

Contrairement aux perturbations modérées, les perturbations majeures
détruisent toutes la communauté en place et poussent le système vers
l'état pionnier, i.e. des peuplements dominés par des espèces
intolérantes à l'ombre, comme \emph{Betula papyrifera} et \emph{Populus
tremuloides}, ou bien avec pas ou très peu d'arbres. Leur effet à long
terme est difficile à prévoir à partir des données d'inventaire puisque
les systèmes n'ont sans doute pas eu le temps de revenir à un état
stable. En effet, l'état pionnier est en général un état transitoire. Il
est donc fort probable que la majorité des forêts soient encore en train
de se déplacer vers leur état d'équilibre. Toutefois, il se pourrait que
certaines forêts se soient effondrées définitivement. Par exemple, les
peuplements de \emph{Populus tremuloides} dans les pessières noires
représentent normalement un état de transition, mais, sous l'effet des
coupes forestières, ces peuplements sont en expansion et semblent se
maintenir \citep{grondin_les_2003}.

\hypertarget{ruxe9silience-et-capacituxe9-adaptative}{%
\subsection{Résilience et capacité
adaptative}\label{ruxe9silience-et-capacituxe9-adaptative}}

La biodiversité est un élément clé de la résilience et la capacité
adaptative des forêts
\citep{messier_functional_2019, filotas_viewing_2014}. Alors que la
résilience permet à une forêt de retrouver sa structure et ses fonctions
d'origine, la capacité adaptative lui permet de diverger d'un état
antérieur qui était mal adapté aux conditions environnementales
\citep{messier_managing_2013, filotas_viewing_2014}. Bien que moins
résilientes, les forêts mixtes montrent une bonne capacité adaptative
face au changement climatique puisqu'elles arrivent à se réorganiser de
manière à ajuster leur composition aux nouvelles conditions
environnementales \citep{messier_functional_2019, filotas_viewing_2014}.
Mais qu'en est-il des forêts boréales pures? Contrairement aux forêts
mixtes, celles-ci montrent peu de changements de composition en réponse
au réchauffement climatique. En effet, alors qu'il y a eu très peu de
transitions vers l'état mixte et pas de thermophilisation des
communautés, on a plutôt observé une dynamique de remplacement entre les
états pionnier et boréal (Chapitres 1 et 2). Comme les forêts du nord du
Québec sont très pauvres en espèces, étant largement dominées par
\emph{Picea mariana} et \emph{Abies balsamea}, elles ont moins de
ressources pour faire face aux changements récents et futurs et ce qui
pourrait limiter leur capacité à s'ajuster et s'éloigner d'un état
possiblement mal adapté. De plus, on prévoit que le climat de la forêt
boréale de l'est de l'Amérique du Nord devrait ressembler à celui de la
forêt tempérée d'ici la fin du siècle \citep{gauthier_boreal_2015}.
Toutefois, la migration des espèces tempérées dans ces régions semble
limitée par plusieurs facteurs non-climatiques, notamment leur capacité
de dispersion, la compétition par les espèces boréales, ainsi que les
conditions édaphiques \citep[Chapitre
3;][]{solarik_priority_2019, carteron_soil_2020}. Ainsi, si le
réchauffement continue de s'accentuer et que les espèces tempérées ne
parviennent pas à coloniser les régions boréales, on peut se demander
comment se transformeront les sapinières et les pessières du Québec.

La fréquence et la gravité des perturbations naturelles, telles que les
incendies, les épidémies d'insectes, les sécheresses et les vagues de
chaleur, devraient augmenter dans de nombreuses régions du monde
\citep{seidl_forest_2017, bergeron_past_2006}. À la lumière de mes
résultats, cela pourrait conduire à des changements majeurs dans la
composition des forêts au cours des prochaines décennies et
potentiellement à des modifications permanentes des états forestiers.
Cependant, si les perturbations deviennent trop fréquentes et trop
intenses, les forêts pourraient basculer vers une dominance en espèces
pionnières de début de succession. Des comportements non-linéaires dans
les réponses des écosystèmes forestiers impliquent que de nombreuses
projections sous-estiment probablement l'ampleur des changements futurs
de la biodiversité \citep{scheffer_catastrophic_2001}. Une telle
conclusion souligne l'importance de tenir compte de l'effet synergique
des perturbations et du changement climatique dans les stratégies de
gestion forestière ainsi que dans les modèles de prédiction.

\hypertarget{des-foruxeats-en-transformation-des-individus-aux-biomes}{%
\section{Des forêts en transformation: des individus aux
biomes}\label{des-foruxeats-en-transformation-des-individus-aux-biomes}}

Les effets des changements environnementaux se répercutent à chaque
niveau d'organisation de la biodiversité, se transmettant des individus
jusqu'au biome, en passant par les populations, les communautés et les
écosystèmes \citep{bellard_impacts_2012, parmesan_globally_2003}. En
effet, conformément avec les concepts de la science des systèmes
complexes, les changements démographiques au bas de la hiérarchie
peuvent faire émerger, par des processus d'organisation autonome, des
réorganisations massives à l'échelle régionale. De plus, les
interactions entre n'importe quel niveau de la hiérarchie peuvent donner
naissance à des dynamiques non-linéaires, comme les changements de
régime écologique \citep[Fig.
\ref{fig4.2};][]{filotas_viewing_2014, messier_managing_2013}. Les
résultats présentés dans mes trois chapitres de thèse permettent de bien
illustrer ces processus ascendants et en interaction par lequel les
forêts de l'écotone boréal-tempéré sont en train de se transformer.
L'interprétation de mes résultats sous la perspective des systèmes
complexes permet de mieux comprendre la réponse des forêts sous l'effet
combiné du changement climatique et des perturbations.

\begin{figure}
\centering
\includegraphics[width=.9\textwidth]{conclusion/figures/complex.png}
\caption[Représentation conceptuelle des effets des changements environnementaux sur les différents niveaux d'organisation biologique]{Représentation conceptuelle des effets des changements environnementaux sur les forêts sous l'angle d'un système complexe. Les changements dans la dynamique des forêts sont transférés de façon ascendante entre les différents niveaux d'organisation biologique. Des interactions et des rétroactions ont lieu entre les entités à l'intérieur et à travers les échelles spatiales, temporelles et hiérarchiques. Les entités qui interagissent à un niveau donnent naissance à des entités émergentes de niveau supérieur, dont l'existence, à son tour, affecte le comportement des entités de niveau inférieur. Schéma issu de Messier \emph{et al.} 2013.}
\label{fig4.2}
\end{figure}

\hypertarget{changements-duxe9mographiques}{%
\subsection{Changements
démographiques}\label{changements-duxe9mographiques}}

Dans un premier temps, le réchauffement climatique favorise le
recrutement, la survie et la croissance des espèces tempérées à leur
limite nord \citep[Chapitre
3;][]{fisichelli_temperate_2014, boisvertmarsh_divergent_2019, peng_drought-induced_2011, goldblum_tree_2005, grundmann_impact_2011, bolte_understory_2014},
ce qui leur confère un avantage compétitif par rapport aux espèces
boréales. Mais, en l'absence de perturbations, les arbres poussent
lentement, leurs taux de mortalité et de recrutement sont faibles et la
compétition pour l'espace et la lumière est intense. Ainsi, le
renouvellement de la communauté est très lent.

Des perturbations modérées peuvent cependant éliminer les individus des
espèces résidentes. Dans la zone d'étude, les perturbations naturelles
et anthropiques ont provoqué une mortalité disproportionnée d'une espèce
dominante dans les forêts mixtes (Chapitre 2). En effet, \emph{Abies
balsamea} a subi une mortalité massive suite aux grandes épidémies de
tordeuse des bourgeons de l'épinette dans les années 1967-1992
\citep{duchesne_population_2008}. De plus, cette espèce est aussi la
plus coupée au Québec. Les trouées dans la canopée résultant de la perte
de cette espèce boréale ubiquiste et abondante ont probablement permis
de réduire la compétition et libérer des ressources, favorisant ensuite
la croissance rapide des espèces tempérées compagnes, telles que
\emph{Acer saccharum}, \emph{A. rubrum} et \emph{Betula alleghaniensis}
(Chapitre 2). De plus, alors que les perturbations naturelles semblent
avoir un effet plutôt délétère, les coupes partielles favorisent une
hausse du recrutement de ces espèces tempérées à leur limite nord
(Chapitre 3).

\hypertarget{dynamique-de-population}{%
\subsection{Dynamique de population}\label{dynamique-de-population}}

Le réchauffement et les perturbations peuvent donc exercer leur
influence sur la dynamique de population des espèces par le biais de
plusieurs processus démographiques, notamment la reproduction, le
recrutement, la croissance et la mortalité. Ces changements à l'échelle
des individus et des sites s'accumulent dans le temps et dans l'espace
pour altérer l'abondance et l'aire de répartition des populations
\citep{holt_theoretical_2005}. Si les effets sur les individus d'une
espèce sont généralement négatifs, le taux de croissance de la
population diminuera; certaines populations pourraient être amenées à
disparaître localement et, dans le pire des cas, régionalement. À
l'inverse, lorsque les effets sur les taux démographiques d'une espèce
sont positifs, les populations grandissent, ce qui peut entraîner une
augmentation locale de l'abondance et une expansion régionale de l'aire
de répartition.

Ces effets sur le fitness des espèces ne sont pas aléatoires, mais sont
déterminés par leurs tolérances physiologiques, leurs stratégies
d'histoire de vie et leurs capacités de dispersion
\citep{aubin_traits_2016, estrada_species_2015}. Ces caractéristiques
écologiques spécifiques sont à l'origine de la grande variabilité dans
les réponses des arbres face aux changements environnementaux. Alors que
le réchauffement peut favoriser le taux de croissance des populations
d'espèces qui sont limitées par les températures très froides
\citep{de_frenne_microclimate_2013, devictor_birds_2008}, les
perturbations devraient promouvoir les espèces opportunistes,
intolérantes à l'ombre, avec une bonne capacité de dispersion et de
reproduction végétative \citep{danneyrolles_stronger_2019}. Par exemple,
\emph{Acer rubrum}, dont la distribution au nord est en partie limitée
par un faible niveau de reproduction sexuée
\citep{tremblay_potential_2002}, est reconnue comme une espèce
super-généraliste et opportuniste, capable de coloniser rapidement des
sites variés après une perturbation
\citep{abrams_red_1998, fei_rapid_2009}. Ces caractéristiques pourraient
donc expliquer le grand succès de \emph{Acer rubrum} sous l'effet
combiné du réchauffement et des perturbations. En revanche, certaines
espèces limitées par la dispersion, comme \emph{Tilia americana},
limitées à un habitat spécifique, comme \emph{Acer saccharinum}, ou
encore intolérante aux perturbations, comme \emph{Tsuga canadensis},
pourrait ne pas bénéficier des opportunités de recrutement et de
croissance suivant des perturbations, et ce, même si le climat devient
plus favorable.

De plus, la sensibilité aux variations climatiques peut varier entre les
différents processus démographiques \citep{niinemets_responses_2010}.
Par exemple, comme la régénération est souvent plus sensible que la
survie des adultes aux stress hydrique ou thermique
\citep{niinemets_responses_2010}, une population peut persister pendant
des décennies ou des siècles sur un site donné, même si les conditions
climatiques sont devenues défavorables \citep{talluto_extinction_2017}.
Pour cette raison, \citet{jump_altitude-for-latitude_2009} ont suggéré
que l'expansion à la limite nord de la distribution d'une espèce sera
plus rapide que les changements à la limite sud puisque la reproduction
et le recrutement sont plus sensibles aux changements environnementaux
que la mortalité des individus établis. Dans les forêts du Québec, les
données montrent effectivement une augmentation rapide du recrutement de
plusieurs espèces tempérées (Chapitre 3). Toutefois, les changements
découlant de la mortalité des espèces boréales étaient plus rapides
puisque ce processus n'était pas contrôlé par un stress climatique mais
surtout par des perturbations directes (e.g., coupe, feu, épidémie).
Ainsi, les perturbations risquent d'accélérer la contraction de l'aire
de répartition des espèces boréales, tandis que l'effet sur l'expansion
des espèces tempérées à leur limite nord pourrait être beaucoup plus
lent en raison des contraintes au recrutement (Chapitre 3).

\hypertarget{dynamique-des-communautuxe9s}{%
\subsection{Dynamique des
communautés}\label{dynamique-des-communautuxe9s}}

Ces changements démographiques se traduisent donc en pertes et en gains
d'espèces à l'échelle locale et influencent la composition et la
structure des communautés forestières. Les gains en espèces tempérées
combinés aux pertes en espèces boréales ont entraîné une
thermophilisation de nombreuses communautés au Québec, particulièrement
suivant des perturbations modérées (Chapitre 1). J'ai pu identifier
trois mécanismes qui contribuent conjointement aux changements de
composition en altérant la trajectoire de succession après perturbation:
(1) la mortalité d'une espèce dominante; (2) le relâchement de la
croissance des espèces compagnes (Chapitre 2); et (3) le recrutement
accru des gaules des espèces compagnes et présentes dans le voisinage
(Chapitre 3).

Les conséquences au niveau des communautés des changements d'aires de
répartition spécifiques aux espèces pourraient mener à la formation de
communautés sans analogues, i.e.~des communautés dans lesquelles
coexistent des espèces dans des combinaisons historiquement inconnues
\citep{williams_novel_2007}. Dans la zone de transition entre le biome
boréal et tempéré, l'effet combiné du réchauffement et des perturbations
semblent particulièrement favorables à une poignée d'espèces seulement,
avec en première position \emph{Acer rubrum} (Chapitre 1-3). Les modèles
de distribution d'espèces sous l'effet du changement climatique
prédisent un grand potentiel d'augmentation de la richesse au Québec
\citep{berteaux_changements_2014}. Toutefois, si seul un nombre
restreint d'espèces tempérées prospèrent sous ce nouveau régime de
perturbations anthropiques, il est fort probable que les forêts de
l'écotone ne connaîtront pas un enrichissement, mais plutôt un
appauvrissement associé à l'expansion d'une ou quelques espèces. De
plus, que deviendront ces forêts si cette réorganisation se fait au
détriment des espèces boréales actuellement dominantes, notamment
\emph{Abies balsamea} et \emph{Picea spp.}? Le déclin de ces espèces
résineuses risque de s'accentuer dans les prochaines décennies dans les
domaines de la sapinière en raison du réchauffement climatique
\citep{dorangeville_beneficial_2018}. Le remplacement de ces espèces
résineuses par des espèces feuillues pourrait avoir des conséquences
économiques importantes. Par exemple, l'expansion de \emph{Acer rubrum}
pourrait compromettre l'approvisionnement en espèces résineuses à grande
valeur commerciale dans la forêt mixte. De plus, les pratiques
sylvicoles devront être révisées pour s'adapter à la nouvelle réalité
puisqu'elles sont élaborées en fonction de la composition et de la
dynamique naturelle des forêts et dépendent de la régénération naturelle
\citep{pinna_amenagement_2009}.

\hypertarget{duxe9placement-des-grands-biomes-forestiers}{%
\subsection{Déplacement des grands biomes
forestiers}\label{duxe9placement-des-grands-biomes-forestiers}}

Les effets cumulés des changements à l'échelle des individus, des
populations et des communautés, peuvent finalement provoquer des
changements de régime (\emph{regime shift}), dans lesquels le
déséquilibre pousse rapidement le système dans un nouvel état
\citep{scheffer_catastrophic_2001}. Ces changements de régime se
traduisent par un déplacement des grands biomes forestiers à l'échelle
régionale, notamment une expansion de la forêt tempérée au détriment de
la forêt mixte (Chapitre 2). Cette réorganisation régionale de la
composition des forêts peut interagir avec le fonctionnement des
écosystèmes à l'échelle locale et les processus à l'échelle globale
\citep[\emph{cross-scale
interaction};][]{peters_crossscale_2007, messier_managing_2013}. Ces
changements de fonctionnement risquent d'être d'autant plus grand à
l'écotone puisque les espèces feuillues et les espèces résineuses
présentent des différences importantes sur le plan de leurs
caractéristiques et fonctions écologiques
\citep{wardle_terrestrial_2011}. L'enfeuillement des forêts mixtes
pourrait par exemple influencer localement la qualité de la litière, le
taux de décomposition de la matière organique et la composition des
microorganismes du sol
\citep{laganiere_how_2010, legare_influence_2005}. Les changements dans
la composition, la structure d'abondance et la distribution spatiale des
forêts affecteront également la survie et la distribution des nombreuses
espèces de mammifères, d'oiseaux et d'insectes qui en dépendent pour
s'abriter et se nourrir \citep{friggens_effects_2018}. Comme les espèces
feuillues sont moins inflammables et moins sensibles aux insectes
ravageurs que les conifères, leur augmentation dans le paysage forestier
peut modifier le régime régional de perturbations
\citep{terrier_potential_2013, mffp_insectes_2018}. À long terme,
l'expansion du biome tempéré au détriment des forêts mixtes et des
forêts boréales pourrait avoir un effet sur le climat global en
augmentant la séquestration du carbone \citep{thurner_carbon_2014} ainsi
que l'albédo \citep{anderson_biophysical_2011}.

\hypertarget{lamuxe9nagement-forestier-dans-un-monde-en-changement-et-incertain}{%
\section{L'aménagement forestier dans un monde en changement et
incertain}\label{lamuxe9nagement-forestier-dans-un-monde-en-changement-et-incertain}}

Les effets multiples des changements globaux sur la dynamique forestière
soulèvent un défi majeur pour l'aménagement de nos forêts. Face à la
rapidité et à l'incertitude de ces changements, nos pratiques visant à
contrôler et prédire l'évolution de nos forêts risquent d'être
contreproductive \citep{puettmann_critique_2009}. Étant donné que les
coupes forestières ont une influence majeure sur la composition
(Chapitre 1), la dynamique (Chapitre 2) et la régénération des forêts
(Chapitre 3) et interagissent avec le changement du climat, il est
évident que les futures politiques d'aménagement auront un rôle
fondamental à jouer pour aider les forêts à s'adapter rapidement et
faire face aux changements globaux.

Le réchauffement que nous avons connu jusqu'à présent n'est que mineur
par rapport à ce qui est attendu d'ici la fin du siècle. Néanmoins, tel
que mis en évidence dans l'ensemble de ma thèse, de grandes
transformations sont déjà évidentes à toutes les échelles de
l'organisation biologique. Avec l'accélération des changements
environnementaux et l'inertie inhérente de la dynamique forestière, le
déséquilibre ne pourra que s'accentuer et les réponses des forêts
dépendront des dynamiques transitoires déjà en cours. Actuellement, les
principes fondamentaux de l'estimation de la productivité des
peuplements forestiers reposent sur des taux de mortalité et de
croissance prévisibles sous un climat constant. En aménagement
forestier, on présume que le climat est stable et que les forêts sont à
équilibre. À court terme, ces hypothèses sont valables. Mais, à long
terme, elles sont particulièrement problématiques dans le contexte du
changement climatique. En effet, comme la dynamique forestière n'est pas
à l'équilibre \citep{talluto_extinction_2017}, la trajectoire de
succession et la niche de régénération sont facilement altérées sous
l'effet combiné du réchauffement et des perturbations (Chapitre 2, 3).
Les prédictions issues de calculs qui supposent l'équilibre pourraient à
la fois sur-estimer la capacité de régénération et de croissance de
certaines espèces et sous-estimer la mortalité associée à des extrêmes
climatiques, menant ainsi à une planification mal adaptée. Par
conséquent, les activités de gestion doivent non seulement anticiper le
changement, mais aussi reconnaître que les systèmes actuels ont déjà été
transformés et sont en train de se transformer davantage. L'importance
de ce point a été bien exprimée par \citet{seastedt_management_2008} :

\begin{quote}
In managing novel ecosystems, the point is not to think outside the box
but to recognize that the box has moved, and in the 21st century, the
box will continue to move more rapidly.
\end{quote}

\hypertarget{uxe9volution-de-lamuxe9nagement-au-quuxe9bec}{%
\subsection{Évolution de l'aménagement au
Québec}\label{uxe9volution-de-lamuxe9nagement-au-quuxe9bec}}

Une importante volonté de gestion durable des forêts s'est développée
dans les dernières décennies à travers le monde. Jusqu'à la fin du
XX\textsuperscript{e} siècle, les modèles de gestion ont une vision
utililariste de la forêt et sont centrés sur la production
\citep{kuuluvainen_natural_2012}. Depuis les années 1990, la foresterie
a évolué vers un aménagement qui intègre davantage de critères
écologiques et sociaux \citep{messier_managing_2013}. Au Québec, dans la
foulée du documentaire \emph{L'erreur boréale}, sorti en 1999, une
grande réflexion s'est amorcée sur l'exploitation de la forêt publique.
Pour répondre aux inquiétudes de la population, la Commission d'étude
sur la gestion de la forêt publique québécoise est formée en 2003 et
dépose un rapport en 2004 qui fait état de la situation et formule de
nombreuses recommandations pour améliorer et moderniser la gestion des
forêts
\citep{commission_detude_sur_la_gestion_de_la_foret_publique_quebecoise_commission_2004}.
En réponse à ces recommandations, le Québec se dote la Loi sur
l'aménagement durable du territoire forestier, en vigueur depuis 2013,
qui promeut un aménagement écosystémique. L'aménagement écosystémique
s'inspire des patrons spatio-temporels générés par les perturbations
naturelles, qui servent d'états de référence, afin de maintenir les
écosystèmes dans leur plage de variabilité naturelle historique, et
ainsi réduire les écarts entre la forêt naturelle et aménagée
\citep{vaillancourt_implementation_2009, attiwill_disturbance_1994}.

L'aménagement écosystémique représente une grande avancée car il intègre
un ensemble d'objectifs sociaux et écologiques plus larges et reconnait
l'importance de la biodiversité et des processus écologiques qui
influencent la dynamique forestière
\citep{messier_functional_2019, kuuluvainen_forest_2009}. Cette nouvelle
approche de gestion présente néanmoins une lacune majeure; ces pratiques
de gestion ne sont pas conçues pour faire face au rythme rapide des
changements globaux et à l'incertitude croissante qui en découle
\citep{messier_dealing_2016, millar_climate_2007}. En effet, des
pratiques de gestion qui visent à maintenir la composition et la
structure des forêts historiques de référence vont devenir de plus en
plus difficile à mettre en oeuvre
\citep{duveneck_measuring_2016, boulanger_climate_2019} et ne permettent
pas nécessairement d'améliorer la capacité des écosystèmes à s'adapter à
un nouvel ensemble de conditions environnementales
\citep{seastedt_management_2008}. Bien que l'utilisation stricte des
conditions de référence historiques deviendra contre-productive en tant
qu'objectifs spécifiques, les informations historiques documentant la
dynamique naturelle des écosystèmes forestiers seront essentielles pour
mieux comprendre les dynamiques du futur \citep{harris_ecological_2006}.

\hypertarget{amuxe9nager-les-foruxeats-pour-augmenter-leur-ruxe9silience-et-leur-capacituxe9-adaptative}{%
\subsection{Aménager les forêts pour augmenter leur résilience et leur
capacité
adaptative}\label{amuxe9nager-les-foruxeats-pour-augmenter-leur-ruxe9silience-et-leur-capacituxe9-adaptative}}

La solution privilégiée pour faire face au changement climatique est
d'accroître la résilience et la capacité adaptative des forêts
\citep{messier_managing_2013, seastedt_management_2008}. Les
perturbations naturelles et les variations climatiques sont inévitables
mais en développant une grande diversité, les forêts auront les outils
nécessaires pour se réorganiser et s'adapter à des conditions futures
sans précédent \citep{messier_dealing_2016}. S'appuyant sur l'hypothèse
d'assurance \citep[de l'anglais \emph{insurance
hypothesis};][]{yachi_biodiversity_1999}, l'idée est de favoriser la
diversité génétique, spécifique, fonctionnelle et structurale dans les
forêts afin d'augmenter les chances que certaines espèces continueront à
assurer le fonctionnement de l'écosystème même si d'autres
disparaissent.

Pour favoriser la capacité adaptative des forêts, il faut avant tout
maintenir la diversité naturelle que l'on trouve à toutes les échelles
spatiales, du peuplement jusqu'au biome, de manière à garder les options
d'adaptation qui existent déjà. Par exemple, contrairement aux forêts
boréales, la diversité des forêts mixtes leur confère une meilleure
capacité d'adaptation; elles ont plus de trajectoires possibles pour
réagir aux changements (Chapitre 2). En effet, suite à une perturbation,
des trajectoires diversifiées peuvent émerger dans les peuplements
présentant une hétérogénéité en termes de structure, d'âge, de tolérance
physiologique et de stratégies d'histoire de vie. Dans certains cas, il
sera peut-être nécessaire de cultiver activement la capacité adaptative
des écosystèmes grâce à l'aménagement. Ce principe pourrait devenir
important dans les forêts boréales étant donné leur composition très
homogène et leur très grande inertie face au changement climatique
(Chapitres 1, 2). Pour l'instant, les espèces tempérées semblent avoir
de la difficulté à s'établir naturellement en forêt boréale (Chapitre
3). Les plantations d'enrichissement pourraient alors s'avérer utiles
pour faciliter la migration des espèces tempérées vers le nord et
assurer la résilience des forêts \citep{duveneck_measuring_2016}. La
création d'îlots de forêts mixtes sur les sommets de collines dans les
forêts boréales assurerait la présence de semenciers de différentes
espèces capables de coloniser rapidement les sites après perturbation
lorsque les conditions seront adéquates (Chapitre 3). Finalement, étant
donné les interactions entre échelles, les changements de régime
écologique et la variabilité des réponses des espèces, il devient de
plus en plus clair qu'on ne peut forcer un peuplement à se développer
dans une direction précise prédéterminée en fonction de nos besoins en
bois \citep{puettmann_critique_2009}. Des recherches récentes
encouragent donc à revoir la planification de façon à avoir des
objectifs plus larges et plus flexibles qui permettent un ensemble de
différentes trajectoires futures à l'échelle régionale
\citep{messier_dealing_2016, puettmann_critique_2009}.

Cette idée de favoriser la diversité pour assurer la résilience est déjà
prise en compte dans l'aménagement écosystémique et constitue donc une
porte d'entrée à l'adaptation aux changements climatiques
\citep{samuel_foret_2011}. Toutefois, il faut éviter de mettre tous nos
oeufs dans le même panier. Les modèles de projections climatiques nous
ont informé d'un risque croissant de vagues de chaleur et de sécheresses
\citep{ipcc_climate_2014}. Par conséquent, il apparaît logique de mettre
l'emphase sur la promotion des espèces qui résistent à la sécheresse.
Mais, dans un contexte d'incertitude, cette stratégie ne suffit pas
puisqu'il est possible que ce ne soit pas la sécheresse qui causera le
plus grand stress aux forêts, mais plutôt l'augmentation de la fréquence
des feux, l'arrivée de nouveaux insectes ravageurs ou encore les
variations de températures printanières. De plus, tous ces facteurs de
risque peuvent interagir entre eux et mener à des changements rapides et
inattendus des écosystèmes forestiers. La grande incertitude associée
aux prédictions des effets attendus des changements climatiques doit
être intégrée dans la gestion forestière de façon à prendre en compte du
large éventail de vulnérabilités \citep{messier_dealing_2016}. Pour
permettre à l'écosystème de résister ou s'adapter à ces multiples
facteurs de stress, les politiques de sélection des espèces d'arbres
pourraient, par exemple, promouvoir le mélange d'espèces ayant des
caractéristiques fonctionnelles diverses, allant des tolérances
physiologiques (aux feux, aux ravageurs, à la sécheresse), aux modes de
régénération (e.g., banque de graines, cônes sérotineux, reproduction
végétative) et aux stratégies de croissance
\citep{messier_functional_2019, puettmann_critique_2009}.

\hypertarget{au-deluxe0-des-foruxeats-aplanir-la-courbe}{%
\section{Au-delà des forêts --- Aplanir la
courbe}\label{au-deluxe0-des-foruxeats-aplanir-la-courbe}}

\begin{quote}
Aujourd'hui, il faut revoir nos approches de prévention, de préparation
et d'intervention. Des plans particuliers orientés sur un aléa ou une
conséquence ont fait leurs temps. Nous gérons des conséquences multiples
et des effets domino à grandes échelles, qui s'enchaînent de manière de
plus en plus rapprochée.
\end{quote}

\begin{itemize}
\tightlist
\item
  L'Actualité, 26 avril 2020
\end{itemize}

Cet extrait de L'Actualité traitait de la gestion des risques associés
aux inondations printanières en pleine pandémie de COVID-19. Au cours
des premiers mois de 2020, et au moment où j'écris ces lignes, les
gouvernements du monde entier ont adopté des mesures draconiennes pour
tenter d'atténuer la menace de la COVID-19. Le message a été clair et
bien compris par l'ensemble de la population: il faut aplanir la courbe
en ralentissant le rythme de propagation de la maladie pour éviter de
dépasser la capacité des hôpitaux à traiter les malades.

Ce même concept s'applique également à la crise climatique: la
limitation du réchauffement climatique donnerait aux sociétés et aux
écosystèmes une plus grande marge de manoeuvre pour s'adapter sans
dépasser la capacité de support de la Terre et des systèmes humains. En
d'autres mots, si nous n'agissons pas maintenant pour décarboniser notre
économie, le climat mondial continuera de se dérégler et la
multiplication des catastrophes environnementales dépassera notre
capacité à les gérer. Le casse-tête de la gestion des inondations
pendant la pandémie de COVID-19 à Fort McMurray est un bon exemple des
difficultés à gérer plusieurs catastrophes en même temps. Les
catastrophes naturelles en cascade sont un autre exemple. Au cours des
dernières décennies, l'ouest du Canada a subi une épidémie sans
précédent de dendroctone du pin ponderosa, une grave sécheresse
(2001-2003) et des saisons de feux extrêmes
\citep{michaelian_massive_2011, williamson_climate_2009}. Ces
perturbations peuvent apparaître soudainement et avoir des effets
synergiques qui excédent la capacité du gouvernement à contrôler les
dommages.

Le récent rapport spécial du GIEC a conclu que limiter le réchauffement
climatique à 1.5 \(^{\circ}\)C est possible mais exigera des transitions
rapides et radicales dans tous les aspects de la société
\citep{ipcc_summary_2018}. La réponse à la pandémie montre que nous
pouvons mettre en place rapidement des mesures d'urgences radicales. Par
contre, cette crise sanitaire nous montre aussi qu'il est préférable
d'adopter des mesures préventives afin d'atténuer les changements
climatiques plutôt que d'attendre passivement de frapper un mur et être
forcé de vivre dans l'état d'urgence avec des mesures extrêmes.
